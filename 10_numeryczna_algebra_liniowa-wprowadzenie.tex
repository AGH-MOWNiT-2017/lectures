\documentclass[aspectratio=43,english]{beamer} %If you want to create Polish presentation, replace 'english' with 'polish' and uncomment 3-th line, i.e., '\usepackage{polski}'
\usepackage[utf8]{inputenc}
\usepackage{polski} %Uncomment for Polish language
\usepackage{babel}
\usepackage{listings} %We want to put listings

\mode<beamer>{ 	%in 'beamer' mode
	\hypersetup{pdfpagemode=FullScreen}		%Enable Full screen mode
	\usetheme{JuanLesPins} 		%Show part title in right footer
	%\usetheme[dark]{AGH}                 		%Use dark background
	%\usetheme[dark,parttitle=leftfooter]{AGH}  	%Use dark background and show part title in left footer
}
\mode<handout>{	%in 'handout' mode
	\hypersetup{pdfpagemode=None}		
	\usepackage{pgfpages}
  	\pgfpagesuselayout{4 on 1}[a4paper,border shrink=5mm,landscape]	%show 4 slides on 1 page
  	\usetheme{boxes}
  	\addheadbox{structure}{\quad\insertpart\hfill\insertsection\hfill\insertsubsection\qquad} 	%content of header
 	\addfootbox{structure}{\quad\insertauthor\hfill\insertframenumber\hfill\insertsubtitle\qquad} 	%content of footer
}

\AtBeginPart{ %At begin part: display its name
	\frame{\partpage}
} 


%%%%%%%%%%% Configuration of the listings package %%%%%%%%%%%%%%%%%%%%%%%%%%
% Source: https://en.wikibooks.org/wiki/LaTeX/Source_Code_Listings#Using_the_listings_package
%%%%%%%%%%%%%%%%%%%%%%%%%%%%%%%%%%%%%%%%%%%%%%%%%%%%%%%%%%%%%%%%%%%%%%%%%%%%
\lstset{ %
  backgroundcolor=\color{white},   % choose the background color
  basicstyle=\footnotesize,        % the size of the fonts that are used for the code
  breakatwhitespace=false,         % sets if automatic breaks should only happen at whitespace
  breaklines=true,                 % sets automatic line breaking
  captionpos=b,                    % sets the caption-position to bottom
  commentstyle=\color{green},      % comment style
  deletekeywords={...},            % if you want to delete keywords from the given language
  escapeinside={\%*}{*)},          % if you want to add LaTeX within your code
  extendedchars=true,              % lets you use non-ASCII characters; for 8-bits encodings only, does not work with UTF-8
  frame=single,	                   % adds a frame around the code
  keepspaces=true,                 % keeps spaces in text, useful for keeping indentation of code (possibly needs columns=flexible)
  keywordstyle=\color{blue},       % keyword style
  morekeywords={*,...},            % if you want to add more keywords to the set
  numbers=left,                    % where to put the line-numbers; possible values are (none, left, right)
  numbersep=5pt,                   % how far the line-numbers are from the code
  numberstyle=\tiny\color{gray},   % the style that is used for the line-numbers
  rulecolor=\color{black},         % if not set, the frame-color may be changed on line-breaks within not-black text (e.g. comments (green here))
  showspaces=false,                % show spaces everywhere adding particular underscores; it overrides 'showstringspaces'
  showstringspaces=false,          % underline spaces within strings only
  showtabs=false,                  % show tabs within strings adding particular underscores
  stepnumber=2,                    % the step between two line-numbers. If it's 1, each line will be numbered
  stringstyle=\color{cyan},        % string literal style
  tabsize=2,	                   % sets default tabsize to 2 spaces
  title=\lstname,                  % show the filename of files included with \lstinputlisting; also try caption instead of title
                                   % needed if you want to use UTF-8 Polish chars
  literate={?}{{\k{a}}}1
           {?}{{\k{A}}}1
           {?}{{\k{e}}}1
           {?}{{\k{E}}}1
           {�}{{\'o}}1
           {�}{{\'O}}1
           {?}{{\'s}}1
           {?}{{\'S}}1
           {?}{{\l{}}}1
           {?}{{\L{}}}1
           {?}{{\.z}}1
           {?}{{\.Z}}1
           {?}{{\'z}}1
           {?}{{\'Z}}1
           {?}{{\'c}}1
           {?}{{\'C}}1
           {?}{{\'n}}1
           {?}{{\'N}}1
}
%%%%%%%%%%%%%%%%%


\title{Metody Obliczeniowe w Nauce i Technice}
\author{Marian Bubak, PhD}
\date{}
\institute[AGH]{
	Institute of Computer Science\\ul. Kawiory 21\\30-055 Krakow\\
	Poland\\
	\url{http://www.icsr.agh.edu.pl/~mownit/}
}




\usepackage{tikz}
\usetikzlibrary{decorations.pathreplacing,calc}

\newcommand{\tikzmark}[1]{\tikz[overlay,remember picture] \node (#1) {};}

\newcommand*{\AddNote}[5]{
    \begin{tikzpicture}[overlay, remember picture]
        \draw [decoration={brace,amplitude=0.5em},decorate,ultra thick,red]
            ($(#3)!(#1.north)!($(#3)-(0,1)$)$) --
            ($(#3)!(#2.south)!($(#3)-(0,1)$)$)
                node [align=center, text width={#4}, pos=0.5, anchor=west] {#5};
    \end{tikzpicture}
}

%%%%%%%%

\subtitle{10. Numeryczna algebra liniowa -- wprowadzenie.}
\setcontributors{Magdalena Nowak\\Paweł Taborowski}


\begin{document}
  \maketitle
	\begin{frame}{Plan wykładu}
		\tableofcontents
	\end{frame}

  \section{Zastosowania numerycznej algebry liniowej}

\begin{frame}{Zastosowania numerycznej algebry liniowej}


  \begin{itemize}
    \item \textbf{Fizyka:} CFD, lattice-gauge, atomic spectra, \dots
    \item \textbf{Chemia:} chemia kwantowa, inżynieria chemiczna, 
    \item \textbf{Mechanika:} obliczenia z wykorzystaniem FEM, FD, \dots
    \item \textbf{Elektrotechnika, elektronika:} systemy energetyczne, symulacja obwodów, symulacja urządzeń, \dots
    \item \textbf{Geodezja,}
    \item \textbf{Ekonomia:} modelowanie systemów ekonomicznych,
    \item \textbf{Demografia:} migracja,
    \item \textbf{Psychologia:} grupy, powiązania,
    \item \textbf{Badania operacyjne:} programowanie liniowe,
    \item \textbf{Polityka}\dots
  \end{itemize}
\end{frame}

\section{Macierze testowe}
\begin{frame}{Macierze testowe}
% Lista wypunktowana była kłamstwem – to wszystko było opisem tylko tego jednego linku
\begin{block}{}
Standardowy zbiór macierzy rzadkich -- \\ 
\textit{Harwel - Boeing Sparse Matrix Collection} \cite{matrix}
\end{block}
\end{frame}


  \section{Dlaczego algorytmy równoległe?}

\begin{frame}{Dlaczego algorytmy równoległe?}

\begin{exampleblock}{Przykład -- przewidywanie pogody (globalnie):}
rozwiązywanie równania Naviera-Stokesa na 3D siatce wokół Ziemi

Zmienne:

  $\left. \parbox{10em}
{\begin{itemize}
   \item \text{temperatura}
   \item \text{ciśnienie}
   \item \text{wilgotność}
   \item \text{prędkość wiatru}
  \end{itemize}}
\right \} \text{równanie Naviera-Stokesa (6 zmiennych)}$
\end{exampleblock}

Obliczenia:

$\left. \parbox{15em}
{\begin{itemize}
   \item \text{elementarna komórka 1 km}
   \item \text{10 warstw}
  \end{itemize}}
\right \} 5\cdot10^9\text{ komórek}$

\begin{itemize}
    \item w każdej komórce:
	$6\cdot 8 \text{ Bytes} \Rightarrow 2\cdot 10^{11} \text{ Bytes} = 200 \text{ Gbytes}$,

    \item w każdej komórce 100 operacji fp,
    \item obliczenia 1 kroku czasowego 1 min $\Rightarrow \frac{100\cdot5\cdot10^9}{60} = 8 \text{ GFLOPS}$
 \end{itemize}

\end{frame}


  \section{Basic matrix algorithms}

  \subsection{Dot product}

\begin{frame}[fragile]{Dot product}


$ c = x^T\cdot y; \quad x,y \in R^n\text{ (vectors)} $ % \quad robi niezbyt duży odstęp w trybie matematycznym.

\vspace{10px}
 \begin{lstlisting}[language=Matlab] %Trochę więcej się koloruje przy ustawieniu Matlab, ale jeśli wolisz, wróć do Pascala.

function: c = dot(x,y)

   c = 0
   n = length(x)
 	for i = 1 : n
 		c = c + x(i)%*$\cdot$*)y(i)
 	end

 end dot
 \end{lstlisting}


\end{frame}

\subsection{saxpy}

\begin{frame}[fragile]{saxpy}

$ z = \alpha \cdot x + y == $ (saxpy = \textbf{s}calar \textbf{a}lpha \textbf{x} \textbf{p}lus \textbf{y})  $\alpha$ -- scalar % Tu chodziło o zaznaczenie pierwszych liter w celu wyjaśnienia nazwy
%Myślę też, że tamto "==" nie ma żadnego sensu, lepiej by było zamiast niego zrobić przejście do nowej lini.
\vspace{10px}
 \begin{lstlisting}[language=Matlab]

function: z = saxpy(%*$\alpha$*),x,y)

  n = length(x)
 	for i = 1 : n

 		z(i) = %*$\alpha \cdot \text{x(i)} + \text{y(i)} $*)

 	end

 end saxpy
 \end{lstlisting}

\end{frame}


\subsection{The colon notation}

\begin{frame}{The colon notation}

$ A\in R^{m\cdot n} $ \\
\vspace{10px}
$k$-th row of $A$: % Wyróżniłem k i A
$$A(k, :) = [a_{k1}, a_{k2}, \dots , a_{kn}]$$ % Zmieniłem kropki
$k$-th column of $A$:
\vspace{10px}

\begin{center}
$A(:, k) = \left [ \begin{array}{l}
   a_{1k} \\
   a_{2k}\\
   \vdots\\
   a_{nk}
\end{array}
\right ] $
\end{center}

\end{frame}


  \subsection{Matrix-Vector multiplication}

\begin{frame}[fragile]{Matrix-Vector multiplication}

$z = A \cdot x; $ \\
$z_i = \sum_{i=1}^n a_{i j}  x_j $ \\
\vspace{10px}
\begin{enumerate}[a)]
	\item row version \\
	$m = rows(A); \quad n = cols(A)$ \\
	$z(1:m) = 0$
\vspace{10px}

\begin{lstlisting}[language=Matlab, mathescape]
  function: z = matvec.ij(A, x)
     for i = 1:m $\tikzmark{mark1}$
         for j = 1:n
	       z(i) = z(i) + A(i,j)$\cdot$x(j) $\tikzmark{mark3}$
         end
     end $\tikzmark{mark2}$
   end matvec.ij\end{lstlisting}
 \AddNote{mark1}{mark2}{mark3}{0.7cm}{$*$} %{góra}{dół}{poziomo}{szerokość tekstu}{tekst}
\end{enumerate}
\end{frame}

\begin{frame}[fragile]
	\begin{itemize}
		\item $*$: in colon notation
		\vspace{7px}
		 \begin{lstlisting}[language=Matlab]
		   for i = 1:m
		       z(i) = A(i,:)%*$\cdot$*)x
		   end\end{lstlisting}

		 \item with dot
		 \vspace{7px}
		  \begin{lstlisting}[language=Matlab]
		    for i = 1:m
		        z(i) = dot(A(i,:),x)
		    end\end{lstlisting}
	\end{itemize}
 \begin{alertblock}{Ważne}
Dostęp do macierzy A $\Rightarrow$ wierszami
  \end{alertblock}

\end{frame}

\begin{frame}[fragile]
	\begin{enumerate}[b)]
		\item column version \vspace{10px}

$\left(  \begin{array}{ll}
   1 & 2\\
   3 & 4\\
   5 & 6\\
\end{array}
\right ) \cdot
\left ( \begin{array}{l}
   7\\
   8\\
\end{array}
\right ) =
\left ( \begin{array}{l}
   1 \cdot 7 + 2 \cdot 8\\
   3 \cdot 7 + 4 \cdot 8\\
   5 \cdot 7 + 6 \cdot 8\\
\end{array}
\right ) = 7 \cdot
\left ( \begin{array}{l}
   1 \\
   3 \\
   5
\end{array}
\right ) + 8 \cdot
 \left ( \begin{array}{l}
   2 \\
   4 \\
   6
\end{array}
\right )
$
\vspace{10px}

$m = rows(A); \quad n = cols(A) \quad z(1:m) = 0$
\vspace{10px}

 \begin{lstlisting}[language=Matlab, mathescape]
  function: z = matrec.ij(A, x)
     for j = 1:n $\tikzmark{mark1}$
         for i = 1:m
	       z(i) = z(i) + A(i,j) $\cdot$ x(j) $\tikzmark{mark3}$
         end
     end $\tikzmark{mark2}$
   end matrec.ij\end{lstlisting}
 \AddNote{mark1}{mark2}{mark3}{0.7cm}{$*$}
\end{enumerate}
\end{frame}

\begin{frame}[fragile]
	\begin{itemize}
	\item $*$: in colon notation
 \begin{lstlisting}[belowskip=-1.4 \baselineskip, language=Matlab]
   z(1:m) = 0
   for j = 1:n
      z = z + x(j)%*$\cdot$*)A(:,i)
   end\end{lstlisting}

\item with dot
 \begin{lstlisting}[belowskip=-1.4 \baselineskip, language=Matlab]
   z(1:m) = 0
   for j = 1:n
       z = saxpy(x(j), A(:,i),z)
   end\end{lstlisting}
 \end{itemize}
\vspace{24px}
 \begin{alertblock}{Ważne}
Dostęp do macierzy A $\Rightarrow$ kolumnami
  \end{alertblock}

\end{frame}


  
\subsection{The gaxpy computation}

\begin{frame}[fragile]{The gaxpy computation}

$z = y + A\cdot x; $ \\

$x \in R^n; $ \\
$y \in R^n; $ \\
$A \in R^{n\cdot n};$ \\
(gaxpy = \textbf{g}eneral \textbf{A} \textbf{x} \textbf{p}lus \textbf{y})

\vspace{10px}
\begin{lstlisting}[language=Matlab]

  function: z = gaxpy(A,x,y)
       n = cols(a); z = y
       for j = 1:n
           z = z + x(j)%*$\cdot$*)A(:,j)
       end
   end
 \end{lstlisting}

\end{frame}


\subsection{Outer product updates}

\begin{frame}[fragile]{Outer product updates}

$A \leftarrow A + x\cdot y^T; $ \\
$a_{ij} = a_{ij} + x_i\cdot y_j, \quad i = 1:m, \quad j = 1:n $ \\

\begin{itemize}
\item ``ij'' version:
\end{itemize}
\begin{lstlisting}[belowskip=-1.4 \baselineskip, language=Matlab]
       for i = 1:m
           A(i,:) = A(i,:) + x(i)%*$\cdot y^T$*)
       end\end{lstlisting}
A $\rightarrow$ wierszami


\begin{itemize}
\item ``ji'' version:
\end{itemize}
\begin{lstlisting}[belowskip=-1.4 \baselineskip, language=Matlab]
       for j = 1:n
           A(:,j) = A(:,j) + y(j)%*$\cdot $*)x
       end\end{lstlisting}
A $\rightarrow$ kolumnami \hfill (obie ,,saxpy - based'')

\end{frame}


 
\lstset{
  mathescape = true,
  basicstyle = \ttfamily}
\newcommand{\dollar}{\mbox{\textdollar}}

\subsection{Matrix -- matrix multiplication}

\begin{frame}[fragile]{Matrix -- matrix multiplication}

$C = A^{m\cdot r}\cdot B^{r\cdot n} $

$$ c_{i j} = \sum_{k=1}^r a_{i k}  x_{k j} $$
\begin{enumerate}[a)]
  \item dot version
\end{enumerate}

\scriptsize{
\begin{lstlisting}[belowskip=-1.4 \baselineskip, language=Matlab, mathescape]
function: C=matmat.ijk(A,B)
  m=rows(A); r=cols(A); n=cols(B); C(1:m,1:n)=0
  for i=1:m
    for j=1:n $\tikzmark{mark4}$
      for k=1:r $\tikzmark{mark1}$
        C(i,j)=C(i,j)+A(i,k) $\cdot$ B(k,j) $\tikzmark{mark3}$
      end $\tikzmark{mark2}$
    end $\tikzmark{mark5}$
  end
end matmat.ijk
\end{lstlisting}
}
\AddNote{mark1}{mark2}{mark3}{1.5cm}{inner loop $\tikzmark{mark6}$}
\AddNote{mark4}{mark5}{mark6}{1.6cm}{middle loop}
\end{frame}

\subsection{Matrix multiplication: loop ordering and properties}

\begin{frame}[fragile]{Matrix multiplication: loop ordering and properties}

$i j k \rightarrow 3! = 6 \text{ sposobów} $ \\ 

\vspace{5px}
\begin{tabular}{ |p{1cm}||p{1.6cm}||p{3.4cm}||p{3.7cm}|  }
 \hline

 loop order& Inner loop & Middle loop&Inner loop data access\\
 \hline
 ijk   & dot   &vector x matrix&   A by row, B by column\\
\hline
 jik&   dot & matrix x vector  &A by row, B by column\\
\hline
 ikj &saxpy & row gaxpy&  B by row\\
\hline
 jki    &saxpy & column gaxpy&  A by column\\
\hline
 kij &   saxpy  & row outer product& B by row\\
\hline
 kji& saxpy  & column outer product & A by column\\
 \hline
\end{tabular}

\begin{itemize}
\item Operations $\rightarrow$ dot, saxpy,
\item modes of access.
\end{itemize}
%link??
%Link ponownie jest adresem WZGLĘDNYM.
%Aplet demonstrujący powyższe algorytmy \\
%(autor: Dariusz Szabliński) ./applets/num_al_lin/mat.html \\

Wybór zależy od architektury komputera.
\end{frame}


\section{Bibliografia}
\begin{frame}{Bibliografia}
  \begin{thebibliography}{9}
    \setbeamertemplate{bibliography item}[online]
      \bibitem[HARWELL-BOEING]{matrix}{Matrix Market \newblock The Harwell-Boeing Sparse Matrix Collection \newblock \small{\url{http://math.nist.gov/MatrixMarket/data/Harwell-Boeing/}}}
  \end{thebibliography}
\end{frame}

\end{document}
