\section{Metoda zmiennej metryki}

\subsection{Metoda Newtona -- podsumowanie}
  \begin{frame}
    \begin{block}{Uogólniona met. Newtona -- znajdowanie minimum}
      \textbf{Założenia: } Dana jest funkcja $F$, taka że $F(x) \in \mathbb{C}^2$.
    \end{block}
    \begin{block}{Procedura iteracyjna}
      W k-tym kroku:
      \begin{equation}
        \left.
          \begin{aligned}
            x_{k + 1} &= x_{k} + \alpha_{k} \cdot d_{k}\\
            d_{k}     &= - V(x_k) \cdot \nabla F(x_{k})\\
            V(x_k)    &= H^{-1}(x_{k})
          \end{aligned}
       \right\}
      \end{equation}
    \end{block}
  \end{frame}

  \begin{frame}{}
    \begin{block}{Oznaczenia}
      $d_{k}$ -- wektor kierunku poszukiwań\\
      $\alpha_{k}$ -- długość kroku; dobrana tak, aby\\
      \hspace*{8mm} minimalizować funkcję $\alpha \mapsto F(x_{k} + \alpha \cdot d_{k}), \alpha \geq 0$\\
      $H(x_k) = \bigg[ \frac{\partial^2 F(x_{k})}{\partial x_{i} x_{j}} \bigg]$ -- hesjan funkcji $F(x_{k})$\\
      $G(x_k)$ -- hesjan formy kwadratowej $Q$ w punkcie $x_k$\\
      \begin{equation*}
        Q(x) = (Ax - c)^T \cdot (Ax - c)
      \end{equation*}
      $V(x_k) = G^{-1}(x_k)$ -- macierz kowariancji w k-tym kroku
    \end{block}

    \begin{block}{}
      \textbf{Istota: } $V(x_{k})$ -- obliczane w każdym kroku.\\
      Duży narzut obliczeniowy.
    \end{block}
  \end{frame}

  \begin{frame}{Optymalizacja metodami zmiennej metryki}
    \begin{block}{}
      \textbf{Cel: } Zmniejszenie narzutu obliczeniowego na wyznaczanie $V_k$.\\
      \textbf{Pomysł: } Wykorzystanie informacji z poprzednich kroków iteracyjnych.
    \end{block}

    \begin{block}{}
      \textbf{Szukamy: }
        \begin{equation}
          V_{k + 1} = V_k + C_k
        \end{equation}
      \textbf{Dysponujemy: } $x_k, x_{k + 1}, \nabla F(x_k), \nabla F(x_{k + 1})$
    \end{block}
  \end{frame}

  \begin{frame}{Kierunki sprzężone}
    \begin{block}{Kierunki wzajemnie sprzężone}
      \textbf{Def: } Kierunki $d_0, d_1, ..., d_i, ..., d_r, d_i \neq 0, r \leq n - 1$\\
      nazywamy wzajemnie sprzężonymi względem macierzy\\
      $A$ (A-sprzężonymi), jeśli:
      \begin{equation*}
        d_i^T A d_j = 0, i \neq j
      \end{equation*}
      Jest to uogólnienie def. ortogonalności:
      \begin{equation*}
        V^T \cdot u = V^T \cdot I \cdot u = 0
      \end{equation*}
      Wektory własne $A$ symetrycznej są wektorami A-sprzężonymi.
    \end{block}
  \end{frame}

  \begin{frame}{Minimalizacja w kierunkach sprzężonych}
    \begin{block}{Twierdzenie o minimalizacji w kierunkach sprzężonych}
      \textbf{Założenie: } $d_0, d_1, .., d_{n - 1}$ -- ciąg wektorów A-sprzężonych\\
        A -- macierz $(n \times n)$, symetryczna, dodatnio określona.\\
      \textbf{Twierdzenie: } Niezależnie od wyboru punktu początkowego $x_0$\\
       można zminimalizować formę kwadratową:
       \begin{equation}\label{eq:tw_1}
         Q(x) = Q(x_0) + b^T \cdot x + \frac{1}{2} x^T \cdot A \cdot x
       \end{equation}
       \begin{enumerate}
         \item w najwyżej $n$ krokach iteracyjnych\\
         \item kolejne minimalizacje -- wzdłuż $d_i$\\
         \item każdy wektor $d_i$ -- używany tylko jeden raz\\
         \item kolejność wprowadzenia $d_i$ do obliczeń -- obojętna
       \end{enumerate}
    \end{block}
  \end{frame}

  \begin{frame}{Minimalizacja w kierunkach sprzężonych -- dowód}
    $V$ -- dowolny wektor; $d_i$ -- liniowo niezależne
    \begin{equation}
      \begin{gathered}
        \begin{aligned}
          &V = \sum_{i=0}^{n - 1} \gamma_i \cdot d_i, &\gamma_i = \frac{d_i A v}{d_i^T A d_i}
        \end{aligned}
      \end{gathered}
    \end{equation}
    W kroku iteracyjnym $p$:
    \begin{equation*}
        \begin{aligned}
          &x_p = x_0 + \sum_{i = 0}^{p - 1} \alpha_i \cdot d_i, &p > 1
        \end{aligned}
    \end{equation*}
    $x_p$ -- kolejny punkt uzyskany w trakcie minimalizacji\\
    formy kw. $Q(x)$ wzdłuż kierunków $d_0, d_1, ..., d_{p - 1}$.
  \end{frame}

  \begin{frame}{Minimalizacja w kierunkach sprzężonych -- dowód cd.}
    \begin{enumerate}
      \item Z punktu $x_p$ do $x_{p + 1}$ przechodzimy wzdłuż $d_p$
      \item Forma $Q$ ma mieć minimum wzdłuż $d_p$
    \end{enumerate}
    Otrzymujemy więc, że:
    \begin{equation}\label{eq:d_n_q_zero}
        \begin{aligned}
          d_p^T \nabla Q(x_{p + 1}) = 0
        \end{aligned}
    \end{equation}
    Po podstawieniu $(\ref{eq:tw_1})$ do $(\ref{eq:d_n_q_zero})$:
    \begin{equation*}
        \begin{aligned}
          d_p^T \cdot A \cdot \bigg[(x_0 + \sum_{i = 0}^{p - 1} \alpha_i \cdot d_i) + \alpha_p \cdot d_p \bigg] + d_p^T \cdot b = 0
        \end{aligned}
    \end{equation*}
    Z czego wyznaczamy stałą:
    \begin{equation*}
        \begin{aligned}
          \alpha_p = - \frac{d_p^T (A \cdot x_0 + b)}{d_p^T \cdot A \cdot d_p}
        \end{aligned}
    \end{equation*}
  \end{frame}

  \begin{frame}{Minimalizacja w kierunkach sprzężonych -- dowód cd.}
    W konsekwencji po co najwyżej $n$ krokach, (bo może być $\alpha_i = 0)$ zachodzi:
    \begin{equation*}
        \begin{aligned}
          x_n = x_0 - \sum_{i = 0}^{n - 1} \frac{d_i^T (A \cdot x_0 + b) d_i}{d_i^T \cdot A \cdot d_i} =
          x_0 - \underbrace{\sum_{i = 0}^{n - 1} \frac{d_i^T \cdot A (x_0 + A^{-1} b) d_i}{d_i^T \cdot A \cdot d_i}}_{x_0 + A^{-1} \cdot b}\\
          x_n = x_0 - x_0 - A^{-1} \cdot b = - A^{-1} \cdot b
        \end{aligned}
    \end{equation*}
    Osiągnęliśmy min. formy kwadratowej:
    \begin{enumerate}
      \item niezależnie od kolejności wystąpień $d_i$,
      \item w co najwyżej $n$ krokach minimalizacyjnych
    \end{enumerate}
    Met. kier. sprzężonych prowadzi do \textbf{zbieżności kwadratowej}.
  \end{frame}
