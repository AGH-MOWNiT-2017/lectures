\section{Metoda zmiennej metryki}

\subsection{Metoda Newtona -- podsumowanie}
  \begin{frame}
    \begin{block}{Uogólniona met. Newtona -- znajdowanie minimum}
      \textbf{Założenia: } Dana jest funkcja $F$, taka że $F(x) \in \mathbb{C}^2$.
    \end{block}
    \begin{block}{Procedura iteracyjna}
      W k-tym kroku:
      \begin{equation}\label{eq:def_0}
        \left.
          \begin{aligned}
            x_{k + 1} &= x_{k} + \alpha_{k} \cdot d_{k}\\
            d_{k}     &= - V(x_k) \cdot \nabla F(x_{k})\\
            V(x_k)    &= H^{-1}(x_{k})
          \end{aligned}
       \right\}
      \end{equation}
    \end{block}
  \end{frame}

  \begin{frame}{}
    \begin{block}{Oznaczenia}
      $d_{k}$ -- wektor kierunku poszukiwań\\
      $\alpha_{k}$ -- długość kroku; dobrana tak, aby\\
      \hspace*{8mm} minimalizować funkcję $\alpha \mapsto F(x_{k} + \alpha \cdot d_{k}), \alpha \geq 0$\\
      $H(x_k) = \bigg[ \frac{\partial^2 F(x_{k})}{\partial x_{i} x_{j}} \bigg]$ -- hesjan funkcji $F(x_{k})$\\
      $G(x_k)$ -- hesjan formy kwadratowej $Q$ w punkcie $x_k$\\
      \begin{equation*}
        Q(x) = (Ax - c)^T \cdot (Ax - c)
      \end{equation*}
      $V(x_k) = G^{-1}(x_k)$ -- macierz kowariancji w k-tym kroku
    \end{block}

    \begin{block}{}
      \textbf{Istota: } $V(x_{k})$ -- obliczane w każdym kroku.\\
      Duży narzut obliczeniowy.
    \end{block}
  \end{frame}

  \begin{frame}{Optymalizacja metodami zmiennej metryki}
    \begin{block}{}
      \textbf{Cel: } Zmniejszenie narzutu obliczeniowego na wyznaczanie $V_k$.\\
      \textbf{Pomysł: } Wykorzystanie informacji z poprzednich kroków iteracyjnych.
    \end{block}

    \begin{block}{}
      \textbf{Szukamy: }
        \begin{equation}\label{eq:v_k_c}
          V_{k + 1} = V_k + C_k
        \end{equation}
      \textbf{Dysponujemy: } $x_k, x_{k + 1}, \nabla F(x_k), \nabla F(x_{k + 1})$
    \end{block}
  \end{frame}

  \begin{frame}{Kierunki sprzężone}
    \begin{block}{Kierunki wzajemnie sprzężone}
      \textbf{Def: } Kierunki $d_0, d_1, ..., d_i, ..., d_r, d_i \neq 0, r \leq n - 1$\\
      nazywamy wzajemnie sprzężonymi względem macierzy\\
      $A$ (A-sprzężonymi), jeśli:
      \begin{equation*}
        d_i^T A d_j = 0, i \neq j
      \end{equation*}
      Jest to uogólnienie def. ortogonalności:
      \begin{equation*}
        V^T \cdot u = V^T \cdot I \cdot u = 0
      \end{equation*}
      Wektory własne $A$ symetrycznej są wektorami A-sprzężonymi.
    \end{block}
  \end{frame}

  \begin{frame}{Minimalizacja w kierunkach sprzężonych}
    \begin{block}{Twierdzenie o minimalizacji w kierunkach sprzężonych}
      \textbf{Założenie: } $d_0, d_1, .., d_{n - 1}$ -- ciąg wektorów A-sprzężonych\\
        A -- macierz $(n \times n)$, symetryczna, dodatnio określona.\\
      \textbf{Twierdzenie: } Niezależnie od wyboru punktu początkowego $x_0$\\
       można zminimalizować formę kwadratową:
       \begin{equation}\label{eq:tw_1}
         Q(x) = Q(x_0) + b^T \cdot x + \frac{1}{2} x^T \cdot A \cdot x
       \end{equation}
       \begin{enumerate}
         \item w najwyżej $n$ krokach iteracyjnych\\
         \item kolejne minimalizacje -- wzdłuż $d_i$\\
         \item każdy wektor $d_i$ -- używany tylko jeden raz\\
         \item kolejność wprowadzenia $d_i$ do obliczeń -- obojętna
       \end{enumerate}
    \end{block}
  \end{frame}

  \begin{frame}{Minimalizacja w kierunkach sprzężonych -- dowód}
    $V$ -- dowolny wektor; $d_i$ -- liniowo niezależne
    \begin{equation}
      \begin{gathered}
        \begin{aligned}
          &V = \sum_{i=0}^{n - 1} \gamma_i \cdot d_i, &\gamma_i = \frac{d_i A v}{d_i^T A d_i}
        \end{aligned}
      \end{gathered}
    \end{equation}
    W kroku iteracyjnym $p$:
    \begin{equation*}
        \begin{aligned}
          &x_p = x_0 + \sum_{i = 0}^{p - 1} \alpha_i \cdot d_i, &p > 1
        \end{aligned}
    \end{equation*}
    $x_p$ -- kolejny punkt uzyskany w trakcie minimalizacji\\
    formy kw. $Q(x)$ wzdłuż kierunków $d_0, d_1, ..., d_{p - 1}$.
  \end{frame}

  \begin{frame}{Minimalizacja w kierunkach sprzężonych -- dowód cd.}
    \begin{enumerate}
      \item Z punktu $x_p$ do $x_{p + 1}$ przechodzimy wzdłuż $d_p$
      \item Forma $Q$ ma mieć minimum wzdłuż $d_p$
    \end{enumerate}
    Otrzymujemy więc, że:
    \begin{equation}\label{eq:d_n_q_zero}
        \begin{aligned}
          d_p^T \nabla Q(x_{p + 1}) = 0
        \end{aligned}
    \end{equation}
    Po podstawieniu $(\ref{eq:tw_1})$ do $(\ref{eq:d_n_q_zero})$:
    \begin{equation*}
        \begin{aligned}
          d_p^T \cdot A \cdot \bigg[(x_0 + \sum_{i = 0}^{p - 1} \alpha_i \cdot d_i) + \alpha_p \cdot d_p \bigg] + d_p^T \cdot b = 0
        \end{aligned}
    \end{equation*}
    Z czego wyznaczamy stałą:
    \begin{equation*}
        \begin{aligned}
          \alpha_p = - \frac{d_p^T (A \cdot x_0 + b)}{d_p^T \cdot A \cdot d_p}
        \end{aligned}
    \end{equation*}
  \end{frame}

  \begin{frame}{Minimalizacja w kierunkach sprzężonych -- dowód cd.}
    W konsekwencji po co najwyżej $n$ krokach, (bo może być $\alpha_i = 0)$ zachodzi:
    \begin{equation*}
        \begin{aligned}
          x_n = x_0 - \sum_{i = 0}^{n - 1} \frac{d_i^T (A \cdot x_0 + b) d_i}{d_i^T \cdot A \cdot d_i} =
          x_0 - \underbrace{\sum_{i = 0}^{n - 1} \frac{d_i^T \cdot A (x_0 + A^{-1} b) d_i}{d_i^T \cdot A \cdot d_i}}_{x_0 + A^{-1} \cdot b}\\
          x_n = x_0 - x_0 - A^{-1} \cdot b = - A^{-1} \cdot b
        \end{aligned}
    \end{equation*}
    Osiągnęliśmy min. formy kwadratowej:
    \begin{enumerate}
      \item niezależnie od kolejności wystąpień $d_i$,
      \item w co najwyżej $n$ krokach minimalizacyjnych
    \end{enumerate}
    Met. kier. sprzężonych prowadzi do \textbf{zbieżności kwadratowej}.
  \end{frame}

  \begin{frame}{Minimalizacja w kierunkach sprzężonych 2}
    \begin{block}{Twierdzenie 2}
      \textbf{Założenie: } $x_{k + 1}$ -- kolejny punkt uzyskany w minimalizacji\\
      formy $(\ref{eq:tw_1})$ wzdłuż A-sprzężonych kierunków $d_0, d_1, ..., d_k$.\\
      \textbf{Twierdzenie: } Zachodzi równanie:
      \begin{equation}\label{eq:tw_2}
          \begin{aligned}
            &d_l^T \nabla Q(x_{k + 1}) = 0, &l = 0, 1, ..., k
          \end{aligned}
      \end{equation}
    \end{block}

    \begin{block}{}
        \textbf{Obserwacja: } Gradient formy kwadratowej w punkcie $x_p$ "pamięta"
        wszystkie poprzednie kierunki sprzężone\\ prowadzące do $x_p$ z $x_0$.
    \end{block}
  \end{frame}

  \begin{frame}{Minimalizacja w kierunkach sprzężonych 2 -- dowód}
    \begin{equation*}
      \begin{aligned}
        \nabla Q_{k + 1}  &= A \cdot x_{k + 1} + b = A \cdot \bigg(x_s + \sum_{j = s + 1}^k \alpha_j d_j \bigg) + b =\\
                          &= \nabla Q_{s + 1} + \sum_{j = s + 1}^{k} \alpha_j A d_j
      \end{aligned}
    \end{equation*}
    W konsekwencji:
    \begin{equation*}
      \begin{aligned}
        d_s^T \nabla Q_{s + 1} = \underbrace{d_s^T \nabla Q_{s + 1}}_{= 0 \text{ (wzór \ref{eq:d_n_q_zero})}}
        + \underbrace{\sum_{j = s + 1}^k \alpha_j d_s^T A d_j}_{= 0}
      \end{aligned}
    \end{equation*}
    Słuszne jest więc równanie $(\ref{eq:tw_2})$.
  \end{frame}

  \begin{frame}{Formuła Davidona-Fletchera-Powella}
      Dla macierzy $V = G^{-1}$:\\
      \begin{enumerate}
        \item $(n \times n)$\\
        \item symetrycznej\\
        \item dodatnio określonej
      \end{enumerate}
      wektorów $d_0, d_1, ..., d_{n - 1}$ wyznaczającej kierunki sprzężone wzgl. $G$\\
      zachodzi:
      \begin{equation}
        \begin{aligned}
          V = \sum_{i = 0}^{n - 1} \frac{d_i d_i^T}{d_i^T G d_i}
        \end{aligned}
      \end{equation}
  \end{frame}

  \begin{frame}{Formuła Davidona-Fletchera-Powella 2}
    Bierzemy element tej sumy jako pierwsze przybliżenie do szukanej
    poprawki $C_k$ (wzór $\ref{eq:v_k_c}$):
    \begin{equation}\label{eq:v_k1_k}
      \begin{aligned}
        &V_{k + 1} = V_k + \frac{d_k d_k^T}{d_k^T G d_k} + B_k, &B_k = B_k^T
      \end{aligned}
    \end{equation}
    $B_k$ -- z warunku: $d_0, d_1, ..., d_k$ -- kierunki G-sprzężone.
  \end{frame}


  \begin{frame}{Formuła Davidona-Fletchera-Powella 3}
    Przymując:
    \begin{equation*}
      \begin{aligned}
        V_k G d_i = d_i, i = 0, 1, ..., k
      \end{aligned}
    \end{equation*}
    dla $d_0, d_1, ..., d_k$ będących G-sprzężonymi.\\
    Mamy (wzór \ref{eq:def_0}):
    \begin{equation*}
      \begin{aligned}
        -d_i^T G d_{k + 1} &= d_i^T G V_{k + 1} \nabla Q(x_{k + 1}) =\\
        = \bigg[ \nabla^T Q_{k + 1} V_{k + 1} G d_i \bigg]^T
        &= d_i^T \nabla Q_{k + 1} = 0 \text{, } i = 0, 1, ..., k
      \end{aligned}
    \end{equation*}
    Jest to wzór $(\ref{eq:tw_2})$.\\
    Wniosek: $d_{k + 1}$ -- również sprzężony do poprzednich.
  \end{frame}

  \begin{frame}{Formuła Davidona-Fletchera-Powella 4}
    Żądamy, aby:
    \begin{equation*}
      \begin{aligned}
        &V_{k + 1} G d_i = d_i, &i = 0, 1, ..., k+1
      \end{aligned}
    \end{equation*}

    Podstawiamy $V_{k + 1}$ z równania $(\ref{eq:v_k1_k})$ dla $i = k$.
    Otrzymujemy:\\
    \begin{equation*}
      \begin{aligned}
        \bigg(V_k + \frac{d_k d_k^T}{d_k^T G d_k} + B_k \bigg) G d_k = d_k
      \end{aligned}
    \end{equation*}

    Po przekształceniu:\\
    \begin{equation*}
      \begin{aligned}
        (V_k + B_k) G d_k = 0
      \end{aligned}
    \end{equation*}

    Co po uwzgl. wzorów $(\ref{eq:d_n_q_zero})$ i $(\ref{eq:def_0})$ daje:\\
    \begin{equation}
      \begin{aligned}
        \big[ \nabla Q_{k + 1} - \nabla Q_k \big]^T (V_k + B_k) = 0
      \end{aligned}
    \end{equation}
  \end{frame}

  \begin{frame}{Formuła Davidona-Fletchera-Powella 5}
    $B_k$ powinna mieć postać:
    \begin{equation*}
      \begin{aligned}
        B_k = - \frac{V_k \gamma_k \gamma_k^T V_k}{\gamma_k^T V_k \gamma_k}
      \end{aligned}
    \end{equation*}
    \begin{equation}
      \begin{aligned}
        \gamma_k = \nabla Q(x_{k + 1}) - \nabla Q(x_k)
      \end{aligned}
    \end{equation}
    \begin{equation}
      \begin{aligned}
        \delta_k \equiv x_{k + 1} - x_k \stackrel{(\text{wzór } \ref{eq:def_0})}{=} -\alpha_k G^{-1} \nabla Q_k = \alpha_k d_k
      \end{aligned}
    \end{equation}
    Otrzymujemy:
    \begin{equation*}
      \begin{aligned}
        d_k^T G d_k = \frac{(\alpha_{k + 1} - x_k)^T}{\alpha_k} G d_k = \frac{\gamma_k^T d_k}{\alpha_k}
      \end{aligned}
    \end{equation*}
    Gdyż dla formy kwadratowej:
    \begin{equation}
      \begin{aligned}
        \delta_k = G^{-1} \gamma_k
      \end{aligned}
    \end{equation}
  \end{frame}

  \begin{frame}{Metoda zmiennej metryki}
    \begin{block}{Wzór korekcyjny}
        Wzór korekcyjny ma postać:
        \begin{equation*}
          \begin{aligned}
            V_{k + 1} = V_k + \frac{\delta_k \delta_k^T}{\gamma_k^T \delta_k} - \frac{V_k \gamma_k \gamma_k^T  V_k}{\gamma_k^T  V_k \gamma_k}
          \end{aligned}
        \end{equation*}
    \end{block}
  \end{frame}

  \begin{frame}{Metoda zmiennej metryki -- przykład}
    \begin{equation*}
      \begin{rcases}
        \text{Geometria różniczkowa} \\
        \text{Ogólna teoria wzgl.}
      \end{rcases}
      \text{ Własności F}
    \end{equation*}
    F -- nieeuklidesowa przestrzeni zmiennych $x=(x_1, x_2, ..., x_n)$.\\

    Podst. niezmiennik:
    \begin{equation*}
      \begin{aligned}
        &ds^2 = dx^T A dx &\text{(kwadrat elem. długości)}
      \end{aligned}
    \end{equation*}

    \begin{equation*}
      \begin{aligned}
        dx - \text{przesunięcie, } x &\rightarrow x + dx \\
        A &\rightarrow \text{kowariantny tensor metryczny}
      \end{aligned}
    \end{equation*}
    \begin{equation*}
      \begin{aligned}
        \text{Dla } Q &\rightarrow A - \text{wszędzie stałe}\\
                    F &\rightarrow \text{przestrzeń o zmiennej metryce}
      \end{aligned}
    \end{equation*}
  \end{frame}
