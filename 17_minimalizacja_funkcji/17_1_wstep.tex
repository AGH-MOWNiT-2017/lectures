\section{Wstęp}

  \begin{frame}{Wstęp}
    \begin{block}{Romuald Wit}
      \begin{enumerate}
        \item Wstęp do metod optymalizacji nieliniowej - skrypt
        UJ, 1983
        \item Metody programowania nieliniowego WNT, Biblioteka
        Inżynierii Oprogramowania.
      \end{enumerate}
    \end{block}

  \end{frame}

  \begin{frame}{Wstęp}
    \begin{block}{Linki do stron związanych z minimalizacją:}
      \begin{itemize}
        \item MINUIT -- Function Minimization and Error
        Analysis\footnote{\url{http://consult.cern.ch/writeups/minuit}}
        % TODO: change numeration of footnotes from letters to numbers
        (autor:~F.~Jarnes)
        \item Metody optymalizacji\footnote{\url{http://aquarium.ia.agh.edu.pl/labs/opt/metopt.htm}}
        (autor:~Strona~prywatna)
        \item Przykłady ciekawych wielowymiarowych
        funkcji\footnote{\url{http://www.maths.adelaide.edu.au/Applied/llazausk/alife/realopt.htm}}
        (autor:~John~Smith)
      \end{itemize}
    \end{block}

  \end{frame}

\subsection{Motywacja}

  \begin{frame}{Motywacja}
    Szeroka klasa zagadnień - szukanie najmniejszej wartości
    przyjmowanej przez funkcję jednej lub wielu zmennych.

    \begin{exampleblock}{Przykład}
      \begin{itemize}
        \item minimalizacja kosztów producji
        \item \ldots
        \item linia geodezyjna w ogólnej teorii względności
      \end{itemize}
    \end{exampleblock}

  \end{frame}

  \begin{frame}{Motywacja}
    \begin{exampleblock}{Przykład -- c.d.}
      \begin{itemize}
        \item przykład klasyczny:\\
        \textbf{estymacja} parametrów modelu teoretycznego
        przez \textbf{minimalizację} różnicy między danymi
        teoretycznymi a eksperymentalnymi:\\
        $
          min!F(x) = \sum_{k/1}^{k} \left[ \frac{Y_k - T_k(\vec x)}{\sigma_k}
          \right] ^2 Y_k \text{-- zmierzone wartości,}
        $ \\
        $ \sigma_k $ - błąd pomiaru,\\
        $ T_k $ - wartości teoretyczne, zależne od parametrów~
        $ x_i{,}\ i=1{,} \dots {,}n $ \\
        $ K \geq n{,}\ \text{zwykle}\ K \gg n $
      \end{itemize}
    \end{exampleblock}

  \end{frame}

\subsection{Terminologia}

  \begin{frame}{Terminologia}
    \begin{block}{Terminologia}
      \begin{itemize}
        \item tradycyjnie: \textbf{minimalizacja}
        \item maksymalizacja $ F = -f(x)$
        \item optymalizacja -- kolizija z metodą teorii
        sterowania, (oparta o rachunek wariacyjny)
        \item b. stary termin: programowanie (liniowe,
        nieliniowe, matematyczne)
        \item ekstremalizacja
        \item hill-climbing \\
        \textbf{function minimalization !!!}
      \end{itemize}
    \end{block}

  \end{frame}

\subsection{Zdefiniowanie zagadnienia}

  \begin{frame}{Zdefiniowanie zagadnienia}
    \begin{block}{Zdefiniowanie zagadnienia}
      \textbf{Mając daną funkcę $ F(x) $ znaleźć wartość
      zmiennej $ x $, dla której $ F(x) $ przyjmuje
      najmniejszą wartość, przy czym:}
      \begin{enumerate}
        \item $ F(x) $ nie musi być zadana analitycznie --
        tylko: dla każdego $ x $ znamy jej wartość $ F(x) $
        \item wartości $ x $ mogą być ograniczone do pewnego
        obszaru -- constrained minization
        \item mogą być dostępne $ \frac{\partial F}{\partial x} $
        \item $ f(x) $ jest obliczana w wielu punktach --
        aż do osiągnięcia minimum. ($ F(x) $ -- to procedura)
      \end{enumerate}
    \end{block}

  \end{frame}

  \begin{frame}{Zdefiniowanie zagadnienia}
    \begin{block}{Najlepsza metoda -- taka, która:}
      \begin{itemize}
        \item znajduje minimum (z zadaną tolerancją)
        po najmniejszej liczbie obliczeń funkcji
        \item ma małą złożoność obliczeniową (sama
        metoda)
        \item wymaga mało pamięci
      \end{itemize}

    \end{block}

  \end{frame}

\subsection{Gdzie szukać minimum funcji? (global minimum)}

  \begin{frame}{Gdzie szukać minimum funcji? (global minimum)}
    \begin{block}{$ F(x) $ przyjmuje minimum w jednym z punktów:}
      \begin{enumerate}
        \item wszyskie $ \frac{\partial F}{\partial x} = 0 $
        (p. stacjonarny -- stationary point)
        \item niektóre $ \frac{\partial F}{\partial x} $
        nie istnieją (wierzchołek -- cusp)
        \item na krawędzi obszaru (edge point)
      \end{enumerate}
    \end{block}

    Gdy nie znamy funkcji analitycznie -- musimy ograniczyć~
    się do minimum lokalnego $ x_0 $:
    \begin{displaymath}
      \forall x \in \text{otoczenia}\ x_0{,}\ F(x) > F(x_0)
      \to \text{\emph{unconstrained local minimization}}
    \end{displaymath}

  \end{frame}

\subsection{Ksztalt funkcji $ F(x) $}
  \begin{frame}{Ksztalt funkcji $ F(x) $}
    \begin{block}{Zalożenie}
      $ F(x) $ -- ma sens fizyczny $ \Rightarrow $ w rozpatrywanym
      obszarze istnieją jej wszystkie pochodne.
    \end{block}

  \end{frame}
