\section{Wstęp}

\subsection{Motywacja}

  \begin{frame}{Motywacja}
    Szeroka klasa zagadnień -- szukanie najmniejszej wartości
    przyjmowanej przez funkcję jednej lub wielu zmennych.

    \begin{exampleblock}{Przykład}
      \begin{itemize}
        \item minimalizacja kosztów producji
        \item \ldots
        \item linia geodezyjna w ogólnej teorii względności
      \end{itemize}
    \end{exampleblock}
  \end{frame}

  \begin{frame}{Motywacja}
    \begin{exampleblock}{Przykład cd.}
      \begin{itemize}
        \item przykład klasyczny:\\
        \textbf{estymacja} parametrów modelu teoretycznego
        przez \textbf{minimalizację} różnicy między danymi
        teoretycznymi a eksperymentalnymi:\\
        $
          min!F(x) = \sum_{k/1}^{k} \left[ \frac{Y_k - T_k(\vec x)}{\sigma_k}
          \right] ^2 Y_k \text{-- zmierzone wartości,}
        $ \\
        $ \sigma_k $ - błąd pomiaru,\\
        $ T_k $ - wartości teoretyczne, zależne od parametrów~
        $ x_i{,}\ i=1{,} \dots {,}n $ \\
        $ K \geq n{,}\ \text{zwykle}\ K \gg n $
      \end{itemize}
    \end{exampleblock}
  \end{frame}

\subsection{Terminologia}

  \begin{frame}{Terminologia}
    \begin{block}{Terminologia}
      \begin{itemize}
        \item tradycyjnie: \textbf{minimalizacja}
        \item maksymalizacja $(F = -f(x))$
        \item optymalizacja -- kolizija z metodą teorii
        sterowania, (oparta o rachunek wariacyjny)
        \item b. stary termin: programowanie (liniowe,
        nieliniowe, matematyczne)
        \item ekstremalizacja
        \item hill-climbing \\
        \textbf{function minimalization !!!}
      \end{itemize}
    \end{block}
  \end{frame}

\subsection{Zdefiniowanie zagadnienia}

  \begin{frame}{Zdefiniowanie zagadnienia}
    \begin{block}{Zdefiniowanie zagadnienia}
      \textbf{Mając daną funkcę $ F(x) $ znaleźć wartość
      zmiennej $ x $, dla której $ F(x) $ przyjmuje
      najmniejszą wartość, przy czym:}
      \begin{enumerate}
        \item $ F(x) $ nie musi być zadana analitycznie --
        tylko: dla każdego $ x $ znamy jej wartość $ F(x) $
        \item wartości $ x $ mogą być ograniczone do pewnego
        obszaru -- constrained minization
        \item mogą być dostępne $ \frac{\partial F}{\partial x} $
        \item $ f(x) $ jest obliczana w wielu punktach --
        aż do osiągnięcia minimum. ($ F(x) $ -- to procedura)
      \end{enumerate}
    \end{block}
  \end{frame}

  \begin{frame}{Zdefiniowanie zagadnienia}
    \begin{block}{Najlepsza metoda -- taka, która:}
      \begin{itemize}
        \item znajduje minimum (z zadaną tolerancją)
        po najmniejszej liczbie obliczeń funkcji
        \item ma małą złożoność obliczeniową (sama
        metoda)
        \item wymaga mało pamięci
      \end{itemize}
    \end{block}
  \end{frame}

\subsection{Gdzie szukać minimum funcji? (global minimum)}

  \begin{frame}{Gdzie szukać minimum funcji? (global minimum)}
    \begin{block}{$ F(x) $ przyjmuje minimum w jednym z punktów:}
      \begin{enumerate}
        \item wszystkie $ \frac{\partial F}{\partial x} = 0 $
        (p. stacjonarny -- stationary point)
        \item niektóre $ \frac{\partial F}{\partial x} $
        nie istnieją (wierzchołek -- cusp)
        \item na krawędzi obszaru (edge point)
      \end{enumerate}
    \end{block}
    Gdy nie znamy funkcji analitycznie -- musimy ograniczyć
    się do minimum lokalnego $ x_0 $:
    \begin{displaymath}
      \forall x \in \text{otoczenia}\ x_0{,}\ F(x) > F(x_0)
      \to \text{\emph{unconstrained local minimization}}
    \end{displaymath}
  \end{frame}

\subsection{Ksztalt funkcji $ F(x) $}
  \begin{frame}{Ksztalt funkcji $ F(x) $}
    \begin{block}{Zalożenie}
      $ F(x) $ -- ma sens fizyczny $ \Rightarrow $ w rozpatrywanym
      obszarze istnieją jej wszystkie pochodne.
    \end{block}
    Wokół $ x_1 - F(x) $ -- \textbf{w szereg Taylora (1-D):}
    \begin{displaymath}
      F(x) = F(x_1) + \left. \frac{\partial F}{\partial x} \right|_{x_1}(x - x_1) +
      \left. \frac{1}{2} \frac{\partial^2 F}{\partial x^2} \right|_{x_1}(x - x_1)^2 +
      \dots
    \end{displaymath}
    \begin{itemize}
      \item a priori nie znamy obszaru jego zbieżności
      \item im mniejsza $ (x - x_1) $ - tym mniej ważne człony
      wyższych rzędów
    \end{itemize}
    $ \Rightarrow $ dla małych kroków -- przewidywania oparte
    na wyrazach niższego rzędu powinny być wystarczające.

  \end{frame}

  \begin{frame}{Ksztalt funkcji $ F(x) $}
    dla n-D: $ x \to \vec{x} $
    \begin{displaymath}
      F(\vec{x}) = \underbrace{F(\vec{x_1})}_{\text{stały}} +
      \vec{g}^T * (\vec{x} - \vec{x_1}) +
      \frac{1}{2}(\vec{x} - \vec{x_1}^T) * G * (\vec{x} - \vec{x_1}) + \dots
    \end{displaymath}
    \begin{displaymath}
      g_i = \left. \frac{\partial F}{\partial x} \right|_{\vec{x_1}} \text{-- gradient;}\quad
      G_{ij} = \left. \frac{\partial^2 F}{\partial x_i \partial y_j} \right|_{\vec{x_1}}
    \end{displaymath}
    symbole $ \Rightarrow $ ($ x_i $ -- składowa; $ \vec{x_i} $
    -- wektor; niekiedy zamiast $ \vec{x} $ -- samo~$ x $)

  \end{frame}

  \begin{frame}{Ksztalt funkcji $ F(x) $}
    $ \vec{g}^T * (\vec{x} - \vec{x_1}) $ -- proporcjonalny do
    gradientu, wskazuje kierunek największego spadku funkcji,
    w pobliżu minimum: $ \vec{g} \to 0{,}\ (\vec{x} - \vec{x_1}) \to 0
    \; \Rightarrow $ człon liniowy, nie przepowiada minimum,
    nie można go użyć do określenia wielkości kroku.
    $ \frac{1}{2}((\vec{x} - \vec{x_1})^T) * G * (\vec{x} - \vec{x_1}) $
    -- kwadratowy, najniższy człon przydatny do przewidywania minimum
    $ G \approx $ stale w małych obszarach
    \begin{alertblock}{Uwaga}
      Ta analiza nie jest słuszna dla $ F(\vec{x}) $ liniowych
      od $ \vec{x} $;\\
      \textbf{Wtedy:} programowanie liniowe, rozwiązanie --
      krawędzi obszaru ograniczeń.
    \end{alertblock}
  \end{frame}

  \begin{frame}{Ksztalt funkcji $ F(x) $}
    \begin{block}{Optymalne algorytmy minimalizacji}
      Nie ma uniwersalnej metody minimalizacji. Dla b.~złego
      algorytmu można znaleźć $ F(x) $, którą minimalizuje on
      najszybcziej -- i~odwrodnie.\\
      \textbf{Zasada:} dobór algorytmu do funkcji.

    \end{block}

  \end{frame}
