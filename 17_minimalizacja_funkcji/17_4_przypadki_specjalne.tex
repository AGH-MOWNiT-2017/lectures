\section{Przypadki specjalne}

\subsection{Liniowy model}
  \begin{frame}{Liniowy model}
    \begin{block}{Minimalizacja $\chi^{2}$ (chisquare minimization, least squares fitting)}
      \begin{displaymath}
        min!F(\vec{x}) = \sum_{k=1}^{k} \left[ \frac{Y_{k} - T_{k}(\vec{x})}{\sigma_{k}} \right]^{2} = \sum_{k=1}^{k} f_{k}^{2}(\vec{x})
      \end{displaymath}
      gdzie: $Y_{k}$ -- wartości zamierzone, $\sigma_{k}$ -- błąd,\\
      $T_{k}$ -- wartości teoretyczne; $k \geq n$, zwykle $k \gg n$ \\
      \begin{tabular}{|c|} \hline
        min$F(\vec{x})$ \\ \hline
      \end{tabular}
      $\Rightarrow$ najlepsze estymaty parametrów $\vec{x}$
      (modelu teoretycznego)
    \end{block}

  \end{frame}

  \begin{frame}{Liniowy model}
    \begin{equation}
      (*)\ \frac{\partial^2 F}{\partial x_{i} \partial x_{j}} =
      \frac{\partial}{\partial x_{i}} \frac{\partial}{\partial x_{j}}
      \sum_{k} f_{k}^2 = \frac{\partial}{\partial x_{i}}
      \sum_{k} 2 \cdot f_{k} \cdot \frac{\partial f_{k}}{\partial x_{j}} =
      \nonumber
    \end{equation}
    \begin{equation}
      \sum_{k} 2 \cdot \frac{\partial f_{k}}{\partial x_{i}} \cdot
      \frac{\partial f_{k}}{\partial x_{j}} +
      \sum_{k} 2 \cdot f_{k} \cdot
      \frac{\partial^2 f_{k}}{\partial x_{i} \partial x_{j}}
      \nonumber
    \end{equation}
    $(**)$ - zwkle uznajemy tem człon za mały w porównaniu
    z poprzedniem\\
    $\Rightarrow$ \emph{linearyzacja} (modelu! $T_{k}(x)$ -- liniowe)
  \end{frame}

  \begin{frame}{Liniowy model}
    %needs further formatting
    \emph{gdy drugi człon w} $(*)$ = 0 $\Rightarrow$
    linear least squares $\to F(x)$ -- quadratic; \\
    i cały problem minimalizacji $\Rightarrow$ znalezienie
    macierzy $\left[ \frac{\partial^2 F}{\partial x_{i} \partial x_{j}} \right]^{-1}$
    a znalezienie min!$F(x)$ w pojedyńczym kroku newtonowskim;\\
    \emph{gdy drugi człon w} $(*) \approx 0 \Rightarrow$
    non-linear least squares\\
    \begin{itemize}
      \item $\frac{\partial^2 F}{\partial x_{i} \partial x_{j}}$
      łatwe do policzenia (positiv defined)
      \item kilka kroków $\to$ iteracyjnie
      \item macierz $\left[ \frac{\partial^2 F}{\partial x_{i} \partial x_{j}} \right]^{-1}$
      nie zbiega się do macięrzy kowariancji
    \end{itemize}
  \end{frame}

\subsection{Likelihood maximization}
  \begin{frame}{Likelihood maximization}
    \begin{equation}
      min!F(x) = \underbrace{- \sum_{k=1}^{k} ln\ f_{k}(x)}_{(*)}{,} \quad
      f_{k}(x) = \frac{1}{\sigma^{2}_{k}} \left[ Y_{k} - T_{k}(x) \right]^2
      \nonumber
    \end{equation}
    $(*)$ - logarytmiczna funkcja wiarygodności
    \begin{equation}
      \frac{\partial^2 F}{\partial x_{i} \partial x_{j}} =
      - \frac{\partial}{\partial x_{i}}\frac{\partial}{\partial x_{j}} =
      - \frac{\partial}{\partial x_{i}}
      \sum_{k} \frac{1}{f_{k}} \frac{\partial f_{k}}{\partial x_{j}} =
      \sum_{k} \frac{1}{f^{2}_{k}} \frac{\partial f_{k}}{\partial x_{i}} \frac{\partial f_{k}}{\partial x_{j}} -
      \underbrace{\sum_{k} \frac{1}{f_{k}} \frac{\partial^{2} f_{k}}{\partial x_{i} \partial x_{j}}}_{(*)}
      \nonumber
    \end{equation}
    $(*)$ -- jeżeli można pominąć \dots
  \end{frame}

\subsection{Model with separable computing}

  \begin{frame}{Model with separable computing}
    \begin{itemize}
      \item znaczna część obliczeń zależy tylko od kilku
      składowych $x \Rightarrow$ nia ulega zmianie, gdy
      one również bez zmiany
      \item podział na $n$-części $\to$ a każda zależy tylko
      od jednego parametru
    \end{itemize}
    % block can be added
    \textbf{Z tego wynika} -- efektywne niektóre proste
    metody (np. single parameter variation).
  \end{frame}

\subsection{Podobne zagadnienia}

  \begin{frame}{Podobne zagadnienia}
    \begin{exampleblock}{Naprzykład}
      \begin{itemize}
        \item znajdowanie krzywej $p(y)=min_{x}!F(x,y)$,
        np. przy określaniu przedziału ufności
        \item podobny eksperyment, seria eksperymentów
      \end{itemize}
    \end{exampleblock}
    \begin{block}{Z tego wynika}
      \begin{itemize}
        \item użycie poprzedniego wyniku jako punktu startowego
        \item macierz kowariancji nie powinna się istotnie zmienić
      \end{itemize}
    \end{block}
  \end{frame}
