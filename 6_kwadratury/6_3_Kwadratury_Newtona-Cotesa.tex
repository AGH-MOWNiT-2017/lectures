\section{Kwadratury Newtona-Cotesa}
%%%%%%%%%%%%%%%%%%%%%%%%%
	\begin{frame}{Kwadratury Newtona-Cotesa}
		Zamiana problematycznej funkcji podcałkowej na prosty wielomian interpolujący.
		$\newline$$\newline$
		\textcolor{blue}{Charakterystyka kwadratur N-C:}
    	\begin{itemize}
    	\item wzór interpolacyjny Lagrange'a:
        \[
        f(x)=\sum_{i=0}^{n}L_{i}(x)f(x_{i})+\frac{f^{(n+1)}(\eta(x))}{(n+1)!}\prod_{j=0}^{n}(x-x_{j})
        \]
        \item węzły równoodległe
    	\end{itemize}
    	$\newline$
    	\textcolor{blue}{Uwagi:}
    	\begin{itemize}
    	\item metody prostokątów i trapezów są szczególnymi przypadkami kwadratur N-C
    	\item lepsze efekty uzyskuje się zastępując węzły równoodległe węzłami Czebyszewa
    	\end{itemize}
	\end{frame}
%%%%%%%%%%%%%%%%%%%%%%%%%
	\begin{frame}{Wzory zamknięte N-C}
    	$$
        h=\frac{b-a}{n},\ a=x_{0}<x_{1}<\ldots<x_{n}=b,\newline x_{i}=x_{0}+i\cdot h,\ i=0, 1, \ldots , n
        $$
          $$
\int_{a}^{b}f(x)dx=\sum_{i=0}^{n}a_{i}f(x_{i})+E
          $$
		
        \textbf{Liczby Cotesa:}
          $$
a_{i}=\int_{x_{0}}^{x_{n}}L_{i}(x)dx=\int_{x_{0}}^{x_{n}}\prod_{j=0,j\neq i}^{n}\frac{x-x_{j}}{x_{i}-x_{j}}dx
          $$
          \begin{itemize}
          \item wszystkie $a_{i}>0$ tylko dla $:n\leq 7, n=9$
          \item $\sum a_{i}$ =długości przedziału całkowania(dlaczego?)
          \end{itemize}
         \begin{flushright}
         	\textbf{Zadanie: pokazać}
         \end{flushright}
    
	\end{frame}
%%%%%%%%%%%%%%%%%%%%%%%%%
	\begin{frame}
    	\textbf{Błąd} \textit{E}, $\eta\in[a,\ b]$ \newline
        \begin{itemize}
        \item \textit{n-parzyste}, $f\in C^{n+2}[a,\ b]$
        $$
 		E= \frac{h^{n+3}\cdot f^{(n+2)}(\eta)}{(n+2)!}\int_{0}^{n}t^{2}(t-1)\ldots(t-n)dt
 		$$
        stopień dokładności: (n+1) 
        \item \textit{n-nieparzyste}, $f\in C^{n+1}[a,\ b]$
        $$
		E=\frac{h^{n+2}\cdot f^{(n+1)}(\eta)}{(n+1)!}\int_{0}^{n}t(t-1)\ldots(t-n)dt
 		$$
        stopień dokładności: (n)
        \end{itemize}
        \begin{flushright}
         	Zadanie, A. Ralston, Rozdz. 4 (1983)
        \end{flushright}
	\end{frame}
%%%%%%%%%%%%%%%%%%%%%%%%%
	\begin{frame}{Najczęściej używane formuły zamknięte N-C}
		a) $n=1$, Trapezoid Rule
        $$
          \int_{x_{0}}^{x_{1}}f(x)dx=\frac{h}{2}[f(x_{0})+f(x_{1})]-\frac{h^{3}}{12}f''(\eta)\ ,\ \eta\in[a,\ b]
        $$
        b) $n=2$, Simpson's Rule
      
        $$
          \int_{x_{0}}^{x_{2}}f(x)dx=\frac{h}{3}[f(x_{0})+4f(x_{1})+f(x_{2})]-\frac{h^{5}}{90}f^{(4)}(\eta)
        $$
        
	\end{frame}          
%%%%%%%%%%%%%%%%%%%%%%%%%
	\begin{frame}
        c) $n=3$, Simpson's Three-Eights Rule
        $$
          \int_{x_{0}}^{x_{3}}f(x)dx=\frac{3h}{8}[f(x_{0})+3f(x_{1})+3f(x_{2})+f(x_{3})]-\frac{3h^{5}}{80}f^{(4)}(\eta)
        $$
        d) $n=4$\newline\newline
        \scalebox{0.85}{
        $
        \displaystyle\int_{x_{0}}^{x_{4}}f(x)dx=\frac{2h}{45}[7f(x_{0})+32f(x_{1})+12f(x_{2})+32f(x_{3})+7f(x_{4})]-\frac{8h^{7}}{945}f^{(6)}(\eta))
        $
        }
	\end{frame}
%%%%%%%%%%%%%%%%%%%%%%%%%
	\begin{frame}{Wzory otwarte N-C}
		
$h=\frac{b-a}{n+2} \ \ , a=x_{-1}<x_{0}<x_{1}<\ldots<x_{n}<x_{n+1}=b,$\\ $\newline$
$x_{i}=x_{0}+i\cdot h,\ i=0, 1,\ldots,n$

   	$$
    \int_{a}^{b}f(x)dx=\int_{x_{-1}}^{x_{n+1}}f(x)dx=\sum_{i=0}^{n}a_{i}f(x_{i})+E
    $$
    
\textbf{Liczby Cotesa:} $$a_{i}=\int_{a}^{b}L_{i}(x)dx$$

wszystkie $a_{i}>0$ tylko dla $n=1$, 2, 3, 5.

	\end{frame}
%%%%%%%%%%%%%%%%%%%%%%%%%
	\begin{frame}
	
	\textbf{Błąd} $E, \eta\in[a,\ b]$
	\begin{itemize}
	\item {\it n-parzyste}, $f\in C^{n+2}[a,\ b]$
     $$
		E= \frac{h^{n+3}\cdot f^{(n+2)}(\eta)}{(n+2)!}\int_{-1}^{n+1}t^{2}(t-1)\ldots(t-n)dt,
    $$
     stopień dokładności: (n+1) 
     $\newline$
    \item {\it n-nieparzyste}, $f\in C^{n+1}[a,\ b]$
    $$
		E= \frac{h^{n+2}\cdot f^{(n+1)}(\eta)}{(n+1)!}\int_{-1}^{n+1}t(t-1)\ldots(t-n)dt,
    $$
     stopień dokładności: (n) 
	\end{itemize}
	\end{frame}
%%%%%%%%%%%%%%%%%%%%%%%%%
	\begin{frame}{Najczęściej używane kwadratury otwarte Newtona-Cotesa}
a) $n=0$ \textbf{, Midpoint Rule, Rectangle Rule}

		$$
\int_{x-1}^{x_{1}}f(x)dx=2hf(x_{0})+\frac{h^{3}}{3}f''(\eta)\ ,\ \eta\in[x_{-1},\ x_{n+1}]
		$$

b) $n=1$
		$$
\int_{x-1}^{x_{2}}f(x)dx=\frac{3h}{2}[f(x_{0})+f(x_{1})]+\frac{3h^{3}}{4}f''(\eta)
		$$

c) $n=2$ \textbf{, Milne's Rule}
		$$
\int_{x-1}^{x_{3}}f(x)dx=\frac{4h}{3}[2f(x_{0})-f(x_{1})+2f(x_{2})]+\frac{14h^{5}}{45}f^{(4)}(\eta)
		$$

d) $n=3$
		$$
\int_{x-1}^{x_{4}}f(x)dx=\frac{5h}{24} [11f(x_{0})+f(x_{1})+f(x_{2})+11f(x_{3})]+ \frac{95h^{5}}{144}f^{(4)}(\eta)
		$$
	\end{frame}
%%%%%%%%%%%%%%%%%%%%%%%%%
	\begin{frame}{Kwadratury N-C -- podsumowanie}
    \begin{itemize}
    	\item zwiększenie stopnia dokładności (otwartych i zamkniętych) przez dodanie co najmniej 2 nowych węzłów
    		
    	\item w przypadku dodania jednego punktu dla:
    	\begin{itemize}
    		\item[*] parzystej liczby węzłów $\rightarrow$ brak zmian st. dokładności,
    		\item[*] nieparzytej liczby węzłów $\rightarrow$ wzrost st. dokładności o 2
    	\end{itemize}
    	\item otwarte -- na ogół gorsze od zamkniętych, używane:
    		\begin{itemize}
    			\item[*] osobliwości w granicach przedziału
    			\item[*] w numerycznym rozwiązywaniu równań różniczkowych zwyczajnych
    		\end{itemize}
   	 	\item możliwe formuły półotwarte (półzamknięte)
    	\end{itemize}
	\end{frame}
%%%%%%%%%%%%%%%%%%%%%%%%%
	\begin{frame}
	\begin{itemize}
		\item efekty wzrostu $n$:
        \begin{itemize}
        	\item[*]  zmniejszanie stałego czynnika w $E$ , lecz wzrost rzędu pochodnej
            \item[*] trudności z oszacowaniem $E$
            \item[*] duże wartości $f^{(n+1)}(\eta)$
            \item[*] $a_{i}>0:$
            zamknięte - tylko $n=1$, 2,\ldots 7 i 9
            otwarte - tylko $n=$1, 2, 3 i 5. - złe uwarunkowanie
            \item[*] oscylacyjny charakter wielomianu interpolacyjnego (zwłaszcza, że
 			węzły są równoodległe)
            \item[*] trudne do uzyskania liczby Cotesa
        \end{itemize}
       \item $\sum_{i=1}^{n}a_{i}=(b-a)$ ($\rightarrow$ dlaczego?)
	\end{itemize}
	\end{frame}






