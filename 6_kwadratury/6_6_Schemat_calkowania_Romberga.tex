\section{Schemat całkowania Romberga}
	\begin{frame}{Schemat całkowania Romberga}
    	
      Całkowity błąd złożonej kwadratury trapezów:
      $\newline$
      $$E=\displaystyle \sum_{i=1}^{\infty}a_{i}h^{2i}$$
	  $$a_{i}=a_{i}(a,\ b,\ f(x)) , \ \ \textbf{nie zalezy od h!}$$
    \end{frame}
%%%%%%%%%%%%%%%%%%%%%%%%%
	\begin{frame}
    		
  	  	$\hfill${\it Zadanie}: p. Ralston
	
      	aby tak było -- $f\in C^{\infty}[a,\ b] \quad=\displaystyle \int_{a}^{b}f(x)dx$;

      	$T_{a,j}$ - -wynik kwadratury trapezów dla $h_{j}=\displaystyle \frac{b-a}{2j}$
		$\newline$
        
	  	\textbf{- 1-szy krok:}

		$\left\{\begin{array}{l}
			I-T_{0,j}=a_{2}h_{j}^{2}+a_{4}h_{j}^{4}+\ldots\\
			I-T_{0,j+1}=a_{2}h_{j+1}^{2}+a_{4}h_{j+1}^{4}+\ldots\quad\ |\rightarrow h_{j+1}=\frac{h_{j}}{2};*4,\ -,\ /3
		\end{array}\right.$ 
		$\newline$
        ekstrapolacja Richardsona daje:
        $$
        	I-\displaystyle \underbrace{\frac{1}{3}[4T_{0,j+1}-T_{0,j}]}_{T_{1,j}}=b_{4}h_{j}^{4}+b_{6}h_{j}^{6}+\ldots
        $$
        $T_{1,j} \equiv$ wzór Simpsona dla $2^{j}$ podprzedziałów.

        $\hfill${\it Zadanie}: Pokazać

    \end{frame}
%%%%%%%%%%%%%%%%%%%%%%%%%
	\begin{frame}
    	\textbf{- 2-gi krok:}
        
        $\left\{\begin{array}{l}
        	I-T_{1,j}=b_{4}h_{j}^{2}+b_{6}h_{j}^{4}+\ldots \\
        	I-T_{1,j+1}=b_{4}h_{j+1}^{2}+b_{6}h_{j+1}^{4}+\ldots\ *4_{j}j-j/3
        \end{array}\right.$
        
        $$
        I-\displaystyle \underbrace{\frac{1}{15}[16T_{1,j+1}-T_{1,j}]}_{T_{2,j}}=c_{6}h_{j}^{6}+c_{8}h_{j}^{8}+\ldots
        $$
        
        I ogólnie:
        mając
        \begin{center}
        $T_{0,j},\quad j=0$, 1, . . . , $m$
        \end{center}
        tworzymy:
        
        $$
        	T_{i,j}=\displaystyle \frac{1}{4^{i}-1}(4^{i}T_{i-1,j-1}-T_{i-1,j}) ,\quad i, j=0, 1, 2, . . .   
        $$
        $\hfill${\it Zadanie}: Udowodnić
    
    \end{frame}
%%%%%%%%%%%%%%%%%%%%%%%%%
	\begin{frame}{Romberg $T$-table:}
    	\scalebox{0.92}{
		$\begin{array}{ccccccccccc}
          	T_{0,0} \\
        	\downarrow & \searrow \\
        	T_{0,1} & \rightarrow & T_{1,0} \\
            \downarrow & \searrow & \downarrow & \searrow \\
            T_{0,2} & \rightarrow & T_{1,1} & \rightarrow & T_{2,0} \\
            \downarrow & \searrow & \downarrow & \searrow & \downarrow & \searrow \\
            T_{0,3} & \rightarrow & T_{1,2} & \rightarrow & T_{2,1} & \rightarrow & T_{3,0} \\
            & \ldots \\
            \downarrow & \searrow & \downarrow & \searrow & \downarrow & \searrow & & \searrow \\
            T_{0,m-1} & \rightarrow & T_{1,m-2} & \rightarrow & T_{2,m-3} & \rightarrow & T_{3,m-4} & \ldots & T_{m-1,0} \\
            \downarrow & \searrow & \downarrow & \searrow & \downarrow & \searrow & & \searrow & \downarrow & \searrow \\
            T_{0,m} & \rightarrow & T_{1,m-1} & \rightarrow & T_{2,m-2} & \rightarrow & T_{3,m-3} & \ldots & T_{m-1,1} & \rightarrow &  T_{m,0}\\
            
        \end{array}$
        }
        $\newline$
        Sposób wypełniania wierszy tablicy $T$: (wystarczy 1 wiersz)
    \end{frame}
%%%%%%%%%%%%%%%%%%%%%%%%%
	\begin{frame}
    	\textbf{- 1-sza iteracja:}
        $\newline$
    	$\begin{array}{ccc}
        	T_{0,0} \\
            T_{0,0} & \rightarrow & T_{0,1} \\
            \downarrow & \swarrow \\
            T_{1,0} & & T_{0,1}
        \end{array}$
        
        $\newline$
        
    	\textbf{- 2-ga iteracja:}
        
        $\begin{array}{cccccc}
            T_{0,0} & \rightarrow & T_{0,1} & \rightarrow & T_{0,2} \\
            \downarrow & \swarrow &\downarrow \\
            T_{1,0} & & T_{1,1} & & T_{0,2} \\
            \downarrow & \swarrow \\
            T_{2,0} & & T_{1,1} & & T_{0,2} \\
        \end{array}$
    \end{frame}
%%%%%%%%%%%%%%%%%%%%%%%%%
	\begin{frame}
    	Zbieżność całkowania Romberga $\rightarrow$ patrz Ralston $\newline$

        Ogólnie: im wyższe $\mathrm{i}, h^{i}\rightarrow$ tym większe $f^{i}(\eta)$ $\newline$
		ale -- w całkowaniu trapezów $\rightarrow$ zbieżność! $\newline$
        
		gdy $f(x)\in C^{\infty}[a,\ b]$, {\it to}:

		\quad - w kolumnach $\displaystyle \lim_{l\rightarrow\infty}T_{p,l}\rightarrow I$;\quad $w (p+1)$ szybciej niż w $p$

		\quad - ciąg diagonalny $T_{i,i}$ szybciej zbieżny do I niż ciągi w kolumnach
        $\newline$

		stąd warunek zakończenia:
        \begin{center}
        	$|T_{m,0}-T_{m-1,0}|<\delta \cdot |T_{m,0}|, \delta-$ {\it zadane}.
        \end{center}
        
        Reszta schematu Romberga:
        \begin{center}
        	$E=c(i,\ j)\cdot h^{2j+2}\cdot f^{2j+2}(\eta)$
        \end{center}
		\hfill$c(i,\ j)$ - stała malejąca z $i$, $j$\quad $h=\displaystyle \frac{b-a}{2^{i+j}}.$


    \end{frame}
%%%%%%%%%%%%%%%%%%%%%%%%%
    \begin{frame}
    	Schemat Romberga wymaga by błąd kwadratury w  $i$-tym wierszu:
        \begin{center}
        	$E=\displaystyle \sum_{l=1}^{i}a_{l}h_{i}^{2l}+O(h_{i}^{2i+2})\ ,\quad\ f\in C^{2i+2}[a,\ b]$   
        \end{center}
W procedurze powinno to być sprawdzane!

Procedury całkowania Romberga mogą być adaptacyjne.
    \end{frame}
%%%%%%%%%%%%%%%%%%%%%%%%%
		