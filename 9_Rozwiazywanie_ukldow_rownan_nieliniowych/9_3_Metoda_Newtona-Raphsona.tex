\section{9.3 Metoda Newtona-Raphsona dla układów równań nieliniowych}

\begin{frame}{}
  \begin{itemize}
    \item $\{x_j^{(n-1)}\}$ - przybliżenie pierwiastków $\{\alpha_j\}$
    \item $\alpha_j=x_j^{(n-1)}+h_j$\\
    $f_i(\overrightarrow{x}^{(n-1)}+\overrightarrow{h})=0\quad\rightarrow\quad$szereg Taylora:
    $$f_i(\overrightarrow{x}^{(n-1)})+\sum_{j/1}^{m}(\frac{\partial f_i}{\partial x_j})_{n-1}*h_j+\underbrace{O(\delta x^2)}_{pomijamy}=0\quad(\star\star)$$
    gdzie $(\frac{\partial f_i}{\partial x_j})_{n-1}$- oznacza obliczone w $\overrightarrow{\alpha}^{(n-1)}$
  \end{itemize}
\end{frame}

\begin{frame}{}
  Jako kolejne przybliżenie $\alpha$ bierzemy $\overrightarrow{x}^{(n)}=\overrightarrow{x}^{(n-1)}+\overrightarrow{h}$\\
  przy czym $\overrightarrow{h}$ - wyznaczamy z $(\star\star)$\\
  Można to zapisać:
 $$
 \left.
 \begin{array}{lr}
 J^{(n1)}*\overrightarrow{h}^{(n-1)} = -\overrightarrow{f}^{(n-1)}\\
 \overrightarrow{x}^{(n)} = \overrightarrow{x}^{(n1)}+\overrightarrow{h}^{(n-1)}
 \end{array}\right\}
 \text{- układ równań liniowych}
 $$
  jakobian:
  $$J^{(n-1)}_{i,j}=(\frac{\partial f_i}{\partial x_i})_{(n-1)}$$
\end{frame}

\begin{frame}{}
  \begin{block}{trudności}
    \begin{itemize}
      \item wybór $\overrightarrow{x}^{(0)}$,
      \item sprawa zbieżności - trudna w przypadku ogólnym,
      \item złożoność obliczeniowa
      \begin{itemize}
        \item[-] $J\leftarrow m^2$ elementów:
        \begin{itemize}
          \item[$\star$] formalnie różniczkowanie,
          \item[$\star$] wprowadzanie do programu
        \end{itemize}
        \item[-] w każdej iteracji - J - na nowo,
        \item[-] h - z rozwiązania układu równań liniowych.
      \end{itemize}
    \end{itemize}
  \end{block}
\end{frame}

