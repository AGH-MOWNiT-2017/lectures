\section{Klasyfikacja PDE}

\begin{frame}{Klasyfikacja PDE}
  \begin{block}{Oznaczenie}
    PDE -- partial differential equation (równanie różniczkowe cząstkowe)
  \end{block}
\end{frame}

\begin{frame}
  \begin{block}{Definicja -- równanie różniczkowe cząstkowe drugiego rzędu}
    Niech:\\
    $R$ -- zbiór na płaszczyźnie

    Równanie:
\begin{equation}\label{1}
\begin{split}      
a(x, y, u, \frac{{\partial}u}{{\partial}x}, \frac{{\partial}u}{{\partial}y}) \cdot \frac{{\partial}^2u}{{\partial}x^2} +
      2 \cdot b(x, y, u, \frac{{\partial}u}{{\partial}x}, \frac{{\partial}u}{{\partial}y}) \cdot \frac{{\partial}^2u}{{\partial}x{\partial}y} + \\
      + c(x, y, u, \frac{{\partial}u}{{\partial}x}, \frac{{\partial}u}{{\partial}y}) \cdot \frac{{\partial}^2u}{{\partial}y^2} +
      f(x, y, u, \frac{{\partial}u}{{\partial}x}, \frac{{\partial}u}{{\partial}y}) = 0 
\end{split}
\end{equation}


    z warunkiem $a^2 + b^2 + c^2 \not = 0 \quad \forall_{(x,y) \in R}$

    nosi nazwę \textbf{równania różniczkowego cząstkowego drugiego rzędu}.
  \end{block}
\end{frame}

\begin{frame}
  W szczególnym przypadku:

  \begin{block}{Definicja -- równanie liniowe i słabo nieliniowe}
    Niech:
    $$a = a(x,y), \quad b = b(x,y), \quad c = c(x,y)$$

    równanie to jest \textit{liniowym}, gdy:
    $$f \equiv d(x,y) \frac{{\partial}u}{{\partial}x} + e(x,y) \frac{{\partial}u}{{\partial}y} + g(x,y)u + h(x,y)$$

    zaś \textit{słabo nieliniowym}, gdy:
    $$f \equiv f(x,y,u)$$
  \end{block}
\end{frame}

\begin{frame}
  W dowolnym $(x,y) \in R$ równanie $(\ref{1})$ jest:
  \begin{itemize}
    \item eliptyczne, gdy $b^2 - ac < 0$
    \item paraboliczne, gdy $b^2 - ac = 0$
    \item hiperboliczne, gdy $b^2 - ac > 0$
  \end{itemize}
\end{frame}

\begin{frame}
  \begin{exampleblock}{Przykłady}
      \begin{itemize}
        \item r. potencjału; Laplace'a $\rightarrow$ eliptyczne (wszędzie)
        $$\frac{{\partial}^2u}{{\partial}x^2} + \frac{{\partial}^2u}{{\partial}y^2} = 0$$

        \item r. transportu ciepła $\rightarrow$ paraboliczne
        $$\frac{{\partial}^2u}{{\partial}x^2} - \frac{{\partial}u}{{\partial}y} = 0$$

        \item r. falowe $\rightarrow$ hiperboliczne
        $$\frac{{\partial}^2u}{{\partial}x^2} - \frac{{\partial}^2u}{{\partial}y^2} = 0$$
      \end{itemize}
  \end{exampleblock}
\end{frame}
