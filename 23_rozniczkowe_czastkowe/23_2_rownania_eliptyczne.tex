\section{Równania eliptyczne}

\begin{frame}{Równania eliptyczne}
  \textbf{Równania eliptyczne} opisują zagadnienia równowagi zastosowania PDE

  \begin{block}{Równanie prototypowe -- równanie Laplace'a}
    $$\left\{ \begin{array}{l}
    \frac{{\partial}^2u}{{\partial}x^2} + \frac{{\partial}^2u}{{\partial}y^2} = 0 \\
    u_{xx} + u_{yy} = 0 \\
    \left. \begin{array}{l}
    \Delta u = 0 \\
    {\nabla}^2 u = 0
    \end{array} \right\} \text{nie jest sprecyzowany układ współrzędnych}
    \end{array} \right.$$
  \end{block}
  $\rightarrow$ teoria potencjału, grawitacji.
\end{frame}

\begin{frame}
  rozwiązania Laplace'a $\rightarrow$ funkcje harmoniczne

  Ważna własność funkcji harmonicznych:
  \begin{block}{Własność min-max}
    Jeżeli
    \begin{itemize}
      \item $R$ -- obszar jednospójny (simply connected)
      \item $S$ -- brzeg obszaru
      \item $U$ -- f. harmoniczna na $R$ i ciągła na $R \bigcup S$ % Czy to na pewno ma być wielkie U???
    \end{itemize}

    to \textit{$u$ przyjmuje największą i najmniejszą wartość na brzegu $S$}.
  \end{block}
\end{frame}
