\section{Algorytmy numerycznie poprawne}
%%%%%%%%%%%%%%%%
\begin{frame}{Algorytmy numerycznie poprawne}
	Ograniczenia:
    \begin{itemize}
    	\item dane - {\it zaburzone}
    	\item wyniki - {\it zaburzone}
    \end{itemize}
    stąd: błędy reprezentacji
\end{frame}
%%%%%%%%%%%%%%%%
\begin{frame}{Algorytmy numerycznie poprawne}
	\begin{block}{Definicja}
		{\bf Algorytmy numerycznie poprawne} - takie, które dają rozwiązania będące nieco zaburzonym dokładnym rozwiązaniem zadania o nieco {\it zaburzonych} danych. Są to algorytmy najwyższej jakości.
        
        {\bf Dane nieco zaburzone} - zaburzone na poziomie reprezentacji.
	\end{block}
\end{frame}
%%%%%%%%%%%%%%%%
\begin{frame}{Algorytmy numerycznie poprawne - definicje}
	\begin{block}{Definicja}
    	Algorytm A jest {\bf numerycznie poprawny} w klasie zadań $K_d$, $K_w$, jeżeli istnieją stałe $K_d, K_w$ takie, że:
        \begin{itemize}
        	\item $\forall d \in D$,
            \item dla każdej dostatecznie silnej arytmetyki $\left( \beta^{1-t} \right)$
        \end{itemize}
        $\exists \tilde{d} \in D_0$ taki, że:
        
        {\centering
        	$|| \vec{d} - \tilde{d} || \le \varrho_d \cdot K_d \cdot || \vec{d} ||$ \\ \vspace{.1cm}
            $|| \varphi(\vec{d}) - fl(A(\vec{d})) || \le \varrho_w \cdot K_w \cdot || \varphi(\vec{d}) ||$ \\}

        $\varphi(\vec{d})$ - dokładnie rozwiązanie zadania o zaburzonych danych $\tilde{d}$ \\
        $K_w, K_d$ - wskaźniki kumulacji algorytmu A w klasie zadań $\left\{\varphi, D\right\}$
        $K_w, K_d$:
        \begin{itemize}
        	\item dla dowolnych danych klasy $\left\{ \varphi, D \right\}$,
            \item minimalne $\rightarrow$ jakość A.
        \end{itemize}
	\end{block}
\end{frame}
%%%%%%%%%%%%%%%%
\begin{frame}{Algorytmy numerycznie poprawne - definicje}
	\begin{block}{Definicja}
		{\bf Użyteczne algorytmy} - gdy wskaźniki kumulacji rzędu liczby działań
	\end{block}
\end{frame}
%%%%%%%%%%%%%%%%
\begin{frame}[fragile]{Algorytmy numerycznie poprawne - przykład}
	\begin{exampleblock}{Przykład 1}
		Numeryczna poprawność algorytmu $\vec{a} \cdot \vec{b} = \sum_{i=1}^{n} a_i \cdot b_i$
        
\begin{lstlisting}[escapechar=|]
  |$A(\vec{a}, \vec{b}):$|
  s := 0;
  for i:=1 to n do s := s + |$a_1 \cdot b_i$|;
\end{lstlisting}
	\end{exampleblock}
\end{frame}
%%%%%%%%%%%%%%%%
\begin{frame}{Algorytmy numerycznie poprawne - przykład}
	\begin{exampleblock}{Przykład 1}
		{\bf Realizacja algorytmu}: $fl(A(\vec{a}, \vec{b}))$:
        \begin{enumerate}
        	\item dane - reprezentacje \\
                \hspace{1cm} $a_i \to \hat{a_i} = rd(a_i) = a_i \cdot (1 + \alpha_i)$ \\
                \hspace{1cm} $b_i \to \hat{b_i} = rd(b_i) = b_i \cdot (1 + \beta_i)$
        	\item działania - przybliżone, $fl$ \\
            	np. $i = 1, 2, 3:$ \\ 
                \begin{addmargin}[1em]{0em}
                $
                    fl(A(\vec{a}, \vec{b})) = \{
                        [
                            \hat{a_1} \cdot \hat{b_1} \cdot (1 + \varepsilon_1) +
                            \hat{a_2} \cdot \hat{b_2} \cdot (1 + \varepsilon_2) +
                        ]
                        \cdot (1 + \delta_2) + \hat{a_3} \cdot \hat{b_3} \cdot (1 + \varepsilon_3)
                    \} \cdot (1 + \delta_3); 
                $ \\
                $\delta_1 = 0$ \\
                
                \end{addmargin}
            	i ogólnie:
                \begin{addmargin}[1em]{0em}
                $
                	fl(A(\vec{a}, \vec{b})) =
                	\sum_{i=1}^{n} a_i \cdot (1+a_i) \cdot 
                    b_i \cdot (1 + \beta_i) \cdot (1 + \varepsilon_i)
                    \cdot \prod_{j=i}^{n} (1 + \delta_j)
                $                
                \end{addmargin}
        \end{enumerate}
	\end{exampleblock}
\end{frame}
%%%%%%%%%%%%%%%%
\begin{frame}[fragile]{Algorytmy numerycznie poprawne - przykład}
	\begin{exampleblock}{Przykład 1}
    	{\bf Interpretacja (dowolność)}
        \begin{itemize}
        	\item dokładny wynik: $K_w = 0$
            \item dla zaburzonych danych:
            	\begin{addmargin}[1cm]{0cm}
            		$|| \vec{a} - \tilde{a} || \le \beta^{1-t} \cdot || \vec{a} ||$ \hspace{.5cm} $(K_{d_1} = 1)$\\
            		$|| \vec{b} - \tilde{b} || \le (n + 1) \cdot \beta^{1-t} \cdot || \vec{b} ||$ \hspace{.5cm} $(K_{d_2} = n+1)$
            	\end{addmargin}
        \end{itemize}
        {\bf Uwaga:} pominięcie błędów reprezentacji danych $\equiv$ zmniejszenie $K_d$ o 1.
	\end{exampleblock}
\end{frame}
%%%%%%%%%%%%%%%%