\section{Numeryczna reprezentacja liczb całkowitych}
%%%%%%%%%%%%%%%%
\begin{frame}{Reprezentacja stałopozycyjna (integer)}
	\textcolor{blue}{Informacje ogólne:}
	\begin{itemize}
		\item np. kod U2: na d+1 bitach reprezentowane dokładnie liczby
			\begin{center}
				\[
   					 L \in [-2^d, 2^d-1]
    			\]
			\end{center}
		\item gdy argumenty i wynik reprezentowane stałopozycyjnie, to działania na nich: $+$, $-$, $\cdot$, $/ (div)$, $/ (mod)$ są wykonywane dokładnie
		\item kompromis pomiędzy zakresem, a precyzją
		
	\end{itemize}
     
\end{frame}
%%%%%%%%%%%%%%%%
\begin{frame}    
    
    \textbf{Zastosowania:}
    \begin{itemize}
    	\item sprawy walutowe, operacje monetarne
    	\item procesory graficzne np. Sony Nintendo
    	\item rozmiary czcionek w calach np. w TeX
    	\item libfixmath - implementacja biblioteki stałoprzecinkowej w C
    \end{itemize}
    $\newline$
    \textbf{Zalety i wady:}
    \begin{itemize}
    	\item (+) szybkość i prostota
    	\item (-) mała skalowalność
    \end{itemize}
\end{frame}
%%%%%%%%%%%%%%%%