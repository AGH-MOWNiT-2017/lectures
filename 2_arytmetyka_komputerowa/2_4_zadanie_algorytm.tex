\section{Zadanie, algorytm}
\begin{frame}{Zadanie, algorytm}
    \begin{block}{Definicja}
    	\textbf{Zadanie:}
        \begin{columns}
            \column{.5\linewidth}
                \hfill dla danych
                \newline

                \hfill znaleźć wynik
            \column{.5\linewidth}
                $\vec{d} = \left( d_1, d_2, ..., d_n \right) \in R_d$\newline
                $\vec{w} = \left( w_1, w_2, ..., w_m \right) \in R_w$\newline
                $\vec{w} = \varphi(\vec{d})$
        \end{columns}$\newline$
        $R_d$, $R_w$ - skończenie wymiarowe, unormowane przestrzenie kartezjańskie\newline
        $\varphi: D_0 \subset R_d \rightarrow R_w$ - odwzorowanie ciagłe
    \end{block}
    \begin{block}{Definicja}
        {\bf Algorytm A w klasie zadań $\{\varphi, D\}$} jest to sposób wyznaczenia wyniku $\vec{w} = \varphi(\vec{d})$ dla $d \in D \subset D_0$, z dokładną realizacją działań, tj. w zwykłej arytmetyce
    \end{block}
\end{frame}
%%%%%%%%%%%%%%%%
\begin{frame}{Realizacja algorytmu w arytmetyce float - $fl(A(\vec{d}))$}
    Zastąpienie:
    \begin{itemize}
        \item $d, x, ..$
        \item arytmetyki
    \end{itemize}
    Przez:
    \begin{itemize}
        \item $rd(d), rd(x), ..$
        \item arytmetykę float
    \end{itemize}
\end{frame}
%%%%%%%%%%%%%%%%
\begin{frame}{Założenia o reprezentacji danych i wyników}
    \begin{block}{Założenia}
    \[
    || \vec{d} - rd(\vec{d}) ||
    \le 
    \varrho_d ||\vec{d} ||
    \] \[
    || \vec{w} - rd(\vec{w}) ||
    \le 
    \varrho_w ||\vec{w} ||
    \] \[
    \varrho_d, \varrho_w = \underbrace{k}_{\text{małe, } \approx 10} \cdot \  \beta^{1-t}
    \]
    \end{block}
    $\newline$
    W realizacji w arytmetyce float oczekujemy, że $\vec{d}$ oraz $\vec{w}$ będą reprezentowane z małymi błędami.
\end{frame}
