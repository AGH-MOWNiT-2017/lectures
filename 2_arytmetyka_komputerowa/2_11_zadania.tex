\section{Zadania}
%%%%%%%%%%%%%%%%
\begin{frame}{Zadania}
	Elementy analizy:
    \begin{itemize}
    	\item Tw. Rolle'a
        \item Tw. o wartości średniej
        \item Tw. o wartości średniej dla całek
        \item Tw. Taylor'a
    \end{itemize}
\end{frame}
%%%%%%%%%%%%%%%%
\begin{frame}{Zadania}
	Zbadać uwarunkowanie zadania (wskaźniki uwarunkowania):
    \begin{enumerate}
    	\item wyznaczania zer wielomianu: \[
				P(x) = \sum_{i=0}^{20} a_i x^i = \prod_{i=0}^{20} (x-i)
            \]
        \item wyznaczenia zer rzeczywistych: \[
        		x^2 - 2px + q = 0, \hspace{.2cm} p, q \in D = \{ 
                	(p, q): p \neq 0, q \neq 0, p^2 - q > 0
                \}
        	\]
    \end{enumerate}
\end{frame}
%%%%%%%%%%%%%%%%
\begin{frame}{Zadania}
	Sprawdzić poprawność numeryczną:
    \begin{enumerate}
    	\item algorytmu Herona $x = \sqrt{a}$
        	\begin{align*}
                x_1 &= \left\{
                    \begin{array}{ll}
                        a, \hspace{.3cm} a \ge 1 \\
                        1, \hspace{.3cm} a < 1
                    \end{array}
                \right. \\ 
                x_{i+1} &= \frac{1}{2}(x_i + \frac{a}{x_i}), i=1, 2, ...
        	\end{align*}
        \item algorytmu Hornera \[
        	P(x) = (...(a_n x + a_{n-1} x)x + ... + a_1) x + a_0
        \]
    \end{enumerate}
\end{frame}