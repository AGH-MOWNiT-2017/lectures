\section{Zastosowania numerycznej algebry liniowej}

\begin{frame}{Zastosowania numerycznej algebry liniowej}


  \begin{itemize}
    \item \textbf{Fizyka:} CFD, lattice-gauge, atomic spectra, \dots
    \item \textbf{Chemia:} chemia kwantowa, inżynieria chemiczna, 
    \item \textbf{Mechanika:} obliczenia z wykorzystaniem FEM, FD, \dots
    \item \textbf{Elektrotechnika, elektronika:} systemy energetyczne, symulacja obwodów, symulacja urządzeń, \dots
    \item \textbf{Geodezja,}
    \item \textbf{Ekonomia:} modelowanie systemów ekonomicznych,
    \item \textbf{Demografia:} migracja,
    \item \textbf{Psychologia:} grupy, powiązania,
    \item \textbf{Badania operacyjne:} programowanie liniowe,
    \item \textbf{Polityka}\dots
  \end{itemize}
\end{frame}

\section{Macierze testowe}
\begin{frame}{Macierze testowe}
% Lista wypunktowana była kłamstwem – to wszystko było opisem tylko tego jednego linku
\begin{block}{}
Standardowy zbiór macierzy rzadkich -- \\ 
\textit{Harwel - Boeing Sparse Matrix Collection} \cite{matrix}
\end{block}
\end{frame}
