\subsection{Dot product}

\begin{frame}[fragile]{Dot product}


$ c = x^T\cdot y; \quad x,y \in R^n\text{ (vectors)} $ % \quad robi niezbyt duży odstęp w trybie matematycznym.

\vspace{10px}
 \begin{lstlisting}[language=Matlab] %Trochę więcej się koloruje przy ustawieniu Matlab, ale jeśli wolisz, wróć do Pascala.

function: c = dot(x,y)

   c = 0
   n = length(x)
 	for i = 1 : n
 		c = c + x(i)%*$\cdot$*)y(i)
 	end

 end dot
 \end{lstlisting}


\end{frame}

\subsection{saxpy}

\begin{frame}[fragile]{saxpy}

$ z = \alpha \cdot x + y == $ (saxpy = \textbf{s}calar \textbf{a}lpha \textbf{x} \textbf{p}lus \textbf{y})  $\alpha$ -- scalar % Tu chodziło o zaznaczenie pierwszych liter w celu wyjaśnienia nazwy
%Myślę też, że tamto "==" nie ma żadnego sensu, lepiej by było zamiast niego zrobić przejście do nowej lini.
\vspace{10px}
 \begin{lstlisting}[language=Matlab]

function: z = saxpy(%*$\alpha$*),x,y)

  n = length(x)
 	for i = 1 : n

 		z(i) = %*$\alpha \cdot \text{x(i)} + \text{y(i)} $*)

 	end

 end saxpy
 \end{lstlisting}

\end{frame}


\subsection{The colon notation}

\begin{frame}{The colon notation}

$ A\in R^{m\cdot n} $ \\
\vspace{10px}
$k$-th row of $A$: % Wyróżniłem k i A
$$A(k, :) = [a_{k1}, a_{k2}, \dots , a_{kn}]$$ % Zmieniłem kropki
$k$-th column of $A$:
\vspace{10px}

\begin{center}
$A(:, k) = \left [ \begin{array}{l}
   a_{1k} \\
   a_{2k}\\
   \vdots\\
   a_{nk}
\end{array}
\right ] $
\end{center}

\end{frame}
