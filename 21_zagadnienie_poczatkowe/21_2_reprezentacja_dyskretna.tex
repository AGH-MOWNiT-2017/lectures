\section{Reprezentacja dyskretna zagadnienia początkowego}
\begin{frame}{Reprezentacja dyskretna}
  $$\frac{d \vec{u}}{dt} = L\vec{u}$$
  $1^o$\quad L $\rightarrow$ przekształcamy na operator różnicowy (siatka przestrzenna $x_j$) \newline
  $2^o$ \quad wprowadzamy siatkę czasową: $t^n$ 
  $$t^n = \sum_{r=2}^{n}\Delta t_r$$
  $\Delta t_r $- krok czasowy
\end{frame}
%%%%%%%%%%%%%%%%%%%%%
\begin{frame}{Całkowanie równania stanu od $t^n$ do $t^{n+1}$}
  $${\vec{u}}^{n+1} = {\vec{u}}^n + \int_{t^n}^{t^{n+1}}dt' L\vec{u}(\vec{r},t')$$ 
  $\Rightarrow$ związek między stanami w siąsiednich chwilach $t^n$,$t^{n+1}$
  \begin{center}
  	\underline{nie znamy} \qquad $\vec{u}(t')$ , $t' \in [t^n,t^{n+1}]$ \par
  \end{center}
  \underline{dla małych kroków czasowych} $\Delta t = t^{n+1} - t^n$ można: 
  $${\vec{u}}^{n+1} = {\vec{u}}^n + \int_{t^n}^{t^{n+1}}dt'\Bigg\{ \sum_{r/0}^{p-1}\bigg[\frac{dr}{d t^r}(L\vec{u})\bigg]_{t^n}\frac{(t'-t^n)^r}{r!}+0((t'-t^n)^p)\Bigg\}$$
  $${\vec{u}}^{n+1} = {\vec{u}}^n + \sum_{r/1}^{p}\bigg[\frac{d^{r-1}}{d t^{r-1}}(L\vec{u})\bigg]_{t^n}\frac{(\Delta t)^r}{r!}+0(\Delta t^{p+1})$$
  p -rząd wielkości całkowania (ze względu na krok czasowy $\Delta t$)
\end{frame}
%%%%%%%%%%%%%%%%%%%%%
\begin{frame}{W praktyce - ograniczenie do członów drugiego rzędu}
  $$\vec{u} ^{n+1} = \vec{u} ^n + L\vec{u}^n\Delta t+\bigg[\frac{d}{dt}(L\vec{u})\bigg]_{t^n} \frac{(\Delta t)^2}{2}$$
  \textbf{Wyznaczanie wartosci pochodnej:}
  \begin{itemize}
    \item wartości zmiennych na poprzednich poziomach czasu $t^{n-1}$
    \item wprowadzenie poziomów pośrednich
    \item użycie nieznanych $\vec{u}^n+1$ (w czasie o krok późniejszym $t^{n+1}$)
  \end{itemize}
  $\Rightarrow$ \textbf{Szczególny przypadek:} $\rightarrow$ użycie nieznanych wartości $u^{n+1}$:
  \begin{center}
  	$\vec{u}^{n+1} = \vec{u}^n+L\vec{u}^n(1-\varepsilon)\Delta t+L\vec{u}^{n+1}\varepsilon \Delta t $\qquad(*)
  \end{center}
  \begin{description}
    \item - $0 \leqslant \varepsilon \leqslant 1$ - parametr interpolacji
    \item - $\varepsilon = 0$ - metoda jawna 
    \item - $\varepsilon \not= 0$ - metoda niejawna
    \item - dokładność 2-go rzędu zachowana jedynie dla $\varepsilon=\frac{1}{2}$
  \end{description}
  
\end{frame}
%%%%%%%%%%%%%%%%%%%%%
\begin{frame}{W praktyce c.d.}
  po uporządkowaniu w (*) można zapisać:
  $$\vec{u}^{n+1} = (1-\varepsilon\cdot\Delta t\cdot L)^{-1}\cdot[1+(1-\varepsilon)\Delta t\cdot L]\cdot \vec{u}^n $$
  czyli: \fbox{$\vec{u}^{n+1}=T(\Delta,\Delta t)\vec{u}^n$} \qquad por.: $\frac{d\vec{u}}{dt} = L\vec{u}$\newline\newline\newline
  \textbf{T - operator różnicowy}, wiąże kolejne punkty na siatce czasowej $\equiv$ zagadnienie początkowe $\Rightarrow$ ciąg rozwiązań w izolowanych punktach czasowych (T).
\end{frame}