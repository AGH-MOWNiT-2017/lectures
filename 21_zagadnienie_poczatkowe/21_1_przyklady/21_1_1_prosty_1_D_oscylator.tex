%\subsection{Prosty 1-D oscylator harmoniczny}
\begin{frame}{Prosty 1-D oscylator harmoniczny}
  $$m\dot x = -\kappa \cdot x$$
  $$ x(t_0)=x_0$$
  $$\dot x(t_0) = \dot x_0$$
  inaczej:
  $$\dot x = v, \dot v = - \frac{\kappa}{m} \cdot x$$

  $\vec{u}$ - wektor opisującu stan układu - \textbf{wektor stanu} \par
  $L$ - liniowy operator macierzowy \par\par
  $\Rightarrow$ układ równań różniczkowych zwyczajnych
\end{frame}