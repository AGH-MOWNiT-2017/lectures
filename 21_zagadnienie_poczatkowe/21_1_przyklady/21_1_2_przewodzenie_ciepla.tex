%\subsection{Przewodzenie ciepła przez pręt 1-D}
\begin{frame}{Przewodzenie ciepła przez pręt 1-D}
  \begin{center}
      \underline{$T(x,t)$ - temperatura} $\rightarrow$ \quad wektor stanu \par
  \end{center}
  $$\frac{\partial T}{\partial t} = \frac{\partial}{\partial x}\kappa \frac{\partial T}{\partial t};\quad T(x,t_0) = T_0(x)$$
  $\kappa$ - współczynnik przewodnictwa \par
  czyli:
  $$u(x,t) = T(x,t)$$
  $$L = \frac{\partial}{\partial x} \kappa \frac{\partial}{\partial x}$$
  \textbf{L} - przestrzenny operator różniczkowy \par
  Stan układu jest określony w kontinuum przestrzennym. Podobne z punktu widzenia zmiennej t, a ten aspekt rozpatrujemy.
\end{frame}
