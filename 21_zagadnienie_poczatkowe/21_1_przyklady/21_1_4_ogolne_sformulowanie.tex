%\subsection{Ogólne sformułowanie zagadnienia początkowego}
\begin{frame}{Ogólne sformułowanie zagadnienia początkowego}
  Ogólne sformułowanie zagadnienia początkowego - to samo sformułowanie różnych problemów uwzględniające zmienną ``czasopodobną'' 
  \begin{itemize}
    \item stan układu określony przez wektor stanu $\vec{u}(\vec{r},t)$ w obszarze $R = R(\vec{r})$
    \item dla $ t = 0 $ dane $\vec{u}(\vec{r},t_0) = \vec{u}_0(\vec{r})$
    \item stan układu dla $ t > 0$ można otrzymać jako rozwiązanie:
    $$\frac{d\vec{u}}{dt} = L\vec{u}$$ \par
    \begin{center}
    	$\vec{u}$ określone na powierzchni S w R dla wszystkich T
    \end{center}
  \end{itemize}
  
\end{frame}
%%%%%%%%%%%%%%%%
\begin{frame}{L - ogólnie}
\begin{block}{\textbf{L - ogólnie : operator nieliniowy}}
	\begin{itemize}
    	\item dla równań różniczkowych zwyczajnych - algebraiczny
    	\item dla równań różniczkowych cząstkowych - przestrzenny, różniczkowy
 	\end{itemize}
\end{block}
	
  
\end{frame}