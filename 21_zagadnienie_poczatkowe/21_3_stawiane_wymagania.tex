\section{Wymagania stawiane różnicowemu rozwiązaniu zagadnienia początkowego}
%%%%%%%%%%%%%%%%
\begin{frame}{Wymagania}
\textbf{Stawiane wymagania:}
	\begin{itemize}
      \item zgodność
      \item dokładność
      \item stabilność
      \item efektywność
	\end{itemize}
    
	\begin{flushleft}
	Operator \textbf{T} - sprzęga poziomy czasowe, zależy od wyboru schematu 	całkowania i przestrzennego schematu różnicowego 
	\end{flushleft}
\end{frame}
%%%%%%%%%%%%%%%%
\begin{frame}{Wymagania dotyczące operatora T}
  Operator T powinien zapewniać:
  \begin{enumerate}
    \item zgodność aproksymacji różnicowej 
    \item dokładność aproksymacji różnicowej
    \item stabilność schematu różnicowego
    \item efektywność schematu różnicowego
  \end{enumerate}
\end{frame}
%%%%%%%%%%%%%%%%
\begin{frame}{Zgodność aproksymacji różnicowej}
  \begin{block}{Wymóg podstawowy}
  	w granicy - układ różnicowy indentyczny z różniczkowym
  \end{block}
\end{frame}
%%%%%%%%%%%%%%%%
\begin{frame}{Dokładność aproksymacji różnicowej}
  \begin{itemize}
    \item błędy obcięcia\par 
    - ciągła zmienna niezależna \par
    \quad $\Rightarrow$ zbiór punktów (wybór schematu różnicowego) \par
    \quad $\Rightarrow$  od $\Delta$, $\Delta t$
    \item błędy \par
    \quad - ograniczona dokładność reprezentacji \textit{(fl. - arithmetic)}
  \end{itemize}
\end{frame}
%%%%%%%%%%%%%%%%
\begin{frame}{Stabilność schematu różnicowego}
  \begin{block}{Stabilność metody numerycznej}
    Metoda numeryczna jest stabilna, gdy mały błąd na dowolnym etapie przenosi się z minimalną amplitudą
    $$\varepsilon^{n+1} = g \cdot \varepsilon^{n}$$
    gdzie: \par
    \quad $\varepsilon$ - amplituda błędu \par
    \quad $g$ - współczynnik wzmocnienia $\sim$ (T)
  \end{block}
  stabilność $\Rightarrow $ \[\abs{g} < 1\]
  \textit{układ równań:} $\vec{\varepsilon}^{n}$, G  - macierz wzmocnienia
\end{frame}
%%%%%%%%%%%%%%%%
\begin{frame}{Efektywność schematu różnicowego}
$$\begin{array}{ll}
czas \\
\textit{pamięć}
\end{array} \left.\right\} \Rightarrow y$$
\end{frame}
