\subsection{Deflacja czynnikiem kwadratowym: $ x^2 +px + q $}

\begin{frame}{Deflacja czynnikiem kwadratowym: $ x^2 +px + q $}
  \begin{block}{}
    $$ f(x) = \sum_{i=0}^m a_i x^i = (x^2 + px + q) \cdot \sum_{i=0}^{m-2} c_i x^i + Rx +S $$
  \end{block}

  \vspace{6mm}

  \textbf{Dowód: } Przez porównanie współczynników:

  $$ \left \{ \begin{array}{l}
  c_{m-2} = a_m \\
  c_{m-3} = a_{m-1} - p \cdot c_{m-2} \\
  c_i = a_{i+2} - p \cdot c_{i+1} - q \cdot c_{i + 2}, \qquad i = m - 4, m - 5, \dots , 0 \\
  R = a_1 - p \cdot c_0 - q \cdot c_1 \\
  S = a_0 - q \cdot c_0
  \end{array} \right. $$
\end{frame}

\begin{frame}
  Można to zapisać wygodniej, przyjmując $ c_m = c_{m-1} = 0 $, $ c_{-1} = R $:

  \begin{block}{}
    $$ \left \{ \begin{array}{l}
    c_i = a_{i+2} - p \cdot c_{i+1} - q \cdot c_i + 2, \qquad i = m-2, m-3, \dots , 0, -1 \\
    S = a_0 - q \cdot c_0
    \end{array} \right. $$
  \end{block}

  \textbf{Zadanie:} Sprawdzić.

  \vspace{5px}

  Oczywiście: $c_i$, $R$, $S$ -- zależne od wybranego $p$ i $q$.

  \vspace{5px}

  Jeżeli $p,q$ takie, że $R = 0$ i $S = 0$ to \\ $x^2 + px + q$ -- czynnik kwadratowy $f$, \\Z niego otrzymujemy dwa pierwiastki funkcji $f$ (ewentualnie zespolone).
\end{frame}
