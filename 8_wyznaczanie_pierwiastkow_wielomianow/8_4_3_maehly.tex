\subsection{Metoda Maehly'ego -- technika wygładzania pierwiastków}

\begin{frame}{Met. Maehly'ego -- technika wygładzania pierwiastków}
  Zapobiega zlewaniu się w jeden dwóch różnych pierwiastków (na~etapie wygładzania) $\leftarrow$ równocześnie unikamy deflacji.

  \begin{block}{Zredukowany wielomian}
    $$P_j(x) \equiv \frac{P(x)}{(x - x_1)\ldots(x - x_j)}$$
  \end{block}

  $$P'_j(x) = \frac{P'(x)}{(x - x_1)\ldots(x - x_j)} - \frac{P(x)}{(x - x_1)\ldots(x - x_j)} \sum_{i=1}^j \frac{1}{x - x_i}$$
\end{frame}

\begin{frame}
  Pojedynczy krok metody Newtona-Raphsona można zapisać:

  $$x_{k+1} = x_k - \frac{P_j(x_k)}{P'_j(x_k)}$$

  czyli:

  $$x_{k+1} = x_k - \frac{P(x_k)}{P'(x_k) - P(x_k) \cdot \sum_{j=0}^j(x - x_i)^{-1}}$$ % INDEKS JEST NA PEWNO ZŁY.

  \textbf{Zadanie:} i- po węzłach już wygładzonych % NIE MAM POJĘCIA, O CO CHODZI, CZEGOŚ TU BRAKUJE...

  $\Rightarrow$ zero suppression.
\end{frame}
