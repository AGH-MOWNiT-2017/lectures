\subsection{Pierwiastki zespolone -- metoda Bairstowa}

\begin{frame}{Pierwiastki zespolone -- metoda Bairstowa}
  \textit{(można też metodą N-R)}

  \vspace{5px}

  \textbf{Metoda Bairstowa} polega na szukaniu czynników kwadratowych

  $$x_1 = \alpha + \textit{\textrm{i}}\beta, \quad x_2 = \alpha - \textit{\textrm{i}}\beta$$
  $$(x - x_1)(x - x_2) = x^2 - 2 \cdot \alpha x + (\alpha^2 + \beta^2) = x^2 + p \cdot x + q$$

  czynnik kwadratowy może objąć 2 pierwiastki rzeczywiste lub zespolone

  \begin{equation}
    P(x) = (x^2 + p \cdot x + q) \cdot Q(x) + R \cdot x + S \label{bairstow}
  \end{equation}
\end{frame}

\begin{frame}
  $$\left. \begin{array}{l}
  R(p,q)=0 \\ S(p,q)=0
  \end{array}\right\} \Rightarrow \text{met. Newtona-Raphsona}$$

  \vspace{5mm}

  $$\left( \begin{array}{l}
  p^{(n+1)} \\ q^{(n+1)}
  \end{array} \right)
  =
  \left( \begin{array}{l}
  p^{(n)} \\ q^{(n)}
  \end{array} \right)
  +
  \left( \begin{array}{ll}
  \frac{{\partial}R}{{\partial}p} & \frac{{\partial}R}{{\partial}q} \\
  \frac{{\partial}S}{{\partial}p} & \frac{{\partial}S}{{\partial}q}
  \end{array} \right)_n^{-1}
  \left( \begin{array}{l}
  R^{(n)} \\ S^{(n)}
  \end{array} \right)$$
\end{frame}

\begin{frame}
  Pochodne znajdujemy w następujący sposób:

  $P(x)$ -- nie zależy od $p$, $q$, \\ z (\ref{bairstow}) mamy:

  $$\left. \begin{array}{l}
  0 = (x^2 + p \cdot x + q) \cdot \frac{{\partial}Q}{{\partial}q} + Q + \frac{{\partial}R}{{\partial}q} + \frac{{\partial}S}{{\partial}q} \\ % NIE POWINNO BYĆ *x PRZY R'???
  0 = (x^2 + p \cdot x + q) \cdot \frac{{\partial}Q}{{\partial}p} + x \cdot Q + \frac{{\partial}R}{{\partial}p} + \frac{{\partial}S}{{\partial}p} % NIE POWINNO BYĆ *x PRZY R'???
  \end{array}\right\}
  \begin{array}{l}
    \left( \leftarrow \frac{{\partial}R}{{\partial}q} \right) \\
    \left( \leftarrow \frac{{\partial}R}{{\partial}q} \right) % NAPRAWDĘ NIE POWINNO BYĆ PO p?
  \end{array}$$

  \begin{equation}
    \left. \begin{array}{l}
    (x^2 + p \cdot x + q) + \frac{{\partial}Q}{{\partial}q} + \frac{{\partial}R}{{\partial}q} \cdot x + \frac{{\partial}S}{{\partial}q} = -Q(x) \\ % NIE POWINNO BYĆ * dQ/dq ??? (zamiast +)
    (x^2 + p \cdot x + q) + \frac{{\partial}Q}{{\partial}p} + \frac{{\partial}R}{{\partial}p} \cdot x + \frac{{\partial}S}{{\partial}p} = -x \cdot Q(x) % NIE POWINNO BYĆ * dQ/dp ??? (zamiast +)
    \end{array}\right\}
    \label{bairstow2}
  \end{equation}
\end{frame}

\begin{frame}
  (\ref{bairstow2}) mają podobną strukturę jak (\ref{bairstow}) $\rightarrow$ pochodne $\frac{{\partial}R}{{\partial}p}$, $\frac{{\partial}R}{{\partial}q}$, $\frac{{\partial}S}{{\partial}p}$, $\frac{{\partial}S}{{\partial}q}$ \\ \vspace{2mm} mogą być uzyskane przez syntetyczne dzielenie wielomianów \\ \vspace{3mm} $\left.\begin{array}{l} -Q(x) \\ -x Q(x) \end{array} \right\}$ przez czynnik kwadratowy.
\end{frame}
