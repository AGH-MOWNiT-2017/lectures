\section{Deflacja -- dzielenie syntetyczne}

\begin{frame}{Deflacja -- dzielenie syntetyczne}
  \textbf{Deflacja} -- bardzo użyteczny element wyznaczania pierwiastków wielomianów. % Deflacja – obniżenie stopnia wielomianu po znalezieniu jego pierwiastka; bardzo użyteczna część algorytmu wyznaczania pierwiastków wielomianów.

  \begin{block}{Nested form (Horner)}
    $$ f(x) = \sum_{i=0}^m a_i \cdot x^i = ( ( \dots (( a_m \cdot x + a_{m-1}) \cdot x + a_{m-2} ) \cdot \dots) \cdot x + a_1) \cdot x + a_0 $$
  \end{block}
\end{frame}

\begin{frame}
  Rekurencyjny algorytm obliczania wartości wielomianu dla $x = \lambda$ ma postać:

  \begin{block}{Algorytm}
    $$ \left \{ \begin{array}{l}
    b_{m-1} = a_m \\
    b_i = a_{i+1} + \lambda \cdot b_{i+1}, \quad i = m - 2, m-3, \dots, 0 \\
    f( \lambda ) = a_{0} + \lambda \cdot b_{0}
    \end{array} \right. $$
  \end{block}
\end{frame}

\begin{frame}

  \begin{block}{Twierdzenie}
    $$ \frac{f(x) - f(\lambda)}{x - \lambda} = \sum_{i=0}^{m-1} b_i x^i $$
  \end{block}

  Dowód:

  \begin{itemize}
    \item mnożenie przez $(x - \lambda)$
    \item $ b_{i-1} = a_i + \lambda \cdot b_i $
  \end{itemize}

  \vspace{5px}

  \textbf{Zadanie:} Przeprowadzić dowód.
\end{frame}
