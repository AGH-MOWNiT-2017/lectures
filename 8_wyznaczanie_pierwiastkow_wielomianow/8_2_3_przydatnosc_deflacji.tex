\subsection{Przydatność deflacji}

\begin{frame}{Przydatność deflacji}
  $P(x)$ -- wielomian,

  $r$ -- znaleziony pierwiastek wielomianu $P$.

  Realizujemy \textit{faktoryzację}:
  \begin{block}{}
    $$ P(x) = (x - r) \cdot Q(x) $$
  \end{block}

  \begin{itemize}
    \item Zmniejszenie złożoności obliczeniowej\\
     ($Q$ -- niższego stopnia niż $P$)
    \item Uniknięcie pomyłki -- powrotu do pierwiastka już znalezionego.
  \end{itemize}
\end{frame}

\begin{frame}
  Stosowanie deflacji musi być \textbf{ostrożne!}

  \textbf{Powód:} -- pierwiastki są wyznaczane ze skończoną dokładnością (kilka deflacji pociąga kumulację błędu).

  \begin{block}{}
    \begin{itemize}
      \item \textbf{Forward deflation} -- nowe współczynniki $Q(x)$ obliczane od najwyższych potęg $x$ \dots Stabilna, gdy zaczynamy od pierwiastków o \textbf{najmniejszej} wartości bezwzględnej,
      \item \textbf{Backward deflation} -- współczynniki $Q(x)$ wyznaczane poczynając od wyrazu wolnego \dots Stabilna, gdy zaczynamy od pierwiastków o \textbf{największej} wartości bezwzględnej.
    \end{itemize}
  \end{block}

  \textbf{Zadanie:} Dlaczego?
\end{frame}

\begin{frame}
  \begin{block}{Podejście minimalizujące błędy}
    \begin{itemize}
      \item Kolejne pierwiastki uzyskane w procesie deflacji są \textit{próbnymi (tentative)},
      \item \textit{Polishing (re-solving)} -- wygładzanie -- z użyciem pełnego $P(x)$ np. metodą Newtona-Raphsona,
      \item Niebezpieczeństwo -- zlanie się dwóch pierwiastków w jeden.\\
        Dodatkowa deflacja tylko jeden raz.
    \end{itemize}
  \end{block}
\end{frame}
