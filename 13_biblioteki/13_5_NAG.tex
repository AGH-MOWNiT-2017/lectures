% !TEX encoding = UTF-8 Unicode
\section{NAG - Numerical PVM Library}
%%%%%%%%%%%%%%%%
	\begin{frame}{NAG - Numerical PVM Library}
	W oparciu o: 
		\begin{itemize}
			\item PVM (Parallel Virtual Machine)
			\item BLACS
		\end{itemize}
		Zgodne ze ScaLAPACK \\
		Model programowania: SPMD (Single Program Multiple Data)
		\begin{itemize}
			\item prostota - jeden kod źródłowy programu	
			\item w bibliotece najmniej zmian w stosunku do biblioteki NAG F77
		\end{itemize}
		Zaplanowano możliwości konwersji do HPF, topologia procesorów: 2-D grid. 
	\end{frame}
	\begin{frame}{Zawartość NAG}
		\begin{itemize}
			\item algebra liniowa:
				\begin{itemize}
					\item rozwiązywanie układów równań gęstych i rzadkich,
					\item faktoryzacja LU, Cholesky,
					\item SVD,
					\item symetryzacja zagadnienia własne.
				\end{itemize}
			\item kwadratury,
			\item minimalizacja.
			\item generatory liczb pseudolosowych.
		\end{itemize}
		
	\end{frame}