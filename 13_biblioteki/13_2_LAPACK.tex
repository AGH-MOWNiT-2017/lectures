\section{LAPACK - Linear Algebra Package}
%%%%%%%%%%%%%%%%
	\begin{frame}{LAPACK -- Linear Algebra Package}
		\begin{itemize}
			\item Biblioteka \textit{Open Source} -- BSD-new
			\item Implementacja w języku Fortran 90.
			\item Pierwsze wydanie: 1.0 -- 29 luty 1992 r.
			\item Obecne wydanie: 3.8.0 -- 12 listopad 2017 r.
		\end{itemize}
		Twórcy:
		\textbf{CRPC} - \textbf{C}enter for \textbf{R}esearch on \textbf{P}arallel \textbf{C}omputing:
		\begin{itemize}
			\item Rice,
			\item Tennessee,
			\item ORNL.
		\end{itemize}
	\end{frame}
	\begin{frame}{LAPACK - cechy}
		\begin{itemize}
			\item Projektowany dla procesorów wektorowych i równoległych, posiadających dzieloną pamięć.
			\item Łączy algorytmy z bibliotek LINPACK oraz EISPACK.
			\item Zbudowany na BLAS -- wykorzystanie optymalizacji.
			\item Złożoność projektu -- stan na grudzień 2018:
			\begin{itemize}
				\item $843$ tyś. linii kodu -- biblioteka
				\item $541$ tyś. linii kodu -- testy
			\end{itemize}
		\end{itemize}
	\end{frame}
	\begin{frame}{LAPACK - Linear Algebra Package}
		\begin{itemize}
			\item Rozwiązania problemów -- algorytmy blokowe
			\begin{itemize}
				\item Układy równań liniowych: $Ax = b$,
				\item Faktoryzacje LU, LQ, QR,
				\item Metoda najmniejszych kwadratów: $min_x ||Ax - b ||_2$,
				\item Rozkład według wartości osobliwych: $A = U\sum V^T$,
				\item Macierze i wektory własne: $Ax = \lambda x$.
			\end{itemize}

			\item Podział metod w LAPACK:
			\begin{itemize}
				\item simple and expert driver routines,
				\item computational routines,
				\item auxiliary routines.
			\end{itemize}
		\end{itemize}
	\end{frame}

	\begin{frame}{LAPACK -- konwencja nazw}
	Nazwa metody postaci: TXXYY \\
	\vspace{5mm}
	\textbf{T} -- Typ: \\
	\begin{itemize}
		\item S - Single Precision REAL*4 \\
		\item D - Double Precision REAL*8 \\
		\item C - Complex COMPLEX*8 \\
		\item Z - Double Complex COMPLEX*16
	\end{itemize}
	\end{frame}

	\begin{frame}{LAPACK -- konwencja nazw}
	\textbf{XX} -- Rodzaj macierzy:
	\vspace{5mm}
	\begin{columns}
		\begin{column}{0.5\textwidth}
		\scriptsize
	BD - bidiagonal \\
	DI - diagonal \\
	GB - general band \\
	GE - general full \\
	GG - general matrices, generalized problem \\
	GT - general tridiagonal \\
	HB - complex Hermitian Band \\
	HE - complex Hermitian \\
	HG - upper Hessenberg matrix, generalized problem \\
	HP - complex Hermitian, packed storage \\
	HS - upper Hessenberg \\
	OP - real orthogonal, packed storage \\
	OR - real orthogonal \\
	PB - symmetric or Hermitian positive definite band \\
	PO - symmetric or Hermitian positive definite \\
	\end{column}
	\begin{column}{0.5\textwidth}
	\scriptsize
	PP - symmetric or Hermitian positive definite packed storage \\
	PT - symmetric or Hermitian positive definite tridiagonal \\
	SB - real symmetric band \\
	SP - symmetric, packed storage \\
	ST - real symmetric tridiagonal \\
	SY - symmetric \\
	TB - triangular band \\
	TG - triangular matrices, generalized problem \\
	TP - triangular packed storage \\
	TR - triangular \\
	TZ - trapezoidal \\
	UN - complex unitary \\
	UP - complex unitary packed storage \\
		\end{column}
	\end{columns}
	\end{frame}
