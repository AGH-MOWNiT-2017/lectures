% !TEX encoding = UTF-8 Unicode
\section{Biblioteki do Javy}
%%%%%%%%%%%%%%%%
	\begin{frame}{Biblioteki do Javy}
		Rosnąca popularność języka Java spowodowała, iż wielu twórców bibliotek numerycznych zaczęło przenosić je do tego języka. Większość z nich nie jest jeszcze w pełni gotowa lub znajduje się w fazie testów. Jednym ze źródeł informacji może być strona WWW:
		\begin{thebibliography}{9}
			\setbeamertemplate{bibliography item}[online]
      				\bibitem{javanumerics}{JavaNumerics \newblock Online resources \newblock \url{http://math.nist.gov/javanumerics}}
		\end{thebibliography}
	\end{frame}
	\begin{subsection}{Jampack - JAva Matrix PACKage}
	\begin{frame}{Jampack - JAva Matrix PACKage}
		\begin{block}{Jampack}
			Jest pakietem wielu klas do prowadzenia obliczeń na macierzach. Obecna wersja Jampak-u jest w fazie wstępnej i niekompletnej. Została opublikowana, aby sprawdzić czy istnieje zapotrzebowanie na tego typu pakiet. Jampack został stworzony przez osoby pracujące w NIST oraz w University of Matyland. Twórcy Jampacka postanowili, aby miał cechy takie jak:
			\begin{itemize}
				\item Przyjazność użytkownikowi
				\item Możliwość dodawania nowych algorytmów
				\item Otwartość kodu, użytkownik może ingerować w kod źródłowy
				\item Możliwość uczestnictwa w rozbudowie przez użytkowników\
			\end{itemize}
		\end{block}
	\end{frame}
	\begin{frame}{Jampack - moduły}
		\begin{itemize}
			\item Macierze liczb zespolonych. Implementacja macierzy liczb rzeczywistych została odłożona na później.
			\item Indeksowanie macierzy (pierwszy element macierzy $a_{1,1}$ lub $a_{0,0}$ może być dowolnie wybrane przez programistę
			\item Zostały utworzone klasy dla odpowiednich typów macierzy. Zmat - ogólne, Zdigmat - macierze diagonalne. Klasa Zmat ma dwie podklasy macierzy trójkątnych oraz dodatnio określonych. 
			\item Zdefiniowano większość operacji na macierzach. 
			\item Faktoryzacje.
			\item Wielokrotne użycie obliczonych faktoryzacji macierzy, np. przy rozwiązywaniu układów równań ze zmianą jedynie wektora b.
			\item Zastosowane algorytmy nie są bardzo wyszukane. Pakiet ten nie zamierza konkurować z pakietami high-performance.
		\end{itemize}
		%%%%%%%%%%%%%%%%
		% DEAD LINK
		%http://gams.nist.gov/pub/Jampack
		%%%%%%%%%%%%%%%%
	\end{frame}
	\end{subsection}
	\begin{frame}[allowframebreaks]{Biblioteki do Javy}
		\begin{block}{JAMA}
			JAMA jest pakietem, który powstał równolegle z pakietem Jampack. Posiada on mniej zaimplementowanych metod. Obliczenia prowadzi na macierzach gęstych o wartościach rzeczywistych. Jej twórcy mają zamiar przedstawić pakiet jako standard dla macierzy. 
			\begin{thebibliography}{9}
				\setbeamertemplate{bibliography item}[online]
      					\bibitem{jama}{JavaNumerics - JAMA \newblock Online resources \newblock \url{http://math.nist.gov/javanumerics/jama}}
			\end{thebibliography}
			%%%%%%%%%%%%%%%%
			% DEAD LINK
			%/applets/num_lib/Decomp.html
			%%%%%%%%%%%%%%%%
		\end{block}
		\begin{block}{JLAPACK}
			Twórcy LAPACKA postanowili przenieść go na platformę JAVY. Implementuje on wszystkie metody oferowane przez LAPACKA. Pakiet ten nie został jeszcze w pełni zoptymalizowany. 
			%%%%%%%%%%%%%%%%
			% DEAD LINK
			%www.cs.utk.edu/f2j/download.html
			%%%%%%%%%%%%%%%%
		\end{block}
		\begin{block}{OR-Objects}
			Jest chyba najbogatszą z bibliotek. Oferuje on ponad 450 różnych funkcji operujących na macierzach, grafach, itp. Podstawowa wersja tego pakietu jest dostępna bezpłatnie pod adresem
			%%%%%%%%%%%%%%%%
			% DEAD LINK
			%http://opsresearch.org 				!!!!!! Mentioned in the text
			%%%%%%%%%%%%%%%%
		\end{block}
	\end{frame}