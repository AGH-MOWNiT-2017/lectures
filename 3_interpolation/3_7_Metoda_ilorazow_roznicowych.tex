\section{3.7 Metoda ilorazów różnicowych (divided differences)}
\begin{frame}
{3.7 Metoda ilorazów różnicowych (divided differences)}

Interesuje nas nie wartość, a postać wielomianu.

$P_{n}(x)-\text{LIP, stp. } \leq n$, zgodny z $f(x)$ w $\{x_{0},\ x_{1},\ .\ .\ .\ ,\ x_{n}\}$

Można go zapisać w postaci:
\begin{equation*}\begin{split}
P_{n}(x)=a_{0}+a_{1}(x-x_{0})+a_{2}(x-x_{0})(x-x_{1})+ \\
\cdots+a_{n}(x-x_{0})(x-x_{1})\cdots(x-x_{n-1})
\end{split}
\end{equation*}

$a_{0}$: $a_{0}=P_{n}(x_{0})=f(x_{0})$

$a_{1}$ : $f(x_{0})+a_{1}(x_{1}-x_{0})=P_{n}(x_{1})=f(x_{1}) \Rightarrow$

$$
a_{1}=\frac{f(x_{1})-f(x_{0})}{x_{1}-x_{0}}
$$
\end{frame}

\begin{frame}
Wprowadzamy notację:
\begin{itemize}
\item 0-wy iloraz różnicowy wzgl. $x_{i}$ : $f[x_{i}]=f(x_{i})$ \\
pozostałe - indukcyjnie:

\item 1-szy:
$$
f[x_{i},\ x_{i+1}]=\frac{f[x_{i+1}]-f[x_{i}]}{x_{i+1}-x_{i}}
$$
Gdy zaś określone są ilorazy aż do $(k-1)$ , czyli
$$
f[x_{i},\ x_{i+1},\ x_{i+k-1}] \: i \: f[x_{i+1},\ x_{i+2},\ x_{i+k}]
$$
\item to wtedy k-ty iloraz różnicowy:
\end{itemize}
$$
f[x_{i},\ x_{i+1}, ...\ x_{i+k}]=\frac{f[x_{i+1},x_{i+2},\ldots.x_{i+k}]-f[x_{i},x_{i+1},\ldots.x_{i+k-1}]}{x_{i+k}-x_{i}}
$$
\end{frame}

\begin{frame}
Interpolacyjny wzór Newtona z ilorazami różnicowymi:
$$
P_{n}(x)=f[x_{0}]+\sum_{k/1}^{n}f[x_{0},\ x_{1},\ .\ .\ .\ ,\ x_{k}](x-x_{0})\cdots(x-x_{k-1})
$$
\textbf{Zadanie}: Policzyć $a_{2}, a_{3}$, zapisać algorytm.

\end{frame}

\begin{frame}

\begin{block}
{Twierdzenie iloraz różnicowy a pochodna}

Założenia:
\begin{itemize}
\item $f\in C^{n}[a,\ b]$
\item $x_{0}, x_{1}$, . . . , $x_{n}\in[a,\ b] $ i są różne
\end{itemize}

Teza:
$$
\exists\eta\in(a,\ b)\ f[x_{0},\ x_{1},\ .\ .\ .\ ,\ x_{n}]=\frac{f^{(n)}(\eta)}{n!}
$$
\end{block}
\vspace{5mm}

\textbf{Zadanie:} Podać dowód.
\end{frame}

\begin{frame}
\textbf{Węzły równoodległe}

$h=x_{i+1}-x_{i} \quad i=0, 1, ..., n-1$ \\
$x=x_{0}+s*h$ \\
$x-x_{i}=(s-i)h$ \\

\begin{equation*} \begin{split} P_{n}(x)=P_{n}(x_{0}+s*h) = f[x_{0}]+s*h*f[x_{0}, x_{1}]+s(s - 1)h^{2}f[x_{0},x_{1},x_{2}]+ \\
+ \cdots +s(s-1)\cdots(s-n+1)h^{n}f[x_{0},x_{1},\dots ,x_{n}] \end{split} \end{equation*}

$\binom{s}{k}=\displaystyle \frac{s(s-1)\ldots(s-k+1)}{k!}$
$$
P_{n}(x)=P_{n}(x_{0}+sh)=\sum_{k/0}^{n}(_{k}^{s})k!h^{k}f[x_{0},\ x_{1},\ .\ .\ .\ ,\ x_{k}]
$$
\end{frame}

\begin{frame}
\begin{itemize}
\item Newton forward divided-difference formule {\it (r. progresywna)} \\
\vspace{2mm}
$f[x_{0},\displaystyle \ x_{1},\cdots,\ x_{k}]=\frac{1}{k!h^{k}}\triangle^{k}f(x_{0})$ \\
\vspace{3mm}
$\rightarrow$ różnica progresywna rzędu $k$
\end{itemize}

$$P_{n}(x)=\displaystyle \sum_{k/0}^{n}(_{k}^{s})\triangle^{k}f(x_{0})$$
\end{frame}
