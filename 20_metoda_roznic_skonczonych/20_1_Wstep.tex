\section{Wstęp}
\begin{frame}{Wstęp}
	\begin{exampleblock}{}
		 \[
    	\textrm{Analiza matematyczna - obiekty}
        \begin{cases}
        	\textrm{nieskończone} \\
            \textrm{ciągłe}
        \end{cases}
    	\] 
    	$\newline$
    	\[
    	\textrm{Arytmetyka komputerowa}
        \begin{cases}
        	\textrm{skończona} \\
            \textrm{dyskretna}
        \end{cases}
    	\]
	\end{exampleblock}
\end{frame}
\begin{frame}{Cel i sposób}
	\begin{block}{Cel:}
		Przełożenie równań procesów fizycznych na język arytmetyki 
    	komputerowej
	\end{block}
    \begin{block}{Sposób:}
     	Metoda różnic skończonych ($\rightarrow$ małe elementy ciągłego układu fizycznego) $\newline$
        \textbf{Istota:} zatrzymanie się na pośrednim etapie $\newline$
        na drodze do równań różniczkowych $\rightarrow$
        układy równań różniczkowych
        \[	\textrm{zmienne ciągłe}
        	\begin{rcases*}
        		\nearrow \textrm{czasowe - t} \\
                \searrow \textrm{przestrzenne - x}
        	\end{rcases*}
        	\rightarrow siatka
            \begin{cases}
            	\textrm{czasowa} \\
                \textrm{przestrzenna}
            \end{cases}
        \]
    \end{block}
     
\end{frame}