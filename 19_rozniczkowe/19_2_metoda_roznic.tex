% !TeX spellcheck = pl_PL
\section{Metoda różnic skończonych}

%%%%%%%%%%%%%%%% 
	\begin{frame}{Metoda różnic skończonych}
		Cel: Przybliżenie pochodnych różnicami skończonymi
		Wprowadzamy siatkę:
		\begin{itemize}
			\item $\Omega = (0, X), \varphi_i = \varphi(x_i), i = 0,1,...,N$
			\item $x_i = i \cdot h$ - punkty siatki					
			\item $h = \frac{X}{N}$ - odstępy między punktami		
		\end{itemize}

		$$\frac{\partial \varphi_i}{\partial x} = \frac{\varphi_{i+1} - \varphi_{i-1}}{2h^2} + O(h^2)$$ \\
		$$\frac{\partial^2 \varphi_i}{\partial x^2} = \frac{\varphi_{i+1} - 2 \varphi_i + \varphi_{i-1}}{h^2} + O(h^2)$$
	\end{frame}
%%%%%%%%%%%%%%%%
\begin{frame}{Metoda różnic skończonych}
		\begin{exampleblock}{Przejście z równania różniczkowego na układ równań}
		Przypomnijmy równanie z warunkami brzegowymi:
		$$
		\begin{cases}
		\varphi^{''}(x) + \varphi(x) = \alpha \\
		\varphi(0) = \varphi(\frac{\pi}{2}) = 1 \\
		\end{cases}
		$$
		Przybliżając pochodne różnicami skończonymi otrzymujemy:
		$$
		\begin{cases}
		\varphi_{i-1} + (h^2 - 2)\varphi_i + \varphi_{i+1} = h^2 \cdot \alpha \quad \forall_{i \in \{1,2, ..., N-1\}} \\
		\varphi_0 = 1 \\
		\varphi_N = 1
		\end{cases}
		$$
	\end{exampleblock}	
\end{frame}

%%%%%%%%%%%%%%%% 

\begin{frame}{Metoda różnic skończonych - zapis macierzowy}
	\begin{exampleblock}{Zapis macierzowy po włączeniu warunku brzegowego}
	{\scriptsize
	$$
	\begin{bmatrix}
	(h^2 -2) & 1 &  &  &  &  & \\ 
	1&  (h^2 -2)& 1 &  &  &  & \\ 
	& 1 &  (h^2 -2)& 1 &  &  & \\ 
	&  &  ...& &  &  & \\ 
	&  &  &  1 & (h^2 -2) & 1 & \\ 
	&  &  &  & 1 &  (h^2 -2) &
	\end{bmatrix}		
	\begin{bmatrix}
	\varphi_1 \\
	\varphi_2 \\
	\varphi_3 \\
	...\\
	\varphi_{N-2}\\
	\varphi_{N-1}\\
	\end{bmatrix}		
	= 		
	\begin{bmatrix} 
	h^2\alpha -1 \\
	h^2\alpha\\
	h^2\alpha\\
	... \\
	h^2\alpha\\
	h^2\alpha - 1\\
	\end{bmatrix}
	$$}
	
	Postać prawej strony (dla $i = 1$ oraz $i = N-1$) jest wynikiem włączenia warunków brzegowych w zapis macierzowy
	\end{exampleblock}
\end{frame}
%%%%%%%%%%%%%%%%
\begin{frame}{Metoda różnic skończonych}
	\begin{exampleblock}{Przejście z równania różniczkowego na układ równań}
		Przypomnijmy równanie z warunkami brzegowymi:
		$$
		\begin{cases}
		\varphi^{''}(x) + \varphi(x) = \alpha \\
		\varphi(0) = \varphi(\frac{\pi}{2}) = 1 \\
		\end{cases}
		$$
		Przybliżając pochodne różnicami skończonymi otrzymujemy:
		$$
		\begin{cases}
		\varphi_{i-1} + (h^2 - 2)\varphi_i + \varphi_{i+1} = h^2 \cdot \alpha \quad \forall_{i \in \{1,2, ..., N-1\}} \\
		\varphi_0 = 1 \\
		\varphi_N = 1
		\end{cases}
		$$
	\end{exampleblock}	
\end{frame}





