\section{Metoda różnic skończonych}

%%%%%%%%%%%%%%%% 
	\begin{frame}{Metoda różnic skończonych}
		Wprowadzamy siatkę:
		\begin{itemize}
			\item $x = i \cdot h, i = 0,1 ... , N$					
			\item $N \cdot h = \frac{\pi}{2}$		
		\end{itemize}

		$$\frac{d^2 \varphi}{dx^2} = \frac{\varphi_{i+1} - 2 \cdot \varphi_i + \varphi_{i-1}}{h^2} + O(h^2)$$

		\begin{block}{Równanie różniczkowe $\Rightarrow$ układ równań różniczkowych}
			$$
			\begin{cases}
				\varphi_{i-1} + (h^2 - 2)\varphi_i + \varphi_{i+1} = h^2 \cdot \alpha, i = 1,2, ..., N-1 \\
				\varphi_0 = 1 \\
				\varphi_N = 1
			\end{cases}
			$$
		\end{block}	
	\end{frame}

%%%%%%%%%%%%%%%% 

	\begin{frame}{Metoda różnic skończonych - zapis macierzowy}
		\textbf{Zapis macierzowy po włączeniu warunku brzegowego}
		$$
		\footnotesize
		\begin{bmatrix}
		(h^2 -2) & 1 &  &  &  &  & \\ 
		 1&  (h^2 -2)& 1 &  &  &  & \\ 
		 & 1 &  (h^2 -2)& 1 &  &  & \\ 
		 &  &  ...& &  &  & \\ 
		  &  & 1 & (h^2 -2) &  1& & \\ 
		  &  &  & 1 &  (h^2 -2)& 1 &
		\end{bmatrix}		
		\begin{bmatrix}
		 \varphi_1 \\
		 \varphi_2 \\
		 \varphi_3 \\
		 ...\\
		 \varphi_{N-2}\\
		 \varphi_{N-1}\\
		\end{bmatrix}		
		= 		
		\begin{bmatrix} 
		 h^2\alpha -1 \\
		 h^2\alpha\\
		 h^2\alpha\\
		 ... \\
		 h^2\alpha\\
		 h^2\alpha - 1\\
		\end{bmatrix}
		$$

		
		Postać prawej strony $(i = 1; i = N-1)$ $\leftarrow$ wynik włączenia warunku brzegowego w zapis macierzowy.
		
	\end{frame}





