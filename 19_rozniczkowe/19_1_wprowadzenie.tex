\section{Wprowadzenie}

%%%%%%%%%%%%%%%% 
	\begin{frame}{Wprowadzenie}
		\begin{itemize}
			\item Większość zagadnień nauki i techniki jest formułowanych w postaci równań (lub układów równań) różniczkowych z zadanymi warunkami granicznymi (początkowymi lub brzegowymi)
					
			\item W większości przypadków brak rozwiązań analitycznych $\rightarrow$ konieczność przybliżonego znajdowania rozwiązań.
		\end{itemize}
	\end{frame}
	
%%%%%%%%%%%%%%%%
		
	\begin{frame}{Metody algebraizacji równań różniczkowych}

		\begin{exampleblock}{Przykład: problem brzegowy dla równania różniczkowgo zwyczajnego}
			$$
			\begin{cases}
				\varphi_{xx}(x) + \varphi(x) = \alpha \\
				\varphi(0) = \varphi(\frac{\pi}{2}) = \alpha \\
				x \in [0,\frac{\pi}{2}]
			\end{cases}
			$$
			
			Dokładnie jedno rozwiązanie:
			$$
			\varphi(x) = (1- \alpha) \cdot (sin(x) + cos(x) + \alpha)
			$$			
		\end{exampleblock}
	
	\end{frame}
%%%%%%%%%%%%%%%%	
