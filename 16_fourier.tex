\documentclass[aspectratio=43,english]{beamer} %If you want to create Polish presentation, replace 'english' with 'polish' and uncomment 3-th line, i.e., '\usepackage{polski}'
\usepackage[utf8]{inputenc}
\usepackage{polski} %Uncomment for Polish language
\usepackage{babel}
\usepackage{listings} %We want to put listings

\mode<beamer>{ 	%in 'beamer' mode
	\hypersetup{pdfpagemode=FullScreen}		%Enable Full screen mode
	\usetheme{JuanLesPins} 		%Show part title in right footer
	%\usetheme[dark]{AGH}                 		%Use dark background
	%\usetheme[dark,parttitle=leftfooter]{AGH}  	%Use dark background and show part title in left footer
}
\mode<handout>{	%in 'handout' mode
	\hypersetup{pdfpagemode=None}		
	\usepackage{pgfpages}
  	\pgfpagesuselayout{4 on 1}[a4paper,border shrink=5mm,landscape]	%show 4 slides on 1 page
  	\usetheme{boxes}
  	\addheadbox{structure}{\quad\insertpart\hfill\insertsection\hfill\insertsubsection\qquad} 	%content of header
 	\addfootbox{structure}{\quad\insertauthor\hfill\insertframenumber\hfill\insertsubtitle\qquad} 	%content of footer
}

\AtBeginPart{ %At begin part: display its name
	\frame{\partpage}
} 


%%%%%%%%%%% Configuration of the listings package %%%%%%%%%%%%%%%%%%%%%%%%%%
% Source: https://en.wikibooks.org/wiki/LaTeX/Source_Code_Listings#Using_the_listings_package
%%%%%%%%%%%%%%%%%%%%%%%%%%%%%%%%%%%%%%%%%%%%%%%%%%%%%%%%%%%%%%%%%%%%%%%%%%%%
\lstset{ %
  backgroundcolor=\color{white},   % choose the background color
  basicstyle=\footnotesize,        % the size of the fonts that are used for the code
  breakatwhitespace=false,         % sets if automatic breaks should only happen at whitespace
  breaklines=true,                 % sets automatic line breaking
  captionpos=b,                    % sets the caption-position to bottom
  commentstyle=\color{green},      % comment style
  deletekeywords={...},            % if you want to delete keywords from the given language
  escapeinside={\%*}{*)},          % if you want to add LaTeX within your code
  extendedchars=true,              % lets you use non-ASCII characters; for 8-bits encodings only, does not work with UTF-8
  frame=single,	                   % adds a frame around the code
  keepspaces=true,                 % keeps spaces in text, useful for keeping indentation of code (possibly needs columns=flexible)
  keywordstyle=\color{blue},       % keyword style
  morekeywords={*,...},            % if you want to add more keywords to the set
  numbers=left,                    % where to put the line-numbers; possible values are (none, left, right)
  numbersep=5pt,                   % how far the line-numbers are from the code
  numberstyle=\tiny\color{gray},   % the style that is used for the line-numbers
  rulecolor=\color{black},         % if not set, the frame-color may be changed on line-breaks within not-black text (e.g. comments (green here))
  showspaces=false,                % show spaces everywhere adding particular underscores; it overrides 'showstringspaces'
  showstringspaces=false,          % underline spaces within strings only
  showtabs=false,                  % show tabs within strings adding particular underscores
  stepnumber=2,                    % the step between two line-numbers. If it's 1, each line will be numbered
  stringstyle=\color{cyan},        % string literal style
  tabsize=2,	                   % sets default tabsize to 2 spaces
  title=\lstname,                  % show the filename of files included with \lstinputlisting; also try caption instead of title
                                   % needed if you want to use UTF-8 Polish chars
  literate={?}{{\k{a}}}1
           {?}{{\k{A}}}1
           {?}{{\k{e}}}1
           {?}{{\k{E}}}1
           {�}{{\'o}}1
           {�}{{\'O}}1
           {?}{{\'s}}1
           {?}{{\'S}}1
           {?}{{\l{}}}1
           {?}{{\L{}}}1
           {?}{{\.z}}1
           {?}{{\.Z}}1
           {?}{{\'z}}1
           {?}{{\'Z}}1
           {?}{{\'c}}1
           {?}{{\'C}}1
           {?}{{\'n}}1
           {?}{{\'N}}1
}
%%%%%%%%%%%%%%%%%


\title{Metody Obliczeniowe w Nauce i Technice}
\author{Marian Bubak, PhD}
\date{}
\institute[AGH]{
	Institute of Computer Science\\ul. Kawiory 21\\30-055 Krakow\\
	Poland\\
	\url{http://www.icsr.agh.edu.pl/~mownit/}
}



%%%%%%%%%%%%%%%%
\usepackage{amsmath}
\usepackage{mathtools}
\usepackage{xfrac}
\usepackage{breqn}
\usepackage{caption}
\usepackage{enumerate}
%%%%%%%%%%%%%%%%
\makeatletter
\newenvironment<>{proofs}[1][\proofname]{%
	\par
	\def\insertproofname{#1\@addpunct{.}}%
	\usebeamertemplate{proof begin}#2}
{\usebeamertemplate{proof end}}
\makeatother
%%%%%%%%%%%%%%%%
\DeclareMathOperator\cas{cas}
%%%%%%%%%%%%%%%%
\newcommand*{\blockbreak}{\usebeamertemplate{block end}\usebeamertemplate{block begin}}
%%%%%%%%%%%%%%%%

\subtitle{16. Szeregi i transformaty Fouriera}
\setcontributors{Maciej Trzebiński\\Mikołaj Biel}


\begin{document}
	\maketitle
    %%%%%%%%%%%%%%%%
    \begin{frame}{Outline}
    	\tableofcontents
    \end{frame}

    %%%%%%%%%%%%%%%%
    \section{Wstęp}
%%%%%%%%%%%%%%%%
\begin{frame}{Wstęp}
	\begin{enumerate}[a)]
		\item metody spektralne:
		\begin{itemize}
			\item badanie zjawisk okresowych: szeregi czasowe (np. sterowanie)
			\item badanie zjawisk nieokresowych:
			\begin{itemize}
				\item przedłużanie okresowe funkcji
				\item całki Fouriera
			\end{itemize}
		\end{itemize}
        \item algorytmy numeryczne:
        \begin{itemize}
        	\item opracowanie
        	\item analiza
        \end{itemize}	
	\end{enumerate}
\end{frame}
    %%%%%%%%%%%%%%%%
	\section{Podstawowe własności szeregów i transformat Fouriera}
%%%%%%%%%%%%%%%%
\begin{frame}[allowframebreaks]{Sposoby opisu procesu fizycznego}
	\begin{enumerate}
		\item w dziedzinie czasu (time domain) $ \implies A(t) $ \\
		$ t - \text{czas}, \quad A - \text{pewna wielkość} $
		%%%%%%%%%%%%%%%%	
		\item w dziedzinie $
		\begin{array}{l}
		\text{częstości} (\omega) \\ \text{częstotliwości} (f)
		\end{array}$
		$\implies \widehat{A}(\omega) \quad \omega = 2 \pi f$
	\end{enumerate}
	$A(t), \quad \widehat{A}(\omega) \,\, - $ dwie różne reprezentacje tej samej funkcji związane równaniami transformat Fouriera:
	%%%%%%%%%%%%%%%%		
	\begin{block}
	\centering
	\renewcommand{\arraystretch}{1.5}
	\setlength{\abovedisplayskip}{0pt}
	\setlength{\belowdisplayskip}{0pt}
	\setlength{\abovedisplayshortskip}{0pt}
	\setlength{\belowdisplayshortskip}{0pt}
	\[
	\begin{rcases*}
		A(t) = \int\limits_{-\infty}^{\infty}\frac{d \omega}{2\pi}\widehat{A}(\omega) \cdot e^{i \omega t} \\
		\widehat{A}(\omega) = \int\limits_{-\infty}^{\infty}dt \cdot A(t) \cdot e^{-i \omega t}
	\end{rcases*}
	\]
	\end{block}
	lub:
	\begin{block}
	\centering
	\renewcommand{\arraystretch}{1.5}
	\setlength{\abovedisplayskip}{0pt}
	\setlength{\belowdisplayskip}{0pt}
	\setlength{\abovedisplayshortskip}{0pt}
	\setlength{\belowdisplayshortskip}{0pt}
	\[
	\begin{rcases*}
		A(t) = \int\limits_{-\infty}^{\infty}df \cdot \widehat{A}(f) \cdot e^{2 \pi ift} \\
		\widehat{A}(f) = \int\limits_{-\infty}^{\infty}dt \cdot A(t) \cdot e^{-2 \pi ift}
	\end{rcases*}
	\begin{array}{r}
		- \,\, \text{nie trzeba pamiętać}\\ \text{o czynniku} \frac{1}{2 \pi}
	\end{array}
	\]
	\end{block}
	$t - \text{czas} \rightleftharpoons \omega - \text{częstość kołowa}$
	\\ $x - \text{położenie} \rightleftharpoons k = \frac{2 \pi}{\lambda} - \text{wektor falowy (liczba falowa)}$
	\\ $\vdots$
	\\ $A(t), \quad \widehat{A}(\omega) \,\, - $ ciągłe f. swych argumentów
	\[
		A(t) \rightleftharpoons \widehat{A}(\omega); \widehat{A}(\omega) = \Gamma[A(t)]
	\]
\end{frame}
%%%%%%%%%%%%%%%%
\begin{frame}{Definicje transformat Fouriera I}
	\begin{enumerate}[a)]
		\item FT - transformata Fouriera (Fourier Transform) \\
		$x$ - ciągłe \\
		$k$ - ciągłe \\
		$A(x), \widehat{A}(k)$ - f. ciągłe
		\begin{block}
		\centering
		\renewcommand{\arraystretch}{1.5}
		\setlength{\abovedisplayskip}{0pt}
		\setlength{\belowdisplayskip}{0pt}
		\setlength{\abovedisplayshortskip}{0pt}
		\setlength{\belowdisplayshortskip}{0pt}
		\[
			\begin{array}{c}
			\widehat{A}(k) = \int\limits_{-\infty}^{\infty}dx A(x) e^{-ikx} \\
			A(x) = \int\limits_{-\infty}^{\infty} \frac{dk}{2 \pi} \widehat{A}(k) e^{ikx}
			\end{array}
			\tag{16.1}
		\]
		\end{block}
	\end{enumerate}
\end{frame}
%%%%%%%%%%%%%%%%
\begin{frame}{Definicje transformat Fouriera II}
	\begin{enumerate}[b)]
		\item FS(i) - szereg Fouriera (Fourier Series) \\
		$x$ - ciągła \\
		$B(x)$ - f. okresowa ciągłej zmiennej x; okres: L
		\begin{center}
			(ciągła odcinkami wraz z pochodną: na tych odcinkach - szereg zbieżny do B(x), w punktach nieciągłości - do wartości średniej)
		\end{center}
		\begin{block}
		\centering
		\renewcommand{\arraystretch}{1.5}
		\setlength{\abovedisplayskip}{0pt}
		\setlength{\belowdisplayskip}{0pt}
		\setlength{\abovedisplayshortskip}{0pt}
		\setlength{\belowdisplayshortskip}{0pt}
		\[
			\begin{array}{c}
			\widehat{B}(k) = \int\limits_{L}dx B(x) e^{-ikx} \\
			B(x) = \frac{1}{L}\sum\limits_{l = -\infty}^{\infty} \widehat{B}(k) e^{ikx}
			\end{array}
			\tag{16.2}
		\]
		\end{block}
		\begin{tabular}{ll}
			$l$ & - liczba całkowita \\
			$k$ & - dyskretne
		\end{tabular}
		\hfill $k = \underbrace{\frac{2 \pi}{L}}_{k_0} \cdot l = k_0 \cdot l$
	\end{enumerate}
\end{frame}
%%%%%%%%%%%%%%%%
\begin{frame}{Definicje transformat Fouriera III}
	\begin{enumerate}[c)]
		\item FS(ii) - szereg Fouriera \\
		$x_p$ - dyskretne o skoku H \\
		$x_p = p \cdot H$ \\
		$p$ - liczba całkowita
		\begin{block}
		\centering
		\renewcommand{\arraystretch}{1.5}
		\setlength{\abovedisplayskip}{0pt}
		\setlength{\belowdisplayskip}{0pt}
		\setlength{\abovedisplayshortskip}{0pt}
		\setlength{\belowdisplayshortskip}{0pt}
		\[
			\begin{array}{c}
			\widehat{C}(k) = H \cdot \sum\limits_{p = -\infty}^{\infty} C(x_p) e^{-ikx} \\
			C(x_p) = \int\limits_{k_g} \frac{dk}{2 \pi} \widehat{C}(k) e^{ikx_p}
			\end{array}
			\tag{16.3}
		\]
		\end{block}
		$k$ - ciągła \\
		$\widehat{C}(k)$ - periodyczna \\
		okres $k_g = \frac{2 \pi}{H}$
	\end{enumerate}
\end{frame}
%%%%%%%%%%%%%%%%
\begin{frame}{Definicje transformat Fouriera IV}
	\begin{enumerate}[d)]
		\item fFT - skończona transformata Fouriera (finite FT) \\
		$x_p$ - dyskretne o skoku H \\
		$x_p = p \cdot H$ \\
		$D(x_p)$ - okresowa; okres: L \\
		$N$ - ilość punktów w okresie $D(x_p)$ \\
		$k$ - dyskretna, skok $k_0 = \frac{2 \pi}{L}; k = l \cdot k_0$ \\
		$\widehat{D}(k)$ - okresowa, okres $k_g = \frac{2 \pi}{H}$
		\begin{block}
		\centering
		\renewcommand{\arraystretch}{1.5}
		\setlength{\abovedisplayskip}{0pt}
		\setlength{\belowdisplayskip}{0pt}
		\setlength{\abovedisplayshortskip}{0pt}
		\setlength{\belowdisplayshortskip}{0pt}
		\[
			\begin{array}{c}
			\widehat{D}(k) = H \cdot \sum\limits_{p = 0}^{N-1} D(x_p) e^{-ikx_p} \\
			D(x_p) = \frac{1}{L} \sum\limits_{l = 0}^{N-1} \widehat{D}(k) e^{ikx_p}
				\end{array}
			\tag{16.4}
		\]
		\end{block}
	\end{enumerate}
\end{frame}
%%%%%%%%%%%%%%%%
\begin{frame}{Związki między transformatami Fouriera}
	\begin{enumerate}
		\item przez przejścia graniczne
	\end{enumerate}
\end{frame}
%%%%%%%%%%%%%%%%
\begin{frame}{Symetrie funkcji i jej transformaty I}
	Transformaty Fouriera $\to$ liniowe
	\\ Oznaczenia:
	\begin{itemize}
		\item E - even (parzysty)
		\item O - odd (nieparzysty)
		\item r - real (rzeczywisty)
		\item i - imaginary (urojony)
	\end{itemize}
	Wszystkie 4 transformaty Fouriera mają te same własności symetrii:
	\begin{align*}
		f(x) = E_r(x) + i \cdot E_i(x) + O_r(x) + i \cdot O_i(x) \widehat{f}(k) =
		\tag{16.5} \\
		E_r(k) + i \cdot E_i(k) + i \cdot O_i(k) + O_r(k)
		\tag{16.6}
	\end{align*}
\end{frame}
%%%%%%%%%%%%%%%%
\begin{frame}{Symetrie funkcji i jej transformaty II}
	\begin{table}
		\centering
		\begin{tabular}{|c|c|}
			\hline
			$f(x)$ & $\widehat{f}(k)$ \\
			\hline
			\hline
			$r \land E$ & $r \land E$ \\
			\hline
			$r \land E$ & $i \land E$ \\
			\hline
			$i \land E$ & $i \land E$ \\
			\hline
			$i \land E$ & $r \land E$ \\
			\hline
			$r$ & hermitowska \\
			\hline
			$i$ & antyherminowska \\
			\hline
			$E$ & $E$ \\
			\hline
			$O$ & $O$ \\
			\hline
		\end{tabular}
	\end{table}
\end{frame}
%%%%%%%%%%%%%%%%
\begin{frame}[allowframebreaks]{Zestawienie własności transformat}
	poza własnościami dotyczącymi pochodnej - obowiązują dla wszystkich 4 transformat \\
	$Z: f(x) \rightleftharpoons \widehat{f}(k); \quad g(x) \rightleftharpoons \widehat{g}(k)$
	\begin{block}
		\centering
		\renewcommand{\arraystretch}{1.5}
		\setlength{\abovedisplayskip}{0pt}
		\setlength{\belowdisplayskip}{0pt}
		\setlength{\abovedisplayshortskip}{0pt}
		\setlength{\belowdisplayshortskip}{0pt}
		\[
		\tag{16.7}
		\begin{array}{@{}ll}
			\text{podobieństwo:} & f(\frac{x}{a}) \rightleftharpoons |a| \cdot \widehat{f}(k \cdot a) \\
			\text{mnożenie przez stałą:} & b \cdot f \rightleftharpoons b \cdot \widehat{f} \\
			\text{suma:} & f + g \rightleftharpoons \widehat{f} + \widehat{g} \\
			\text{odwrotność:} &
			\renewcommand{\arraystretch}{1}
			\begin{array}{@{}l}
				\text{jeżeli:} \quad f(x) \rightleftharpoons \widehat{f}(k) = g(k) \\
				\text{to:} \quad g(x) \rightleftharpoons \widehat{g}(k) = 2 \pi \cdot f(-k)
			\end{array} \\
			\text{przesunięcie:} & f(x + a) \rightleftharpoons e^{ika} \cdot \widehat{f}(k) \\
			\text{pochodna:} & \frac{df}{dx} \rightleftharpoons ik \cdot \widehat{f}(k)
		\end{array}
		\]
	\end{block}
	\begin{theorem}[o mocy]
		\[
			\int\limits_{-\infty}^{\infty} f(x) \cdot g^*(x) dx = \int\limits_{-\infty}^{\infty} \widehat{f}(x) \cdot \widehat{g}^*(x)  \frac{dk}{2 \pi}
			\tag{16.8}
		\]
	\end{theorem}
	Analogicznie dla FS i fFT
\end{frame}
%%%%%%%%%%%%%%%%
\begin{frame}[allowframebreaks]{Splot (konwolucja)}
	$f(x), g(x) -$ funkcje
	\\ \quad ich splot:
	\begin{block}
	\centering
	\renewcommand{\arraystretch}{1.5}
	\setlength{\abovedisplayskip}{0pt}
	\setlength{\belowdisplayskip}{0pt}
	\setlength{\abovedisplayshortskip}{0pt}
	\setlength{\belowdisplayshortskip}{0pt}
	\[
		h(x) = \int\limits_{- \infty}^{\infty} dx' f(x')g(x-x')\equiv f \ast g
		\tag{16.9}
	\]
	\end{block}
	Własności:
	\begin{table}[t]
		\centering
		\begin{tabular}{|c|l|}
			\hline
			$f \ast g = g \ast f$ & commutative \\
			\hline
			$f \ast (g \ast h) = (f \ast g) \ast h$ & associative \\
			\hline
			$f \ast (g +h ) = f \ast g + f \ast h$ & distributive \\
			\hline
		\end{tabular} 
	\end{table}
	%%%%%%%%%%%%%%%%
	\begin{table}[t]
	\centering
	\captionsetup{belowskip=-10pt,aboveskip=-10pt}
	\caption{Splot i jego transformaty (16.10)}
		\begin{tabular}{l|l|l}
			& x & k \\
			\hline
				FT
				&
				\begin{tabular}{@{}l}
					$\int\limits_{-\infty}^{\infty} dx' \cdot f(x') \cdot  g(x-x')$ \\
					$f(x) \cdot g(x)$
				\end{tabular}
				&
				\begin{tabular}{@{}l}
					$\widehat{f}(k) \cdot \widehat{g}(k)$ \\
					$\int\limits_{-\infty}^{\infty} \frac{dk'}{2 \pi} \cdot \widehat{f}(k') \cdot \widehat{g}(k - k')$
				\end{tabular} \\
			\hline
				FS(i)
				&
				\begin{tabular}{@{}l}
					$\int\limits_{L} dx' \cdot f(x') \cdot g(x - x')$ \\
					$f(x) \cdot g(x)$
				\end{tabular}
				&
				\begin{tabular}{@{}l}
					$\widehat{f}(k) \cdot \widehat{g}(k)$ \\
					$\frac{1}{L} \cdot \sum\limits_{L' = -\infty}^{\infty} \widehat{f}(k') \cdot \widehat{g}(k - k')$
				\end{tabular} \\
			\hline
				FS(ii)
				&
				\begin{tabular}{@{}l}
					$H \cdot \sum\limits_{p' = -\infty}^{\infty}  f(x'_p) \cdot g(x_p - x'_p)$ \\
					$f(x_p) \cdot g(x_p)$
				\end{tabular}
				&
				\begin{tabular}{@{}l}
					$\widehat{f}(k) \cdot \widehat{g}(k)$ \\
					$\int\limits_{kg} \frac{dk'}{2 \pi} \cdot  \widehat{f}(k') \cdot \widehat{g}(k - k')$
				\end{tabular} \\
			\hline
				fFT
				&
				\begin{tabular}{@{}l}
					$H \cdot \sum\limits_{p' = 0}^{\infty} f(x'_p) \cdot  g(x_p - x'_p)$ \\
					$f(x_p) \cdot g(x_p)$
				\end{tabular}
				&
				\begin{tabular}{@{}l}
					$\widehat{f}(k) \cdot \widehat{g}(k)$ \\
					$\frac{1}{l} \cdot \sum\limits_{l' = 0}^{N - 1}  \widehat{f}(k') \cdot \widehat{g}(k - k')$
				\end{tabular}
		\end{tabular}
	\end{table}	
\end{frame}
%%%%%%%%%%%%%%%%
\begin{frame}{Transformaty 3-D i \dots}
	Uogólnienie z 1-D na 3-D (i więcej) - bezpośrednie:
	\begin{table}[t]
		\centering
		\renewcommand{\arraystretch}{1.35}
		\[
		\tag{16.11}
		\begin{array}{c|l}
			\text{1-D} & \text{3-D} \\
			\hline
			x & \vec{x} = (x_1, x_2, x_3) \\
			k & \vec{k} = (k_1, k_2, k_3) \\
			k\cdot x & \vec{k} \cdot \vec{x} \\
			dx & d\vec{x} \\
			\frac{d\vec{k}}{2\pi} & \frac{dk}{(2\pi)^3} \\	
			L & V_b = L_1 \cdot L_2 \cdot L_3 \\
			H & V_c = H_1 \cdot H_2 \cdot H_3 \\
			& \dots
		\end{array}
		\]
		\renewcommand{\arraystretch}{1}
	\end{table}
\end{frame}
	%%%%%%%%%%%%%%%%
	\section{Interpolacja trygonometryczna}
%%%%%%%%%%%%%%%%
\begin{frame}{Interpolacja trygonometryczna}
	\begin{itemize}
		\item f. okresowe: $g(y + L) = g(y)$; okres: L \\
		\item f. trygonometryczne: $L \to 2\pi, x = \frac{2\pi}{L} \cdot y, f(x) = g(\frac{x \cdot L}{2\pi})$	
	\end{itemize}
	szukamy wielomianu (trygonometrycznego)
	\[
		t_{n-1}(x) = \sum\limits_{j = 0}^{n-1} c_j \cdot e^{ijx}
		\tag{16.12}
	\]
	który w \underline{n punktach $x_k \in (0, 2\pi]$} przyjmuje te same wartości, co interpolowana funkcja
	\\ $t_{n - 1}(x_k) = f(x), k = 0, 1, \dots, n - 1$
\end{frame}
%%%%%%%%%%%%%%%%
\begin{frame}[allowframebreaks]{Dowód}
	\begin{theorem}
		Zadanie interpolacji trygonometrycznej ma jednoznaczne rozwiązanie.
	\end{theorem}
	\begin{proofs}
		\[
			\sum\limits_{j = 0}^{n-1} c_j \cdot \underbrace{(e^{ix_k})^j}_{Z_k} = f(x_k), k = 0, 1, \dots, n-1
			\tag{16.13}
		\]
		$\implies$ macierz układu: macierz Vandermonde'a jest nieosobliwa, \\ jeżeli \underline{$x_k$ są różne}, i:
		\[
			\det Z = \prod_{k \neq L} (Z_k - Z_L)
			\tag{16.14}
		\]
	\end{proofs}
	\begin{proofs}[\proofname\ (Cont.)]
		W praktyce - ważny przypadek szczególny: \\
		n węzłów równoodległych, \underline{$x_k = \frac{2\pi}{n}\cdot k$}, k = 0, 1, \dots, n - 1 \\
		\underline{funkcje $e^{ijx}$}, j = 0, 1, \dots, n - 1 tworzą \underline{układ ortogonalny} w sensie iloczynu skalarnego zdefinowanego: \\
		\[
			\langle f|g \rangle = \sum\limits_{k = 0}^{n-1} f(x_k) \cdot g^* (x_k), x_k = \frac{2\pi}{n} \cdot k, k = 0, 1, \dots, n-1
			\tag{16.15}
		\]
		\[
			\langle e^{ijx}|e^{ilx} \rangle = \sum\limits_{k = 0}^{n-1} e^{ijx_k}  \cdot e^{-ilx_k} = \sum\limits_{n = 0}^{n-1} e^{i(j - l) \frac{2\pi k}{n}}\stackrel{\Delta}{=}
			\tag{16.16}
		\]
	\end{proofs}
	\begin{proofs}[\proofname\ (Cont.)]
		\underline{dla $j = l$}
		\[
			\stackrel{\Delta}{=} \sum\limits_{k = 0}^{n-1} e^0 = n
			\tag{16.17}
		\]
		\underline{dla $j \neq l$}
		\[
			\stackrel{\Delta}{=} \sum\limits_{k = 0}^{n-1} \underbrace{e^{\frac{i(j - l)2\pi}{n}}}_{\substack{
				\text{-szereg geom.} \\
				\text{-n-wyrazów}
			}} \cdot k = \frac{a_0 - a_n \cdot q}{1 - q} = \frac{1 - e ^{\overbrace{\frac{i(j - l)2\pi}{n} \cdot n}^{=1}}}{1 - e^{\frac{i(j -  l)2\pi}{n}}} = 0
			\tag{16.18}
		\]
		$a_0 = 1; q = e^{\frac{i(j -  l)2\pi}{n}}$
	\end{proofs}
	\begin{proof}[\proofname\ (Cont.)]
		\[
			\langle e^{ijx}|e^{ilx} \rangle = \sum\limits_{k = 0}^{n - 1}e^{ijx_k}  \cdot e^{-ilx_k} = 
			\begin{cases}
				0 & j \neq l \\
				n & j = l
			\end{cases}
			\tag{16.19}
		\]
	\end{proof}
\end{frame}
%%%%%%%%%%%%%%%%
\begin{frame}[allowframebreaks]{Interpolacja trygonometryczna}
	Jakie powinny być współczynniki wielomianu interpolacyjnego $c_j$?
	\begin{enumerate}[1$^\circ$]
		\item
		\begin{align*}
			\langle t_{n-1}(x)|e^{ilx} \rangle = \Bigg\langle \sum\limits_{j = 0}^{n-1} c_j \cdot e^{ijx} \Bigg|e^{ilx} \Bigg\rangle = \sum\limits_{j = 0}^{n-1} c_j \cdot n \cdot \delta jl = \\ = c_l \cdot n \implies \underline{c_l = \frac{1}{n} \langle t_{n-1}(x)|e^{ilx} \rangle}
			\tag{16.20}
		\end{align*}
		\item
		\begin{align*}
			\langle t_{n-1}(x)|e^{ilx} \rangle \underbrace{=}_{(**)} \sum\limits_{k = 0}^{n-1}  \underbrace{t_{n-1}(x_k)}_{\substack{\text{=wartości} \\  \text{funkcji}}} \cdot e^{-ilx_k} = \\ = \sum\limits_{k = 0}^{n-1} f(x_k) \cdot e^{-ilx_k} = \underline{\langle f(x)|e^{ilx} \rangle}
		\end{align*}
	\end{enumerate}
	- def. iloczynu (przypadek dyskretny) \\
	czyli:
	\[
		\underline{c_j = \frac{1}{n} \sum\limits_{k = 0}^{n-1} f(x_k) \cdot e^{-ilx_k}, j = 0, 1, \dots, n-1}
		\tag{16.21}
	\]
	\begin{flushleft}
		$\implies$ \\
		\underline{\textit{analiza Fouriera:}} \\
		dla danych liczb zespolonych $f(x_k), k = 0, 1,\dots, n - 1$ szukamy $c_j, j = 0, 1, \dots, n-1$ \\
		\bigskip
		\underline{\textit{synteza Fouriera:}} \\
		mając liczby $c_j, j = 0, 1, \dots, n-1$ szukamy
		\[
			(k) = \sum\limits_{j = 0}^{n - 1} c_j \cdot e^{ij\frac{2\pi}{n} \cdot k}, k = 0, 1, \dots, n-1
			\tag{16.22}
		\]
	\end{flushleft}
	\begin{flushleft}
		$\implies$ obie:
		\begin{itemize}
			\item dyskretne
			\item wzajemnie odwrotne
		\end{itemize}
		\textcolor{blue}{Podsumowanie:}$\newline$
		klasyczny algorytm: $(f = A \cdot C)$
		\[ n^2
		\begin{cases}
			\text{-zespolonych mnożeń,} \\
			\text{-zespolonych dodawań,} \\
			\text{-obliczeń } e^{-i\frac{2\pi kj}{n}}
		\end{cases}
		\tag{16.23}
		\]
		$\newline$
		Wada $\rightarrow$ duża złożoność operacji
	\end{flushleft}	
\end{frame}
	%%%%%%%%%%%%%%%%
	\section{Szybka transformata Fouriera FFT dla $n=2^m$}
%%%%%%%%%%%%%%%%
\begin{frame}[allowframebreaks]{FFT}
	\begin{itemize}
		\item Danielson, Lanczos (1942)
		\item R.L.Garwin (IBM Yorktown Heights Researcg Center)
		\item J.W.Cooley, J.W.Turket (1962)
	\end{itemize}
	\begin{table}
		\centering
		\caption{Złożoność}
		\begin{tabular}{l|lll}
			& obliczeniowa && czasowa \\
			\hline
			klasyczna FT: & $O(N^2)$ & $\implies$ & 1.5 godziny \\
			FFT: & $O(N\log_2N)$ & $\implies$ & 0.1 sekundy 
		\end{tabular}
		\caption*{Założenia: \\
			rozmiar problemu: $N = 10^6$ \\
			CPU $\sim$ 100 MFLOPS}
	\end{table}
	\textbf{Dane:} $f(x_k), x_k = \frac{2\pi}{n} \cdot k, k = 0, 1, \dots, n-1$ \\
	\textbf{Szukamy:} $c_j = \frac{1}{n} \sum\limits_{k = 0}^{n-1} f(x_k) \cdot e^{-i\frac{2\pi k}{n} \cdot j}, j = 0, 1, \dots, n-1$ \\
	\textbf{gdy:} $a_k = \frac{1}{n} \cdot f(x_k), \omega = e^{-i\frac{2\pi}{n}}$ \\
	\textbf{to:} $\underline{c_j = \sum\limits_{k = 0}^{n-1} a_k \cdot \omega^{jk}}, j = 0, 1, \dots, n-1$ \\
	\textbf{Założenie:} ilość punktów: n = $2^m \implies$ tyleż współczynników Fouriera. \\
	\begin{block}{Istota algorytmu:}
	(k - numer punktu) \\
	gdy k parzyste $k = 2 \cdot k_1$ \\
	gdy k nieparzyste $k = 2 \cdot k_1 + 1$ \\
	Dziedzina k: \\
	z dołu $\underline{k_1 = 0}$ (parzyste k)  \\
	z góry: $n-1 = 2^m - 1 \implies$ k - nieparzyste: $2k_1 + 1 = n-1 \implies k_1 = \frac{n}{2} - 1$ \\
	\textbf{Rozdzielamy wyznaczanie współczyników!!!: }
	\[
		c_j = \sum\limits_{k_1=0}^{\frac{n}{2} - 1} a_2 \cdot k_1 \cdot (\omega^2)^{j \cdot k_1} + \sum\limits_{k_1 = 0}^{\frac{n}{2} - 1} a_2 \cdot k_1 + 1 \cdot (\omega^2)^{j \cdot k_1} \cdot \omega^j
		\tag{16.24}
	\]
	\blockbreak
	$\to$ \underline{niesymetria}: \\
	\[
	\begin{cases}
		0 \leq k_1 \leq \frac{n}{2} - 1 \\
		0 \leq j \leq n-1
	\end{cases}
	\tag{16.25}
	\]
	dla usunięcia niesymetrii: \\
	(j - numer współczynika $c_j$)
	\[
		j = (\frac{1}{2} \cdot n) \cdot l + j_1, 0 \leq j_1 \leq \frac{n}{2} - 1
	\]
	$\to$ \underline{jest to odpowiednik dzielenia j przez $\frac{n}{2} ; j_1$ - reszta}
	\blockbreak
	z kolei:
	\[
		\underline{(\omega^2)^{jk_1}} = \omega^{2 \cdot [\frac{n}{2} \cdot l + j_1] \cdot k_1} = \omega^{n \cdot l \cdot k_1 + 2j_1k_1} = (e^{-\frac{2\pi i}{n}})^{(n \cdot l \cdot k_1 + 2j_1k_1)} = \underline{(\omega^2)^{j_1k_1}}
		\tag{16.26}
	\]
	bo: $\underline{\omega^{nlk_1}} = 1$ \\
	\underline{i uzyskujemy}:
	\[
		c_j = \underbrace{\sum\limits_{k_1 = 0}^{\frac{n}{2} - 1} a_{2 \cdot k_1} \cdot (\omega^2)^{j_1k_1}}_{\varphi(j_1)} + \underbrace{ \sum\limits_{k_1 = 0}^{\frac{n}{2}-1} a_{2 \cdot k_1} + 1 \cdot (\omega^2)^{j_1k_1}}_{\psi(j_1)} \cdot \omega^j, 0\leq j_1 \leq \frac{n}{2} - 1
		\tag{16.27}
	\]
	\blockbreak
	Każdy z 2 członów jest transformatą Fouriera - zamiast pojedyńczej transformaty w n punktach $\to$ suma 2 transformat w $\frac{n}{2}$ punktach:
	\[
		c_j = \varphi(j_1) + \omega^j \cdot \psi(j_1) ; j_1 = 0, 1, \dots, \frac{n}{2} - 1
		\tag{16.28}
	\]
	itd $\dots \to$ dokonując dalszych podziałów. \\
	złożoność obliczeniowa $\leq \underline{2 \cdot Nlog_2N}$ \\
	\end{block}
	\begin{center}
	\textbf{Zasada dziel i zwyciężaj!}
	\end{center}
	Zadanie: Zrozumieć procedurę realizującą FFT dla N = 8 węzłów $\newline$
\end{frame}  

\begin{frame}[fragile]{Rekurencyjny algorytm FFT}
\textbf{Zadanie:} Zapoznaj się z iteracyjną implementacją FFT
\begin{lstlisting}[language=Mathematica, mathescape]
function FFT(a)
	$n \leftarrow length[a]$
	if n = 1 
		then return a
	$\omega_n \leftarrow e^{\frac{2\pi\cdot i }{n}}$
	$\omega \leftarrow 1$
	$a_{even} \leftarrow (a_0, a_2, \dots, a_{n-2})$
	$a_{odd} \leftarrow (a_1, a_3, \dots, a_{n-1})$
	$y^{even} \leftarrow FFT(a_{even})$
	$y^{odd} \leftarrow FFT(a_{odd})$
	for j $\leftarrow$ 0 to $\frac{n}{2} -1$
		$y_j \leftarrow y^{even}_{j} + \omega y^{odd}_{j}$
		$y_{j+\frac{n}{2}} \leftarrow y^{even}_{j} - \omega y^{odd}_{j} $
		$\omega \leftarrow \omega \cdot \omega_n$
	end	
	return y
end
\end{lstlisting}

\end{frame}
	%%%%%%%%%%%%%%%%
	\section{FFT dla przypadku ogólnego}
%%%%%%%%%%%%%%%%
\begin{frame}{FFT dla przypadku ogólnego I}
	\[
		c_j = \frac{1}{n} \sum\limits_{k = 0}^{n-1} f(x_k)e^{-ijx_k}, x_k = \frac{2\pi}{n} \cdot k;\ j = 0, 1, \dots, n-1
	\]
	$\omega = e^{-i\frac{2\pi}{n}}, a_k = \frac{1}{n} f(x_k)$
	\[
		\underline{c_j = \sum\limits_{k = 0}^{n-1} a_k \cdot \omega^{jk}}, j = 0, 1, \dots, n-1
	\]
	dla $n = r_1 \cdot r_2 \cdot \dots \cdot r_p, r_i$ liczby całkowite
	\[
		c_j = \sum\limits_{k = 0}^{n-1} a_k \cdot e^{-i2\pi \overbrace{j \cdot \frac{k}{n}}^{(*)}}
		\tag{16.29}
	\]
	(*) wyrazić przez $r_1, r_2, \dots, r_p$
\end{frame}
%%%%%%%%%%%%%%%%
\begin{frame}{FFT dla przypadku ogólnego II}
	\underline{każda liczba całkowita j, $0 \leq j \leq n-1$ ma jedyną reprezentację}:
	\[
		j = \alpha_1 \cdot (r_2 \cdot r_3 \cdots r_p) + \alpha_2 \cdot (r_3 \cdot r_4 \cdots r_p) + \dots + \alpha_{p-1} \cdot r_p + \alpha_p \cdot 1
	\]
	co, dla uproszczenia, można zapisać:
	\[
		j = \sum\limits_{i = 1}^{p} \alpha_i \cdot n_i
	\]
	gdzie
	\[
		n_r = r_{r+1} \cdot r_{r+2} \cdots r_p = \prod\limits_{i = (r+1)}^{p} r_i
	\]
	$n_p = 1, n_0 = n$ \\
	zaś współczynniki:
	\[
		\alpha_i \in \{0, 1, \dots, r_i - 1\}
	\]
\end{frame}
%%%%%%%%%%%%%%%%
\begin{frame}{FFT dla przypadku ogólnego III}
	Przydatne będą też:
	\[
		j_r = \sum\limits_{i = (r+1)}^{p} \alpha_i \cdot n_i,\  j_0 = j
	\]
	przy czym:
	\[
		\begin{rcases}
			\alpha_i - \text{iloraz} \\
			j_i - \text{reszta}
		\end{rcases}
		\text{z dzielenia } \frac{j_{i-1}}{n_i}
	\]
\end{frame}
%%%%%%%%%%%%%%%%
\begin{frame}{FFT dla przypadku ogólnego IV}
	\underline{Dla każdej liczby całkowitej k, $0 \leq k \leq n-1$} \\
	\underline{istnieje jedyna reprezentacja ilorazu $\frac{k}{n}$}:
	\[
		\frac{k}{n} = \frac{l_1}{\underbrace{r_1 \cdot r_2 \cdots r_p}_{n_0}} + \frac{l_2}{\underbrace{r_2 \cdot r_3 \cdots r_p}_{n_1}} + \cdots + \frac{l_{i+1}}{\underbrace{r_{i+1} \cdots r_p}_{n_i}} + \cdots + \frac{l_p}{r_p} = \underline{\sum\limits_{i = 0}^{p-1} \frac{l_{i+1}}{n_i}}
	\]
	\[
		l_i \in \{0, 1, \dots, r_i - 1\}
	\]
	\[
		k_r = n_r \cdot \sum\limits_{i = r}^{p-1} \frac{l_{i+1}}{n_i}, k_0 = k
	\]
	\[
		\begin{rcases}
			k_i - \text{iloraz} \\
			l_i - \text{reszta}
		\end{rcases}
		\text{z dzielenia } \frac{k_{i-1}}{r_i}
	\]
	\[
		j \cdot \frac{k}{n} = j_0 \cdot \frac{k_0}{n_0} = \Bigg( \sum\limits_{i = 1}^{p} \alpha_i \cdot n_i \Bigg) \cdot \Bigg( \sum\limits_{r = 0}^{p-1} \frac{l_{r+1}}{n_r} \Bigg) = \sum\limits_{r = 0}^{p-1} \sum\limits_{i = 1}^{p} l_{r+1} \cdot \alpha_i \cdot \frac{n_i}{n_r} \stackrel{\Delta}{=}	
	\]
\end{frame}
%%%%%%%%%%%%%%%%
\begin{frame}{FFT dla przypadku ogólnego V}
	\[
		\underline{\overline{\frac{n_i}{n_r} = 
		\begin{cases}
			\text{l. całkowita}, & i \leq r \\
			\text{ułamek}, & i > r
		\end{cases}}}
	\]
	\[
		\stackrel{\Delta}{=} M + \sum\limits_{r = 0}^{p-1} \frac{l_{r+1}}{n_r} \cdot \Bigg( \underbrace{\sum\limits_{i = (r+1)}^{p} \alpha_i \cdot n_i}_{j_r} \Bigg) = M +  \sum\limits_{r = 0}^{p-1} \frac{l_{r+1} \cdot j_r}{n_r}
	\]
	\[
		\underline{\omega^{jk}} = e^{-i2\pi \bigg( M +  \sum\limits_{r = 0}^{p-1} \frac{l_{r+1} \cdot j_r}{n_r} \bigg)} = e^{-i2\pi \sum\limits_{r = 0}^{p-1} \frac{l_{r+1} \cdot j_r}{n_r}} =
	\]
	\[
		= \prod\limits_{r = 0}^{p-1} {\underbrace{\bigg( e^{-\frac{i2\pi}{n_r}} \bigg)}_{\omega_r}}^{l_{r+1} \cdot j_r} = \underline{\prod\limits_{r = 0}^{p-1} \omega_r^{l_{r+1} \cdot j_r}}
	\]
\end{frame}
%%%%%%%%%%%%%%%%
\begin{frame}{FFT dla przypadku ogólnego: przykład}
	\begin{exampleblock}{Przykład}
	$p = 3, n = r_1 \cdot r_2 \cdot r_3$ \\
	$n_0 = r_1r_2r_3 ; n_1 = r_2r_3 ; n_2 = r_3 ; n_3 = 1$ \\
	$j = \alpha_1 \cdot (r_2 \cdot r_3) + \alpha_2 \cdot r_3 + \alpha_3 ; j_0 = j ; j_1 = \alpha_2  \cdot r_3 + \alpha_3 ; j_2 = \alpha_3$ \\
	w miejsce $k \to k(l_1, l_2, l_3)$ zgodnie z $\frac{k}{n} = \sum\limits_{i = 0}^{p-1} \frac{l_{i+1}}{n_i}$ \\
	oznaczamy:
	\[
		c^{(0)}(l_1, l_2, l_3) = a_k
	\]
	\[
		c_j = \sum\limits_{k = 0}^{n-1} a_k \cdot \omega^{jk} = \sum\limits_{l_1 = 0}^{r_1 - 1} \sum\limits_{l_2 = 0}^{r_2 - 1} \sum\limits_{l_3 = 0}^{lr_3 - 1} c^{(0)}(l_1, l_2, l_3) \cdot \omega^{jl_1} \omega_1^{j_1l_2} \omega_2^{j_2l_3}
	\]
	\end{exampleblock}
\end{frame}
%%%%%%%%%%%%%%%%
\begin{frame}{FFT dla przypadku ogólnego: przykład cd}
	\begin{exampleblock}{Przykład cd}
	wyliczamy sumy kolejno, poczynając od wewnętrznej:
	\[
		\sum\limits_{l_3 = 0}^{r_3 - 1} c^{(0)}(l_1, l_2, l_3) \omega^{j_2l_3} = c^{(1)}(l_1, l_2,\alpha_3),
	\]
	\[
		l_1, \stackrel{\uparrow\nearrow}{l_2} - \text{ustalone} ; j_2 = j_2(\alpha_3) - \text{tylko}
	\]
	\[
		\sum\limits_{l_2 = 0}^{r_2 - 1} c^{(1)}(l_1, l_2, \alpha_3)\omega_1^{j_1l_2} = c^{(2)}(l_1,\alpha_2,\alpha_3),
	\] 
	\[
		\sum\limits_{l_1 = 0}^{r_1 - 1} c^{(2)}(l_1, \alpha_2, \alpha_3) \omega^{jl_1} = c_j.
	\]
	(przyp. ogólny $\to$ szczególny)
	\end{exampleblock}
\end{frame}
	%%%%%%%%%%%%%%%%
	\section{Transformata Hartley'a}
%%%%%%%%%%%%%%%%
\begin{frame}[allowframebreaks]{Transformata Hartley'a}
	\begin{block}{Tranformata Fouriera}
	\[
		F(f) = \int\limits_{-\infty}^{\infty} X(t) e^{-i2\pi ft} dt
	\]
	\[
		X(t) = \int\limits_{-\infty}^{\infty} F(f) e^{i2\pi ft} df
	\]
	\[
		c_j = \frac{1}{n} \sum\limits_{k = 0}^{n-1} X(t_k) e^{-i2\pi j \frac{k}{n}}
	\]
	\[
		X(t_k) = \sum\limits_{k = 0}^{n-1} c_j e^{i2\pi j \frac{k}{n}}
	\]
	\end{block}
	\begin{block}{Transformata Hartley'a}
	\[
		H(f) = \int\limits_{-\infty}^{\infty} X(t) \cas(2\pi ft) dt
	\]
	\[
	X(t) = \int\limits_{-\infty}^{\infty} H(f) \cas(2\pi ft) dt
	\]
	\end{block}
	gdzie: $\cas(x) = \cos(x) + \sin(x)$
	\begin{block}{Wersja dyskretna HT}
	\[
		H_j = \frac{1}{n} \sum\limits_{k = 0}^{n-1} f(t_k) \cdot \cas \bigg( \frac{2\pi jk}{n} \bigg)
	\]
	\[
		f(t_k) = \sum\limits_{j=0}^{n-1} H_j \cdot \cas \bigg( \frac{2\pi jk}{n} \bigg)
	\]
	\end{block}
\end{frame}
\begin{frame}{Własności HT}
	\begin{block}{16.30}
		\centering
		\begin{enumerate}[1$^\circ$]
			\item $F_r(j) = H(j) + H(n-j)$ \\ $F_i(j) = H(j) + H(n-j)$ \\
			\item power spectrum: $P_s(j) = [H^2(j) + H^2(n-j)] \cdot \frac{1}{2}$ \\
			\item $f_1(t) \ast f_2(t) = \int\limits_{-\infty}^{\infty} f_1(\tau) \cdot f_2(t - \tau) d\tau$ \hfill - splot \\
			$f_1(t) \ast f_2(t) = F_1(f) \cdot F_2(f)$ \\
			$f_1(t) \ast f_2(t) = H_1(f) \cdot H_{2e}(f) + H_1(-f) \cdot H_{2o}(f)$
		\end{enumerate}
	\end{block}
	dla oznaczeń:
	\begin{itemize}
		\item r - real, i - imaginary \\
		\item o - odd, e - even \\
		\item F, f - Fourier, H - Hartley
	\end{itemize}
\end{frame}
\begin{frame}{Szybka Transformacja Hartley'a}
	\begin{block}{Fourier}
	\[
		F_j = F_{1j} + F_{2j} \cdot e^{-i\frac{2\pi j}{n'}} , n' = \frac{n}{2}
	\]
	\end{block}
	\begin{block}{Hartley}
	\[
		H_j = H_{1j} + H_{2j} \cdot \cos \bigg( \frac{2\pi j}{n'} \bigg) + H_2(n'-1) \cdot \sin \bigg( \frac{2\pi j}{n'} \bigg)
	\]
	\end{block}
\end{frame}
\begin{frame}{Biliografia}
	\begin{itemize}
		\item R.V.L. Hartley: A more symetrical Fourier analysis applied to transmission problems, Proc. IRE, 30 (1942) 144, \\
		\item R.N. Bracewell: The fast Hartley transform, Proc. IEEE 72 (1984) 1010 (No 8), \\
		\item M.A. O'Neill: Faster than fast Fourier, Byte, April 1988, p.293. istotna różnica: zamiast $\underbrace{e^{-x \cdot i}}_{\text{zespolone}}$ mamy $\underbrace{\cas(x)}_{\text{rzeczywiste}}$ ($\implies$ ilość operacji arytmetycznych i pamięc)
	\end{itemize}
\end{frame}
\begin{frame}{FFT - przydatna w:}
	\begin{enumerate}
		\item analiza spektralna \\
		\item projektownie efektywnch algorytmów
		\begin{itemize}
			\item iloczyn wielomianów $\to$ splot 2 wektorów \\
			\item szybki binarny algorytm mnożenia liczb całkowitych
			(m. Sch\"{o}nhagego - Strassena)
		\end{itemize}
		\item $\to$ A.V. Aho, J.E. Hopcroft, J.D. Ullman: \\
		Projektowanie i analiza algorytmów komputerowych. PWN, 1983 (1974)
	\end{enumerate}
\end{frame}
	%%%%%%%%%%%%%%%%
	\section{Transformata Fouriera w internecie}
%%%%%%%%%%%%%%%%
\begin{frame}{Transformata Fouriera w internecie}
	\begin{itemize}
		\item FFT dla przykładowych funcji$^1$ (autor: Robert Wacięga) \\
		\item transformata Fourier'a w ujęciu Matlab'a$^2$ (autor: The Math Works, Inc) \\
		\item zastosowania transformat Fourier'a$^3$ (autor: University of Strathclyde) \\
		\item NAG i FFTPACK$^4$ (autor:)
	\end{itemize}
\end{frame}
	%%%%%%%%%%%%%%%%
\end{document}
