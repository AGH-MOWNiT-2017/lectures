\section{Metoda podstawowa}
%%%%%%%%%%%%%%%%
\begin{frame}{Metoda podstawowa}
	\[
    	I = \int_a^b f(x) dx = \int_a^b h(x)g(x) dx = E\{Y\}
    \]
    $g(x)$ - gęstość pewnej zmiennej losowej $X \in (a, b)$
    
    Np.:
    \begin{itemize}
    	\item $Y = h(x)$
        \item $I = (b-a) \int_a^b f(x) \underbrace{\frac{dx}{b-a}}_\text{dP}$, $P$ - rozkład równomierny na $(a, b)$
    \end{itemize}
    
\end{frame}
%%%%%%%%%%%%%%%%
\begin{frame}{Metoda podstawowa}
    Bez utraty ogólności wystarczy rozpatrywać: \[
    	I = \int_0^1 f(x) dx = E\{Y\}
    \]
    $Y = f(X)$, przy czym $X$ - zmienna losowa o rozkładzie równomiernym $(0, 1)$
    
    Oszacowanie $E\{Y\}$ - średnia z $N$ niezależnych obserwacji: \[
    	\boxed{
        \hat{I} = \frac{1}{N} \sum_{i=1}^n f(x_i) \hspace{.5cm} (**)
        }
    \]
\end{frame}
%%%%%%%%%%%%%%%%
\begin{frame}{Metoda podstawowa - procedura}
    \begin{block}{Procedura}
        \begin{enumerate}[1)]
            \item wylosować $x_1, x_2, ..., x_N$ wg rozkładu równomiernego na $(0, 1)$
            \item obliczyć $f(x_1), f(x_2), ..., f(x_N)$
            \item średnia $(**)$
        \end{enumerate}
    \end{block}
\end{frame}
%%%%%%%%%%%%%%%%
\begin{frame}{Metoda podstawowa - wariancja}
	\vspace{-.5cm}
	\begin{block}{Wariancja}
		\begin{align*}
			\sigma^2(\hat{I}) = 
            \sigma^2 \frac{1}{N} \left[ 
            	\sum_{i=1}^n f(x_i)
            	\right] =
            \frac{1}{N^2} \sum_{i=1}^N \sigma^2 [f(x_i)] =
            \\
            \frac{1}{N} \sigma^2 [f(x)] = 
            \frac{1}{N} \left[ 
            	\int_0^1 f^2(x)dx - I^2
            	\right]
		\end{align*}
	\end{block}
    
    \begin{block}{Estymator wariancji}
    	\[
        	\sigma^2(\hat{I}) = \frac{1}{N} \sigma_f^2; 
            \sigma_f^2 = \frac{1}{N-1} \sum_{i=1}^N [f(x_i) - \hat{I}]^2
        \]
    \end{block}
    
    $\sigma^2(\hat{I}_2) - \sigma^2(\hat{I}_2) > 0$ \hspace{.5cm}
    1 - orzeł/reszka; \hspace{.5cm}
    2 - podstawowa
\end{frame}