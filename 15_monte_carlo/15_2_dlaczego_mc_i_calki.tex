\section{Dlaczego MC i całki?}
%%%%%%%%%%%%%%%%
\begin{frame}{Dlaczego MC i całki?}
	\begin{enumerate}
		\item[a)]
        	$X$ - zmienna losowa ciągła, wartości z $(a, b)$ zgodnie z rozkładem $f(x)$ \\
            $g$ - ustalona funkcja \\
            $y = g(x)$ - zmienna losowa
            
            {\bf Wart. oczekiwana} $Y: E\{Y\} = E\{g(x)\} = \int_a^b g(x) f(x) dx$ \hfill $(*)$
	\end{enumerate}
\end{frame}
%%%%%%%%%%%%%%%%
\begin{frame}{Dlaczego MC i całki?}
	\begin{enumerate}
		\item[b)]
			chcemy obliczyć całkę: $I = \int_a^b h(x) dx$ \\
            niech $f: f(x) > 0$ dla $x \in (a, b), \int_a^b f(x) dx = 1$ \\
            $f(x)$ - gęstość rozkładu pewnej zmiennej losowej przyjmującej wartości z $(a, b)$ \\
            $I = \int_a^b \frac{h(x)}{f(x)} f(x) dx = \int_a^b g(x)f(x)dx$ \hfill $\leftarrow$ całka postaci $(*)$ \\
            Obliczanie całki można zawsze przedstawić jako zagadnienie obliczania wartości oczekiwanej pewnej zmiennej losowej ciągłej.
            \\[8pt]
            (suma szeregu - zmienna losowa dyskretna)
	\end{enumerate}
\end{frame}