\section{Direct factorization (LU decomposition)}
\begin{frame}{Wprowadzenie}
Zaleta: mając $A=L\cdot U\rightarrow $ dla dowolnego $b$
$$
A\cdot x=b\ \ \ L\cdot(U\cdot \mathrm{x})=b
$$
$$
U\cdot x=z\ (**)
$$
$$
L\cdot z=b\ (*)
$$
\end{frame}
\begin{frame}{Wprowadzenie}
\scalebox{0.85}{
$\left(\begin{array}{llll}
a_{11} & a_{12} & \cdots & a_{1n}\\
a_{21} & a_{22} & \cdots & a_{2n}\\
\vdots &  &  & \\
a_{n1} & a_{n2} & \cdots & a_{nn}
\end{array}\right)= \left(\begin{array}{lllll}
l_{11} & 0 & \cdots & & 0\\
l_{21} & l_{22} & 0 &\cdots & 0\\
\vdots &  & \ddots & \\
l_{n1} & l_{n2} & \cdots & & l_{nn}
\end{array}\right) \cdot \left(\begin{array}{llll}
u_{11} & u_{12} & \cdots & u_{1n}\\
0 & u_{22} & \cdots & u_{2n}\\
\vdots &  & \ddots & \\
0 & \cdots & 0 & u_{nn}
\end{array}\right)$
}
\begin{center}
$(*)z_{1}=\displaystyle \frac{b_{1}}{l_{11}};z_{i}=\frac{1}{l_{ii}}\cdot(b_{i}-\sum_{j=1}^{i-1}l_{ij}\cdot z_{j})$ , $i=2$, 3, . . . , $n$
\end{center}
$(**)x_{n}=\displaystyle \frac{z_{n}}{u_{nn}};x_{i}=\frac{1}{u_{ii}}\cdot(z_{i}-\sum_{j=i+1}^{n}u_{ij}\cdot x_{j})$ , $i=i-1, i-2$, . . . , 1

\# niewiadomych: $2 \cdot \left( \frac{n^{2}-n}{2}+n\right)=n^{2}+n$ \newline
\# r\'{o}wna\'{n}: $a_{ij}=\displaystyle \sum_{k=1}^{n}l_{ik}\cdot u_{kj}\rightarrow n^{2}$
\end{frame}
\begin{frame}{Wprowadzenie}
Macierze :

$\bullet$ Doolitle: $l_{ii}=1$

$\bullet$ Crout: $u_{ii}=1$

$\bullet$ Choleski: $l_{ii}=u_{ii}$

Ad.Choleski-macierz symetryczna, dodatkowo określona $A=(LD^{\frac{1}{2}})\cdot(D^{\frac{1}{2}}L^{T})$ \newline
Sposób: na przemian-wiersze, kolumny $\mathrm{w}A.$

Po obliczeniu-L,U zapisane $\mathrm{w}A.$
\end{frame}
\begin{frame}{Faktoryzacja Doolitle'a}
LU inaczej-lepszy wgląd kolejność obliczeń
\scalebox{0.90}{
$\begin{bmatrix}
(1) & 0 & \cdots & 0 & 0\\
l_{21} & (1) &  &  & \\
\vdots &  &  &  & \\
l_{i1} & l_{i2} & \cdots & (1) & 0\\
\vdots &  &  &  & \\
l_{n1} & l_{n2} & \cdots & l_{n,n-1} & (1)
\end{bmatrix} \cdot \begin{bmatrix}
u_{11} & u_{12} & \cdots & u_{1n}\\
0 & u_{22} & \cdots & u_{2n}\\
 &  &  & \vdots\\
 &  & u_{ii} & u_{in}\\
 &  &  & \vdots\\
 0 & \cdots & \cdots & u_{nn}
\end{bmatrix}=\begin{bmatrix}
a_{11} & a_{12} & \cdots & a_{1n}\\
a_{21} & a_{22} & \cdots & a_{2n}\\
 &  &  & \vdots\\
 a_{i1} & a_{i2} & \cdots & a_{in}\\
 &  &  & \vdots\\
 a_{n1} & a_{n2} & \cdots & a_{nn}
\end{bmatrix}$

}
\newline
$W \mathrm{i}$-tym wierszu macierzy $\mathrm{L}$ różne od $0$ tylko pierwsze $\mathrm{i}$ element\'{o}w $W\mathrm{j}$-tej kolumnie macierzy $\mathrm{U}$ r\'{o}zne od $0$ tylko pierwsze $\mathrm{i}$ elementów

\end{frame}
\begin{frame}{Faktoryzacja Doolitle'a}

\begin{center}
$a_{ij}= \displaystyle \sum_{p=1}^{\min(i,j)} l_{ip}\cdot u_{pj}, \ \  i, \ \  j=1$, 2, . . . , $n$

\hspace{55mm}
$\mathrm{i}>\mathrm{j}$


$a_{ij}=\displaystyle \sum_{p=1}^{j-1}l_{ip}\cdot u_{pj}+l_{ij}\cdot u_{jj}, i=1$, 2, . . . , $n$
\end{center}
$l_{ij}=\displaystyle \frac{1}{u_{jj}}(a_{ij}-\sum_{p=1}^{j-1}l_{ip}\cdot u_{pj})\mathrm{i}\leq \mathrm{j}a_{ij}=\sum_{p=1}^{i}l_{ip}\cdot u_{pj}, \ \ i=1$, 2, . . . , $n$
$$
u_{ij}=a_{ij}-\sum_{p=1}^{i-1}l_{ip}\cdot u_{pj}
$$

(Z tym, że $\sum_{p=q}^{m}=0$ gdy $m < g$, np. $u_{1j}=a_{1j}$ dla i=j) \newline
$L_{ij}, U_{ij}$ przechowywane w $\mathrm{L}/\mathrm{U}$ (bez "1" $\mathrm{z}$ diagonali)
\end{frame}
\begin{frame}{Faktoryzacja Doolitle'a}
\begin{itemize}
\item \textbf{ a) Row- wise order } w macierzy $L/U$
w kroku $\mathrm{k}$ wyznaczamy $L_{k1}, L_{k2}$, . . . , $L_{k,k-1}, U_{k,k}$, . . . , $U_{k,n}$ porównać z metodą eliminacji Gaussa: partial sums are stored and retrieved
\item \textbf{b) Column-wise order}  w macierzy $L/U$

$\mathrm{w}$ kroku $\mathrm{k}$ wyznaczamy $U_{1k}, U_{2k}$, . . . , $U_{k,k}, L_{k+1,k}$, . . . , $L_{n,k}$ porównać z metodą eliminacji Gaussa
\end{itemize}


- eliminacja Gaussa-wielokrotne pobieranie, powiększanie \newline - Doolittle-akumulacja do pamięci

\begin{flushright}
{\it Zadanie}: Zbadać złożoność, stworzyć algorytm
\end{flushright} 

\end{frame}
\begin{frame}{ Faktoryzacja Crout'a}
\scalebox{0.90}{
$
\begin{bmatrix}
l_{11} & 0 & \cdots & 0 &  0 \\
l_{21} & l_{22} \\
\vdots \\
l_{i1} & l_{i2}  & \cdots\\
\vdots \\
l_{n1} & l_{n2} & \cdots & l_{n,n-1} & l_{nn}
\end{bmatrix} \cdot
\begin{bmatrix}
1 & u_{12} & \cdots & u_{1n}\\
0 & 1 & \cdots & u_{2n}\\
 &  &  & \vdots\\
 &  & 1 & u_{in}\\
 &  &  & \vdots\\
 0 & \cdots & \cdots & 1
\end{bmatrix}=\begin{bmatrix}
a_{11} & a_{12} & a_{1n}\\
a_{21} & a_{22} & a_{2n}\\
 &  & \vdots\\
 a_{i1} & a_{i2} & a_{in}\\
 &  & \vdots\\
 a_{n1} & a_{n2} & a_{nn}
\end{bmatrix}
$
}

\end{frame}
\begin{frame}{ Faktoryzacja Crout'a}
\begin{flushright}
$a_{ij}= \displaystyle \sum_{p=1}^{\min(i,j)} l_{ip}\cdot 
u_{pj}, i, j=1$, 2, . . . , $n$
\end{flushright}
\begin{flushright}
$
\mathrm{i}\geq \mathrm{j}
$
\end{flushright}
\begin{flushright}
$a_{ij}=\displaystyle \sum_{p=1}^{j-1}l_{ip}\cdot u_{pj}+l_{ij}.\ \  i=1, 2, . . . , n$
\end{flushright}
\begin{flushright}
$l_{ij}=(a_{ij}-\sum_{p=1}^{j-1}l_{ip}\cdot u_{pj})$
\end{flushright}
\begin{flushright}
$
\mathrm{i}<\mathrm{j}
$
\end{flushright}

\begin{flushright}
$a_{ij}= \sum_{p=1}^{i-1}l_{ip}\cdot u_{pj}+l_{ii}\cdot u_{ij}, i=1$, 2, . . . , $n$
\end{flushright}
\begin{flushright}
$u_{ij}=\frac{1}{l_{ii}}(a_{ij}-\sum_{p=1}^{i-1}l_{ip}\cdot u_{pj})$
\end{flushright}
\end{frame}
\begin{frame}{ Faktoryzacja Crout'a}
Kolejność obliczeń w algorytmie Crout'a:

$$
\begin{bmatrix}
1 & 5 & 9 & 13\\
2 & 6 & 10 & 14\\
3 & 7 & 11 & 15\\
4 & 8 & 12 & 16
\end{bmatrix}
\begin{array}{lllll}
| & - & - &- & 2 \\
| & | & - & - & 4 \\
| & | & | & - & 6 \\
| & | & | \\
1 & 3 & 5 
\end{array}
$$
\begin{flushright}

{\it Zadanie}: Zbadać złożoność, stworzyć algorytm
\end{flushright}
Związek z eliminacją Gaussa:
$$
a_{ij}^{(k+1)}\ =\ \underbrace{a_{ij}^{(k)}} -a_{ik}^{(k)}\ \cdot \underbrace{\frac{a_{kj^{(k)}}}{a_{kk}^{(k)}}}
$$
$ \hspace{43mm} =\ \ \  l_{ik}\ \ \ \ \ \ \ \ \ \ \ \ \ u_{kj}
$
\newline \hspace*{30mm} różnica $\rightarrow$ kumulacja w 1 kroku

\end{frame}
\begin{frame}{ Faktoryzacja Crout'a}
\begin{exampleblock}{Ogólnie faktoryzacja do postaci $A=LDU$}
D - diagonalna \hspace{40mm } {\it Doolittle}: \hspace{10mm} $L(DU)$ \newline
\hspace*{55mm} $\rightarrow$ 
\newline
 L, U-z ''$1$'' na głównej diagonali \hspace{15mm} {\it Crout}: \hspace{11mm}  $LD(U)$
\newline 
\end{exampleblock}
\end{frame}
\begin{frame}{Faktoryzacja Choleski'ego}
(symetric matrices) $a_{ij}=a_{ji}$

pivot z diagonali, $A^{(k)}, \mathrm{k}=1,2,\ldots,\mathrm{n}$ -symetryczna, czyli
$$
a_{ij}^{(k)}=a_{ji}^{(k)},\ i\geq j,\ j\geq k
$$
$\mathrm{s}\mathrm{t}\omega$ dla faktoryzacji $A=LDU$
$$
l_{ij}=u_{ij},\ i\geq j
$$
czyli $L=UT$

Jeżeli $A$ jest dodatnio określona, to $d_{ii}>0, \mathrm{i}=1,\ldots,\mathrm{n}$
\begin{flushright}
{\it Zadanie}: Udowodnić powyższe 
\end{flushright}

\end{frame}
\begin{frame}{Faktoryzacja Choleski'ego}
\begin{exampleblock}{}
$$
A=(LD^{\frac{1}{2}})(D^{\frac{1}{2}}L^{T})=\overline{L}\ \overline{L}^{T}
$$
\end{exampleblock}
i ta faktoryzacja nosi nazwę Choleskiego
\begin{flushright}
{\it Zadanie}: Wyznaczyć złożoność i stworzyć algorytm
\end{flushright}
\end{frame}
\begin{frame}{Computational sequences}
\begin{exampleblock}{Factorization method:}
Ability to compute each entry in $L,U$ at one time, accumulating the sum repeated storage and retrieval of 
\end{exampleblock}
 

\end{frame}
\begin{frame}{Computational sequences}
$+$ -entries used (dane użyte)

$*$ -entries changed (dane zmienione)partial sum is avoided

\textbf{ a) Gaussian elimination}
$$
U_{11}\ U_{12}\ U_{13}\ U_{14}\ U_{15}
$$
$$
L_{21}\ U_{22}\ U_{23}\ U_{24}\ U_{25}
$$
$$
L_{31}\ L_{32}\ |U_{33}^{+}\ |U_{34}^{+}\ |U_{35}^{+}
$$
$$
L_{41}\ L_{42}\ a_{43}^{(3)*}\ a_{44}^{(3)*}\ a_{45}^{(3)*}
$$
$$
L_{51}\ L_{52}\ a_{53}^{(3)*}\ a_{54}^{(3)*}\ a_{55}^{(3)*}
$$

\end{frame}
\begin{frame}{Computational sequences}
$+$ -entries used (dane użyte)

$*$ -entries changed (dane zmienione)partial sum is avoided


\textbf{a) row Doolittle LU decomposition}
$$
|U_{11}^{+}\ |U_{12}^{+}\ |U_{13}^{+}\ |U_{14}^{+}\ |U_{15}^{+}
$$
$$
|L_{21}^{+}\ |U_{22}^{+}\ |U_{23}^{+}\ |U_{24}^{+}\ |U_{25}^{+} 
$$
$$
a_{31}^{*}\ a_{32}^{*}\ a_{33}^{*}\ a_{34}^{*}\ a_{35}^{*}
$$
$$
a_{41}\ a_{42}\ a_{43}\ a_{44}\ a_{45} 
$$
$$
a_{51}\ a_{52}\ a_{53}\  a_{54}\ a_{55}
$$

\end{frame}
\begin{frame}{Computational sequences}
$+$ -entries used (dane użyte)

$*$ -entries changed (dane zmienione)partial sum is avoided


\textbf{a) Crout LU decomposition}
$$
L_{11}\ U_{12}\ |U_{13}^{+}\ |U_{14}^{+}\ |U_{15}^{+} 
$$
$$
L_{21}\ L_{22}\ |U_{23}^{+}\ |U_{24}^{+}\ |U_{25}^{+} 
$$
$$
|L_{31}^{+}\ |L_{32}^{+}\ a_{33}^{*}\ a_{34}^{*}\ a_{35}^{*}
$$
$$
|L_{41}^{+}\ |L_{42}^{+}\ a_{43}^{*}\ a_{44}\ a_{45}
$$
$$
|L_{51}^{+}\ |L_{52}^{+}\ a_{53}^{*}\ a_{54}\ a_{55}
$$

\end{frame}
\begin{frame}{Computational sequences}

\textbf{Computational sequences for LDLT decomposition}

$+$ -entries used (dane użyte)

$*$ -entries changed (dane zmienione)

\textbf{a) symetric Gaussian elimination}
$$
\begin{array}{lllll}

d_{11} \\
l_{21} & d_{22}\\
l_{31} & l_{32} & d_{33}\\
l_{41} & l_{42} & a_{43} & a_{44}\\
l_{51} & l_{52} & a_{53} & a_{54} & a_{55}

\end{array}
$$

\end{frame}
\begin{frame}{Computational sequences}

$+$ -entries used (dane użyte)

$*$ -entries changed (dane zmienione)

\textbf{b) row Doolittle LDLT decomposition}
$$
\begin{array}{lllll}
d_{11} \\
l_{21} & d_{22} \\
a_{31} & a_{32} & a_{33}\\
a_{41} & a_{42} & a_{43} & a_{44} \\
a_{51} & a_{52} & a_{53} & a_{54} & a_{55}
\end{array}
$$

\end{frame} 
\begin{frame}{Computational sequences}
$+$ -entries used (dane użyte)

$*$ -entries changed (dane zmienione)

\textbf{c) Crout LDLT decomposition}
$$
\begin{array}{lllll}
d_{11}\\
l_{21} & d_{22}\\
l_{31} & l_{32} & a_{33}\\
l_{41} & l_{42} & a_{43} & a_{44} \\
l_{51} & l_{52} & a_{53} & a_{54} &  a_{55}
\end{array}
$$
\end{frame}
\begin{frame}{Faktoryzacja blokowa}
\textbf{ 1) L,U - dowolne: }
$$
A=\left(\begin{array}{ll}
A_{11} & A_{12}\\
A_{21} & A_{22}
\end{array}\right)=\left(\begin{array}{ll}
L_{11} & 0\\
L_{21} & L_{22}
\end{array}\right)\cdot\left(\begin{array}{ll}
U_{11} & U_{12}\\
0 & U_{22}
\end{array}\right)
$$
$A_{11}$, . . . -square submatrices 

$L_{11}, L_{22}-$ lower triangular submatrices 

$U_{11}, U_{22}-$ upper triangular submatrices
\begin{flushright}
$
A_{11}=L_{11}\cdot U_{11}\ A_{21}=L_{21}\cdot U_{11}
$
\end{flushright}
\begin{flushright}
$
A_{12}=L_{11}\cdot U_{12}\ A_{22}=L_{21}\cdot U_{12}+L_{22}\cdot U_{22}
$
\end{flushright}

\end{frame}
\begin{frame}{Faktoryzacja blokowa}
Sposób wyznaczania $L$ i $U$:

\begin{itemize}
\item   \textbf{1.} $A_{11}=L_{11}\cdot U_{11}\rightarrow$

$L_{11}, U_{11}$ constructed by triangular factorization of $A_{11}$
\item \textbf{2.} $ L_{11}\cdot U_{12}=A_{12}\rightarrow$

columns of $U_{12}$ -backward substitution

\item \textbf{3.} $L_{21}\cdot U_{11}=A_{21}\rightarrow$

rows of $L_{21}$ -forward substitution 
\item \textbf{4.}  $\overset{\bullet}{A_{22}}=A_{22}-L_{21}\cdot U_{12}=L_{22}\cdot U_{22}$
\end{itemize}

\end{frame}
\begin{frame}{Faktoryzacja blokowa}
\textbf{2) L,U-unit triangular:}
$$
A=\left(\begin{array}{ll}
A_{11} & A_{12}\\
A_{21} & A_{22}
\end{array}\right)=\left(\begin{array}{ll}
A_{11} & 0\\
A_{21} & \overset{-}{A_{22}}
\end{array}\right)\cdot \left(\begin{array}{ll}
I & \overset{-}{U_{12}}\\
0 & I
\end{array}\right)
$$
$ A_{12}=A_{11}\cdot \overset{-}{U_{12}}\rightarrow$solution of sets of equations $ \overset{-}{A_{22}}=A_{22}-A_{21}\cdot \overset{-}{U_{12}}\rightarrow$matrix operations

\end{frame}
\begin{frame}{Faktoryzacja blokowa}
$$
A=\left(\begin{array}{ll}
A_{11} & A_{12}\\
A_{21} & A_{22}
\end{array}\right)=\left(\begin{array}{ll}
I & 0\\
\overset{\bullet \bullet}{L_{21}} & I
\end{array}\right)\cdot \left(\begin{array}{ll}
A_{11} & A_{12}\\
 0 & \overset{\bullet \bullet}{A_{22}}
\end{array}\right)
$$
$ \overset{\bullet \bullet}{L_{21}}\cdot A_{11}=A_{21}\rightarrow$solution of sets of equations $ \overset{\bullet \bullet}{A_{22}}=A_{22}-\overset{\bullet \bullet}{L_{21}}\cdot A_{12}\rightarrow$matrix operations

$($?? $)$ block-Schur complement

zachodzi $\overset{\bullet}{A_{22}}=\overset{-}{A_{22}}=\overset{\bullet \bullet}{A_{22}}$

(Zaniedbując błędy obliczeń obliczeń)
\end{frame}
\begin{frame}{Faktoryzacja blokowa}
Kolejne kroki obliczeniowe dla rozkładu $($?? $)$
$$
A=\left(\begin{array}{ll}
A_{11} & 0\\
A_{21} & \overset{-}{A_{22}}
\end{array}\right)\ \left(\begin{array}{ll}
I & \overset{-}{U_{12}}\\
0 & I
\end{array}\right)
$$
Równanie $\mathrm{A}\mathrm{x}=\mathrm{b}$ rozwiązujemy w następujący sposób: 



\end{frame}
\begin{frame}{Faktoryzacja blokowa}
$\hspace*{10mm} \overline{L}\cdot\overline{y}=b\  \hspace*{40mm}\overline{U}\cdot x=y$
\scalebox{0.85}{
$\left(\begin{array}{ll}
A_{11} & 0\\
A_{21} & \overline{A_{22}}
\end{array}\right)\left(\begin{array}{l}
y_{1}\\
y_{2}
\end{array}\right) = \left(\begin{array}{l}
b_{1}\\
b_{2}
\end{array}\right)\ \hspace*{10mm} \left(\begin{array}{ll}
I & \overline{U_{12}}\\
0 & I
\end{array}\right)\left(\begin{array}{l}
x_{1}\\
x_{2}
\end{array}\right) = \left(\begin{array}{l}
y_{1}\\
y_{2}
\end{array}\right)$ }

$\hspace*{33.5mm}\downarrow\ \hspace*{51mm} \downarrow$

$\hspace*{7mm}A_{11}\cdot y_{1}\ \hspace*{12.4mm} =\ \hspace*{3mm} b_{1}\ \hspace*{21.5mm} x_{1}\ \hspace*{15mm} = y_{1}-\overline{U_{12}}\cdot x_{2}$

$A_{21}\cdot y_{1}+\overline{A_{22}}\cdot y_{2}\hspace*{2.5mm} =\ \hspace*{3mm} b_{2}\ \hspace*{20.5mm} x_{2}\ \hspace*{23.5mm} y_{2} $

$\hspace*{8mm} A_{22}\cdot y_{2} \hspace*{11.5mm} =\ b_{2}-A_{21}\cdot y_{1}\ $

Kolejne kroki dla $($??$)$

Kolejne kroki  $($?? $)$
\begin{flushright}
{\it Zadanie}: Przeprowadzić odpowiednie obliczenia
\end{flushright}

\end{frame}