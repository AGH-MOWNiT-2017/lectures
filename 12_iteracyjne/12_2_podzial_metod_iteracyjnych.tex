\section{12.2 Podział metod iteracyjnych}

\begin{frame}{Podział metod iteracyjnych}
  \begin{block}{\textbf{Metody stacjonarne} (\emph{stationary})}
    \begin{itemize}
      \item stałe współczynniki macierzy iteracyjnej,
      \item starsze,
      \item proste w zrozumieniu i implementacji,
      \item na przykład metody Jacobiego, G--S (SR), SOR, SSOR.
    \end{itemize}
  \end{block}
\end{frame}

\begin{frame}{}
  \begin{block}{\textbf{Metody niestacjonarne} (\emph{nonstationary})}
    \begin{itemize}
      \item współczynniki macierzy iteracyjnej zmieniają się w kolejnych krokach iteracji,
      \item oparte na idei
      \begin{itemize}
        \item ciągu wektorów ortogonalnych (CG, MINRES ...),
        \item wielomianów ortogonalnych (metoda Czybyszewa),
      \end{itemize}
      \item stosunkowo nowe,
      \item trudniejsze w zrozumieniu i implementacji,
      \item szybciej zbieżne.
    \end{itemize}
  \end{block}
\end{frame}

\begin{frame}{}
  \begin{block}{}
    \emph{iterate} -- przybliżenie rozwiązania w kolejnej iteracji,
    \newline \emph{residual} -- $r=Ay-b$,
    \newline \emph{preconditioner, preconditioning matrix:} macierz transformująca układ równań do postaci o lepszych własnościach spektralnych
  \end{block}
\end{frame}
