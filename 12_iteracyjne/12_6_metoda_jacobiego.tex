\section{12.6 Metoda Jacobiego}

\begin{frame}{Metoda Jacobiego}
  $$
  A=D+(L+U)
  \begin{cases}
  M=I-D^{-1}A\\
  W=D^{-1}b
  \end{cases}
  $$
  gdzie: L -- poddiagonalna; P -- naddiagonalna;\\
  D = B -- diagonalna, z diagonalnych elementów macierzy A.
  $$Ax = (D+(L+U))x = b \implies Dx = -(L+U)x + b$$
  Korzystając z zależności
  $$\boxed{Dx^{(t+1)}= -(L+U)x^{(t)}+b}$$
  otrzymujemy wzór roboczy:
  $$x_i^{(t+1)}=\frac{1}{a_{ii}}[b_i-\sum_{j=1,j\neq i}^{n} a_{ij}x_j^{(t)}]\  ;\  a_{ii} \neq 0, \forall i,$$
\end{frame}

\begin{frame}
  \begin{block}{Procedura przestawiania wierszy}
    \begin{itemize}
      \item[1.] spośród kolumn z $a_{ii} = 0$ wybieramy tą, która ma najwięcej zer,
      \item[2.] w tej kolumnie wybieramy element o max $|a_{ii}|$ i przestawiamy wiersze tak, aby znalazł się on na diagonali,
      \item[3.] Powtarzamy 1. i 2.
    \end{itemize}
  \end{block}
\end{frame}

\begin{frame}{}
  \begin{block}{Modelowe zadanie}
    równanie Poissona
    \\2-D
    \\w.b. $\Rightarrow\phi\equiv 0$
    \\siatka przestrzenna \emph{N x N}
  \end{block}

  \begin{block}{Dla modelowego zadanie -- met. Jacobiego:}
    $$\rho - cos\frac{\pi}{N}\approx 1 0 \frac{\pi^2}{N^2}, \lambda _J = \rho$$
  \end{block}
\end{frame}

\begin{frame}{}
  \begin{block}{Charakterystyka metody Jacobiego}
    \begin{itemize}
      \item prosta,
      \item znaczenie dydaktyczne,
      \item wolnozbieżna,
      \item nie wykorzystuje całej, dostępnej w danym kroku informacji,
      \item pamiętane $x^{(t)}$ i $x^{(t+1)}$,
      \item zbieżna dla A silnie diagonalnie dominujących,
      $$\text{wierszowo : }|a_{ii} > \sum_{j=1,j\neq i}^{n} |a_{ij},$$
      $$\text{kolumnowo : }|a_{ii} > \sum_{j=1,j\neq i}^{n} |a_{ji},$$
    \end{itemize}
  \end{block}
\end{frame}
