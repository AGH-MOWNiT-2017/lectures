\section{12.3 Istota metod iteracyjnych dla Ax=b}

\begin{frame}{Istota metod iteracynych dla $A\mathrm{x}=b$}
    \begin{center}
      $A \cdot \mathrm{x}=b$
    \end{center}
    \begin{itemize}
      \item gdzie A jest macierzą n x n;
      \item $\mathrm{x}$ -- wektor n niewiadomych;
      \item b -- wektor danych (źródeł)
    \end{itemize}
\end{frame}

\begin{frame}
  \begin{block}{\textbf{Rozkład:}}
      $$A=B+R$$
    \begin{itemize}
      \item B -- macierz dla, której łatwo stworzyć $B^{-1}$
      \item R -- pozostałość
    \end{itemize}
    $$B \cdot x=-R \cdot x+b$$
  $$\boxed{B \cdot x=-(A-B) \cdot x+b}$$
  \end{block}
\end{frame}

\begin{frame}{}
    \textbf{Metody iteracyjne dla Ax = b (mesh relaxation methods) polegają na:}
    \begin{itemize}
      \item odgadnięciu wektora początkowego $x^{(o)}$
      \item generowaniu ciągu iteracyjnego $x^{(t)}$ według postulowanego wzoru:
    \end{itemize}
	$\newline$
    $$B \cdot x^{(t+1)}=-(A-B) \cdot x^{(t)}+b$$
    $$x^{(t+1)}=\underbrace{-B^{-1} \cdot (A-B)}_{I-B^{-1} \cdot A=M \quad \text{M - iteration matrix}} \cdot x^{(t)}+\underbrace{B^{-1}\cdot b}_{W}$$
\end{frame}

\begin{frame}{}
    \textbf{Różne B $\rightarrow$ rodzina metod iteracyjnych:}
    $$\boxed{x^{(t+1)}=M \cdot x^{(t)}+B^{-1} \cdot b}\quad(\star)$$
    Warunek \emph{zgodności} formuły iteracyjnej z szukanym rozwiązaniem
    $$\lim_{t\to\infty} x^{(x+1)}= \lim_{t\to\infty}  (M \cdot x^{(t)}+B^{-1} \cdot b)$$
\end{frame}

