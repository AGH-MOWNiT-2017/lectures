\section{12.3 Istota metod iteracyjnych dla Ax=b}

\begin{frame}{Istota metod iteracynych dla Ax=b}
    \begin{center}
      $A\cdot x=b\quad(1)$
    \end{center}
    \begin{itemize}
      \item gdzie A jest macierzą n x n;
      \item x -- wektor n niewiadomych;
      \item b -- wektor danych (źródeł)
    \end{itemize}
\end{frame}

\begin{frame}
  \begin{block}{\textbf{Rozkład:}}
      $$A=B+R$$
    \begin{itemize}
      \item B -- macierz dla, której łatwo $B^{-1}$
      \item R -- pozostałość
    \end{itemize}
    $\Rightarrow$
    $$B\cdot x=-R\cdot x+b$$<++>
  $$\boxed{B\cdot x=-(A-B)\cdot x+b}\quad(2)$$
  \end{block}
\end{frame}

\begin{frame}{}
    \textbf{Metody iteracyjne dla Ax = b (mesh relaxation methods) polegają na:}
    \begin{itemize}
      \item odgadnięciu wektora początkowego $x^{(o)}$
      \item generowaniu ciągu iteracyjnego $x^{(t)}$ wg. postulowanego wzoru:
    \end{itemize}
    $$B\cdot x^{(t+1)}=-(A-B)\cdot x^{(t)}+b \hskip \textwidth minus \textwidth (12.1)$$
    $$x^{(t+1)}=\underbrace{-B^{-1}\cdot (A-B)}_{I-B^{-1}\cdot A=M \quad M\text{ - iteration matrix }} \cdot x^{(t)}+\underbrace{B^{-1}\cdot b}_{W}\quad(3) \hskip \textwidth minus \textwidth (12.2)$$
\end{frame}

\begin{frame}{}
    \textbf{Różne B $\rightarrow$ rodzina metod iteracyjnych:}
    $$\boxed{x^{(t+1)}=M\cdot x^{(t)}+B^{-1}\cdot b}\quad(4)$$
    Warunek \emph{zgodności} formuły iteracyjnej z szukanym rozwiązaniem
    $$\lim_{t\to\infty} x^{(x+1)}= \lim_{t\to\infty}  (M\cdot x^{(t)}+B^{-1}\cdot b) \quad (5)$$
\end{frame}

