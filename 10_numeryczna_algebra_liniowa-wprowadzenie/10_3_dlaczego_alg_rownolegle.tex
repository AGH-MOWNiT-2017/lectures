\section{Dlaczego algorytmy równoległe?}

\begin{frame}{Dlaczego algorytmy równoległe?}

\begin{exampleblock}{Przykład -- przewidywanie pogody (globalnie):}
rozwiązywanie równania Naviera-Stokesa na siatce 3D wokół Ziemi

Zmienne:

  $\left. \parbox{10em}
{\begin{itemize}
   \item \text{temperatura}
   \item \text{ciśnienie}
   \item \text{wilgotność}
   \item \text{prędkość wiatru}
  \end{itemize}}
\right \} \text{równanie Naviera-Stokesa (6 zmiennych)}$
\end{exampleblock}

Obliczenia:

$\left. \parbox{15em}
{\begin{itemize}
   \item \text{elementarna komórka 1 km}
   \item \text{10 warstw}
  \end{itemize}}
\right \} 5\cdot10^9\text{ komórek}$

\begin{itemize}
    \item w każdej komórce:
	$6\cdot 8 \text{ Bytes} \Rightarrow 2\cdot 10^{11} \text{ Bytes} = 200 \text{ Gbytes}$,

    \item w każdej komórce 100 operacji fp,
    \item obliczenia 1 kroku czasowego 1 min $\Rightarrow \frac{100\cdot5\cdot10^9}{60} = 8 \text{ GFLOPS}$
 \end{itemize}

\end{frame}
