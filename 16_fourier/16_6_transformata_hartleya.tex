\section{Transformata Hartley'a}
%%%%%%%%%%%%%%%%
\begin{frame}[allowframebreaks]{Transformata Hartley'a}
	\begin{block}{Tranformata Fouriera}
	\[
		F(f) = \int\limits_{-\infty}^{\infty} X(t) e^{-i2\pi ft} dt
	\]
	\[
		X(t) = \int\limits_{-\infty}^{\infty} F(f) e^{i2\pi ft} df
	\]
	\[
		c_j = \frac{1}{n} \sum\limits_{k = 0}^{n-1} X(t_k) e^{-i2\pi j \frac{k}{n}}
	\]
	\[
		X(t_k) = \sum\limits_{k = 0}^{n-1} c_j e^{i2\pi j \frac{k}{n}}
	\]
	\end{block}
	\begin{block}{Transformata Hartley'a}
	\[
		H(f) = \int\limits_{-\infty}^{\infty} X(t) \cas(2\pi ft) dt
	\]
	\[
	X(t) = \int\limits_{-\infty}^{\infty} H(f) \cas(2\pi ft) dt
	\]
	\end{block}
	gdzie: $\cas(x) = \cos(x) + \sin(x)$
	\begin{block}{Wersja dyskretna HT}
	\[
		H_j = \frac{1}{n} \sum\limits_{k = 0}^{n-1} f(t_k) \cdot \cas \bigg( \frac{2\pi jk}{n} \bigg)
	\]
	\[
		f(t_k) = \sum\limits_{j=0}^{n-1} H_j \cdot \cas \bigg( \frac{2\pi jk}{n} \bigg)
	\]
	\end{block}
\end{frame}
\begin{frame}{Własności HT}
	\begin{block}{16.30}
		\centering
		\begin{enumerate}[1$^\circ$]
			\item $F_r(j) = H(j) + H(n-j)$ \\ $F_i(j) = H(j) + H(n-j)$ \\
			\item power spectrum: $P_s(j) = [H^2(j) + H^2(n-j)] \cdot \frac{1}{2}$ \\
			\item $f_1(t) \ast f_2(t) = \int\limits_{-\infty}^{\infty} f_1(\tau) \cdot f_2(t - \tau) d\tau$ \hfill - splot \\
			$f_1(t) \ast f_2(t) = F_1(f) \cdot F_2(f)$ \\
			$f_1(t) \ast f_2(t) = H_1(f) \cdot H_{2e}(f) + H_1(-f) \cdot H_{2o}(f)$
		\end{enumerate}
	\end{block}
	dla oznaczeń:
	\begin{itemize}
		\item r - real, i - imaginary \\
		\item o - odd, e - even \\
		\item F, f - Fourier, H - Hartley
	\end{itemize}
\end{frame}
\begin{frame}{Szybka Transformacja Hartley'a}
	\begin{block}{Fourier}
	\[
		F_j = F_{1j} + F_{2j} \cdot e^{-i\frac{2\pi j}{n'}} , n' = \frac{n}{2}
	\]
	\end{block}
	\begin{block}{Hartley}
	\[
		H_j = H_{1j} + H_{2j} \cdot \cos \bigg( \frac{2\pi j}{n'} \bigg) + H_2(n'-1) \cdot \sin \bigg( \frac{2\pi j}{n'} \bigg)
	\]
	\end{block}
\end{frame}
\begin{frame}{Biliografia}
	\begin{itemize}
		\item R.V.L. Hartley: A more symetrical Fourier analysis applied to transmission problems, Proc. IRE, 30 (1942) 144, \\
		\item R.N. Bracewell: The fast Hartley transform, Proc. IEEE 72 (1984) 1010 (No 8), \\
		\item M.A. O'Neill: Faster than fast Fourier, Byte, April 1988, p.293. istotna różnica: zamiast $\underbrace{e^{-x \cdot i}}_{\text{zespolone}}$ mamy $\underbrace{\cas(x)}_{\text{rzeczywiste}}$ ($\implies$ ilość operacji arytmetycznych i pamięc)
	\end{itemize}
\end{frame}
\begin{frame}{FFT - przydatna w:}
	\begin{enumerate}
		\item analiza spektralna \\
		\item projektownie efektywnch algorytmów
		\begin{itemize}
			\item iloczyn wielomianów $\to$ splot 2 wektorów \\
			\item szybki binarny algorytm mnożenia liczb całkowitych
			(m. Sch\"{o}nhagego - Strassena)
		\end{itemize}
		\item $\to$ A.V. Aho, J.E. Hopcroft, J.D. Ullman: \\
		Projektowanie i analiza algorytmów komputerowych. PWN, 1983 (1974)
	\end{enumerate}
\end{frame}