\section{Interpolacja trygonometryczna}
%%%%%%%%%%%%%%%%
\begin{frame}{Interpolacja trygonometryczna}
	\begin{itemize}
		\item f. okresowe: $g(y + L) = g(y)$; okres: L \\
		\item f. trygonometryczne: $L \to 2\pi, x = \frac{2\pi}{L} \cdot y, f(x) = g(\frac{x \cdot L}{2\pi})$	
	\end{itemize}
	szukamy wielomianu (trygonometrycznego)
	\[
		t_{n-1}(x) = \sum\limits_{\sfrac{j}{0}}^{n-1} c_j \cdot e^{ijx}
		\tag{16.12}
	\]
	który w \underline{n punktach $x_k \in (0, 2\pi]$} przyjmuje te same wartości, co interpolowana funkcja
	\\ $t_{n - 1}(x_k) = f(x), k = 0, 1, \dots, n - 1$
\end{frame}
%%%%%%%%%%%%%%%%
\begin{frame}[allowframebreaks]{Dowód}
	\begin{theorem}
		Zadanie interpolacji trygonometrycznej ma jednoznaczne rozwiązanie.
	\end{theorem}
	\begin{proofs}
		\[
			\sum\limits_{\sfrac{j}{0}}^{n-1} c_j \cdot \underbrace{(e^{ix_k})^j}_{Z_k} = f(x_k), k = 0, 1, \dots, n-1
			\tag{16.13}
		\]
		$\implies$ macierz układu: macierz Vandermonde'a jest nieosobliwa, \\ jeżeli \underline{$x_k$ są różne}, i:
		\[
			\det Z = \prod_{k \neq L} (Z_k - Z_L)
			\tag{16.14}
		\]
	\end{proofs}
	\begin{proofs}[\proofname\ (Cont.)]
		W praktyce - ważny przypadek szczególny: \\
		n węzłów równoodległych, \underline{$x_k = \frac{2\pi}{n}\cdot k$}, k = 0, 1, \dots, n - 1 \\
		\underline{funkcje $e^{ijx}$}, j = 0, 1, \dots, n - 1 tworzą \underline{układ ortogonalny} w sensie iloczynu skalarnego zdefinowanego: \\
		\[
			\langle f|g \rangle = \sum\limits_{\sfrac{k}{0}}^{n-1} f(x_k) \cdot g^* (x_k), x_k = \frac{2\pi}{n} \cdot k, k = 0, 1, \dots, n-1
			\tag{16.15}
		\]
		\[
			\langle e^{ijx}|e^{ilx} \rangle = \sum\limits_{\sfrac{k}{0}}^{n-1} e^{ijx_k}  \cdot e^{-ilx_k} = \sum\limits_{\sfrac{n}{0}}^{n-1} e^{i(j - l) \frac{2\pi k}{n}}\stackrel{\Delta}{=}
			\tag{16.16}
		\]
	\end{proofs}
	\begin{proofs}[\proofname\ (Cont.)]
		\underline{dla $j = l$}
		\[
			\stackrel{\Delta}{=} \sum\limits_{\sfrac{k}{0}}^{n-1} e^0 = n
			\tag{16.17}
		\]
		\underline{dla $j \neq l$}
		\[
			\stackrel{\Delta}{=} \sum\limits_{\sfrac{k}{0}}^{n-1} \underbrace{e^{\frac{i(j - l)2\pi}{n}}}_{\substack{
				\text{-szereg geom.} \\
				\text{-n-wyrazów}
			}} \cdot k = \frac{a_0 - a_n \cdot q}{1 - q} = \frac{1 - e ^{\overbrace{\frac{i(j - l)2\pi}{n} \cdot n}^{=1}}}{1 - e^{\frac{i(j -  l)2\pi}{n}}} = 0
			\tag{16.18}
		\]
		$a_0 = 1; q = e^{\frac{i(j -  l)2\pi}{n}}$
	\end{proofs}
	\begin{proof}[\proofname\ (Cont.)]
		\[
			\langle e^{ijx}|e^{ilx} \rangle = \sum\limits_{\sfrac{k}{0}}^{n - 1}e^{ijx_k}  \cdot e^{-ilx_k} = 
			\begin{cases}
				0 & j \neq l \\
				n & j = l
			\end{cases}
			\tag{16.19}
		\]
	\end{proof}
\end{frame}
%%%%%%%%%%%%%%%%
\begin{frame}[allowframebreaks]{Interpolacja trygonometryczna}
	Jakie powinny być współczynniki wielomianu interpolacyjnego $c_j$?
	\begin{enumerate}[1$^\circ$]
		\item
		\begin{align*}
			\langle t_{n-1}(x)|e^{ilx} \rangle = \Bigg\langle \sum\limits_{\sfrac{j}{0}}^{n-1} c_j \cdot e^{ijx} \Bigg|e^{ilx} \Bigg\rangle = \sum\limits_{\sfrac{j}{0}}^{n-1} c_j \cdot n \cdot \delta jl = \\ = c_l \cdot n \implies \underline{c_l = \frac{1}{n} \langle t_{n-1}(x)|e^{ilx} \rangle}
			\tag{16.20}
		\end{align*}
		\item
		\begin{align*}
			\langle t_{n-1}(x)|e^{ilx} \rangle \underbrace{=}_{(**)} \sum\limits_{\sfrac{k}{0}}^{n-1}  \underbrace{t_{n-1}(x_k)}_{\substack{\text{=wartości} \\  \text{funkcji}}} \cdot e^{-ilx_k} = \\ = \sum\limits_{\sfrac{k}{0}}^{n-1} f(x_k) \cdot e^{-ilx_k} = \underline{\langle f(x)|e^{ilx} \rangle}
		\end{align*}
	\end{enumerate}
	- def. iloczynu (przypadek dyskretny) \\
	czyli:
	\[
		\underline{c_j = \frac{1}{n} \sum\limits_{\sfrac{k}{0}}^{n-1} f(x_k) \cdot e^{-ilx_k}, j = 0, 1, \dots, n-1}
		\tag{16.21}
	\]
	\begin{flushleft}
		$\implies$ \\
		\underline{\textit{analiza Fouriera:}} \\
		dla danych liczb zespolonych $f(x_k), k = 0, 1,\dots, n - 1$ szukamy $c_j, j = 0, 1, \dots, n-1$ \\
		\bigskip
		\underline{\textit{synteza Fouriera:}} \\
		mając liczby $c_j, j = 0, 1, \dots, n-1$ szukamy
		\[
			(k) = \sum\limits_{\sfrac{j}{0}}^{n - 1} c_j \cdot e^{ij\frac{2\pi}{n} \cdot k}, k = 0, 1, \dots, n-1
			\tag{16.22}
		\]
	\end{flushleft}
	\begin{flushleft}
		$\implies$ obie:
		\begin{itemize}
			\item dyskretne
			\item wzajemnie odwrotne
		\end{itemize}
		klasyczny algorytm: $(f = A \cdot C)$
		\[ n^2
		\begin{cases}
			\text{-zespolonych mnożeń,} \\
			\text{-zespolonych dodawań,} \\
			\text{-obliczeń } e^{-i\frac{2\pi kj}{n}}
		\end{cases}
		\tag{16.23}
		\]
	\end{flushleft}	
\end{frame}