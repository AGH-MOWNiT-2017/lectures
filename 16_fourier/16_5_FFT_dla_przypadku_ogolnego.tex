\section{FFT dla przypadku ogólnego}
%%%%%%%%%%%%%%%%
\begin{frame}{FFT dla przypadku ogólnego I}
	\[
		c_j = \frac{1}{n} \sum\limits_{k = 0}^{n-1} f(x_k)e^{-ijx_k}, x_k = \frac{2\pi}{n} \cdot k;\ j = 0, 1, \dots, n-1
	\]
	$\omega = e^{-i\frac{2\pi}{n}}, a_k = \frac{1}{n} f(x_k)$
	\[
		\underline{c_j = \sum\limits_{k = 0}^{n-1} a_k \cdot \omega^{jk}}, j = 0, 1, \dots, n-1
	\]
	dla $n = r_1 \cdot r_2 \cdot \dots \cdot r_p, r_i$ liczby całkowite
	\[
		c_j = \sum\limits_{k = 0}^{n-1} a_k \cdot e^{-i2\pi \overbrace{j \cdot \frac{k}{n}}^{(*)}}
		\tag{16.29}
	\]
	(*) wyrazić przez $r_1, r_2, \dots, r_p$
\end{frame}
%%%%%%%%%%%%%%%%
\begin{frame}{FFT dla przypadku ogólnego II}
	\underline{każda liczba całkowita j, $0 \leq j \leq n-1$ ma jedyną reprezentację}:
	\[
		j = \alpha_1 \cdot (r_2 \cdot r_3 \cdots r_p) + \alpha_2 \cdot (r_3 \cdot r_4 \cdots r_p) + \dots + \alpha_{p-1} \cdot r_p + \alpha_p \cdot 1
	\]
	co, dla uproszczenia, można zapisać:
	\[
		j = \sum\limits_{i = 1}^{p} \alpha_i \cdot n_i
	\]
	gdzie
	\[
		n_r = r_{r+1} \cdot r_{r+2} \cdots r_p = \prod\limits_{i = (r+1)}^{p} r_i
	\]
	$n_p = 1, n_0 = n$ \\
	zaś współczynniki:
	\[
		\alpha_i \in \{0, 1, \dots, r_i - 1\}
	\]
\end{frame}
%%%%%%%%%%%%%%%%
\begin{frame}{FFT dla przypadku ogólnego III}
	Przydatne będą też:
	\[
		j_r = \sum\limits_{i = (r+1)}^{p} \alpha_i \cdot n_i,\  j_0 = j
	\]
	przy czym:
	\[
		\begin{rcases}
			\alpha_i - \text{iloraz} \\
			j_i - \text{reszta}
		\end{rcases}
		\text{z dzielenia } \frac{j_{i-1}}{n_i}
	\]
\end{frame}
%%%%%%%%%%%%%%%%
\begin{frame}{FFT dla przypadku ogólnego IV}
	\underline{Dla każdej liczby całkowitej k, $0 \leq k \leq n-1$} \\
	\underline{istnieje jedyna reprezentacja ilorazu $\frac{k}{n}$}:
	\[
		\frac{k}{n} = \frac{l_1}{\underbrace{r_1 \cdot r_2 \cdots r_p}_{n_0}} + \frac{l_2}{\underbrace{r_2 \cdot r_3 \cdots r_p}_{n_1}} + \cdots + \frac{l_{i+1}}{\underbrace{r_{i+1} \cdots r_p}_{n_i}} + \cdots + \frac{l_p}{r_p} = \underline{\sum\limits_{i = 0}^{p-1} \frac{l_{i+1}}{n_i}}
	\]
	\[
		l_i \in \{0, 1, \dots, r_i - 1\}
	\]
	\[
		k_r = n_r \cdot \sum\limits_{i = r}^{p-1} \frac{l_{i+1}}{n_i}, k_0 = k
	\]
	\[
		\begin{rcases}
			k_i - \text{iloraz} \\
			l_i - \text{reszta}
		\end{rcases}
		\text{z dzielenia } \frac{k_{i-1}}{r_i}
	\]
	\[
		j \cdot \frac{k}{n} = j_0 \cdot \frac{k_0}{n_0} = \Bigg( \sum\limits_{i = 1}^{p} \alpha_i \cdot n_i \Bigg) \cdot \Bigg( \sum\limits_{r = 0}^{p-1} \frac{l_{r+1}}{n_r} \Bigg) = \sum\limits_{r = 0}^{p-1} \sum\limits_{i = 1}^{p} l_{r+1} \cdot \alpha_i \cdot \frac{n_i}{n_r} \stackrel{\Delta}{=}	
	\]
\end{frame}
%%%%%%%%%%%%%%%%
\begin{frame}{FFT dla przypadku ogólnego V}
	\[
		\underline{\overline{\frac{n_i}{n_r} = 
		\begin{cases}
			\text{l. całkowita}, & i \leq r \\
			\text{ułamek}, & i > r
		\end{cases}}}
	\]
	\[
		\stackrel{\Delta}{=} M + \sum\limits_{r = 0}^{p-1} \frac{l_{r+1}}{n_r} \cdot \Bigg( \underbrace{\sum\limits_{i = (r+1)}^{p} \alpha_i \cdot n_i}_{j_r} \Bigg) = M +  \sum\limits_{r = 0}^{p-1} \frac{l_{r+1} \cdot j_r}{n_r}
	\]
	\[
		\underline{\omega^{jk}} = e^{-i2\pi \bigg( M +  \sum\limits_{r = 0}^{p-1} \frac{l_{r+1} \cdot j_r}{n_r} \bigg)} = e^{-i2\pi \sum\limits_{r = 0}^{p-1} \frac{l_{r+1} \cdot j_r}{n_r}} =
	\]
	\[
		= \prod\limits_{r = 0}^{p-1} {\underbrace{\bigg( e^{-\frac{i2\pi}{n_r}} \bigg)}_{\omega_r}}^{l_{r+1} \cdot j_r} = \underline{\prod\limits_{r = 0}^{p-1} \omega_r^{l_{r+1} \cdot j_r}}
	\]
\end{frame}
%%%%%%%%%%%%%%%%
\begin{frame}{FFT dla przypadku ogólnego: przykład}
	\begin{exampleblock}{Przykład}
	$p = 3, n = r_1 \cdot r_2 \cdot r_3$ \\
	$n_0 = r_1r_2r_3 ; n_1 = r_2r_3 ; n_2 = r_3 ; n_3 = 1$ \\
	$j = \alpha_1 \cdot (r_2 \cdot r_3) + \alpha_2 \cdot r_3 + \alpha_3 ; j_0 = j ; j_1 = \alpha_2  \cdot r_3 + \alpha_3 ; j_2 = \alpha_3$ \\
	w miejsce $k \to k(l_1, l_2, l_3)$ zgodnie z $\frac{k}{n} = \sum\limits_{i = 0}^{p-1} \frac{l_{i+1}}{n_i}$ \\
	oznaczamy:
	\[
		c^{(0)}(l_1, l_2, l_3) = a_k
	\]
	\[
		c_j = \sum\limits_{k = 0}^{n-1} a_k \cdot \omega^{jk} = \sum\limits_{l_1 = 0}^{r_1 - 1} \sum\limits_{l_2 = 0}^{r_2 - 1} \sum\limits_{l_3 = 0}^{lr_3 - 1} c^{(0)}(l_1, l_2, l_3) \cdot \omega^{jl_1} \omega_1^{j_1l_2} \omega_2^{j_2l_3}
	\]
	\end{exampleblock}
\end{frame}
%%%%%%%%%%%%%%%%
\begin{frame}{FFT dla przypadku ogólnego: przykład cd}
	\begin{exampleblock}{Przykład cd}
	wyliczamy sumy kolejno, poczynając od wewnętrznej:
	\[
		\sum\limits_{l_3 = 0}^{r_3 - 1} c^{(0)}(l_1, l_2, l_3) \omega^{j_2l_3} = c^{(1)}(l_1, l_2,\alpha_3),
	\]
	\[
		l_1, \stackrel{\uparrow\nearrow}{l_2} - \text{ustalone} ; j_2 = j_2(\alpha_3) - \text{tylko}
	\]
	\[
		\sum\limits_{l_2 = 0}^{r_2 - 1} c^{(1)}(l_1, l_2, \alpha_3)\omega_1^{j_1l_2} = c^{(2)}(l_1,\alpha_2,\alpha_3),
	\] 
	\[
		\sum\limits_{l_1 = 0}^{r_1 - 1} c^{(2)}(l_1, \alpha_2, \alpha_3) \omega^{jl_1} = c_j.
	\]
	(przyp. ogólny $\to$ szczególny)
	\end{exampleblock}
\end{frame}