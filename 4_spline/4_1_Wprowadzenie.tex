\section{Wprowadzenie}
	\begin{frame}{Wprowadzenie}
    	\textit{Spline} - listwa traserska, krzywik ustawny; wyginany tak, aby
        przechodził przez zadane punkty $\{ x_{i},y_{i} \} \newline \newline$
		\textbf{Teoria elastyczności} - liniał przyjmuje kształt minimalizujący
        energię
        potencjalną $E_{p} \newline$
        $s(x)$ - funkcja $\newline$
        $s''(x)$ - krzywizna, dx - długość łuku $\newline$
        $s,\ s',\ s'' $ - ciągłe w $[x_{i},x_{n}] \newline$
        local curvature: $\frac{f''(x)}{(1+f'(x))^{\frac{5}{2}}}$
        \[
        	\min ! \int_{x_{i}}^{x_{n}}\underbrace{[s''(x)]^{2}dx}_{E_{p}}
            \Rightarrow w_{3}(x)
            \ \ \textrm{na} \ \ [x_{i},x_{i+1}]
        \]
        \[
        	\textrm{"strain energy"} = \int_{a}^{b} 
            \frac{(f''(x))^{2}}{(1+f'(x))^{\frac{3}{2}}}dx
        \]
        
	\end{frame}
    %%%%%%%%%%%%%%%%%%%%%%%%%%%%%%%%%%%%%%%%%%%%%%%%%%
    \begin{frame}{Definicja funkcji sklejanej}
    	\begin{exampleblock}{}
    		Funkcję $s(x) = s(x, \Delta n)$ określona na $[a,b]$ nazywamy funkcją 						sklejaną stopnia m $(m\geq1)$ jeżeli:
            \begin{itemize}
            \item $s(x)$ jest wielomianem stopnia $ \leq m$ na każdym $[x_{i},x_{i+1}]$
            \item $s(x) \in C^{m-1}[a,b]\rightarrow s(x)$ i jej $(m-1)$ pochodnych są ciągłe 				na $[a,b]$
            \end{itemize}
            $\newline$
            $\Delta n$ - podział $[a,b]$ na (n-1) podprzedziałów przez węzły: 
            $a=x_{1}<x_{2}<...<x_{i}<...<x_{n}=b$
    	\end{exampleblock}
		
    \end{frame}